%%  This file is really -*-LaTeX-*-
%%          Project:  ms280690/Thesis-Proposal
%%             File:  toDo.tex
%%       Created by:  David Casperson
%%          Created:  Sun Oct  4 11:34:28 2015
%%
%%      Description:  Work to do.

\begin{scope}
\nolinenumbers
% ---------------------------------------------------------------------------
\chapter{Editing To Do}\label{chap:to-do}
% ---------------------------------------------------------------------------

\textit{\color{red} This chapter needs to be removed from the final
  work.}

\section{Conventions for section and chapter capitalization}
\begin{description}
\item [Chapter titles]
  \dots should be entirely capitalized, excepting small words like
  ``\textit{of}'',
  ``\textit{an}'',
  ``\textit{the}'',
  and so on.
\item [Section, Sub-section, and lower titles]
  \dots should have a leading capital only, excepting words 
  that are normally capitalized in text.
\item [Tables, Figures, and Listings]
  \dots should have a leading capital only, excepting words 
  that are normally capitalized in text.
\end{description}

\section{The ``To Do'' list}

\begin{enumerate}
\item [\textbf{Meeting on 5th Novemeber 2015}]
  (\textit{Items merged into list below.})

\item [\textbf{Both}]

\item
  Decide on which version of \codeLibrary{ghc} this compiles against, and ensure that the code does indeed compile 
  against that version (\Verb!ErrorT! versus \Verb!ExceptT! and the like).

\item [\textbf{Mehul}]

\item (David: I did this.)

\item
  Fix the capitalization of chapter,
  section, sub-section, and paragraph titles according to the rules above.

\item
  Write ``about this chapter'' and ``recapituation'' sections for all chapters.

\item
  (David: I did this.)

\item
    Move material from \S~\ref{sec:existing-work-by-others} to \S~\ref{sec:things-fixed}. 

  \item

Move material from Chapters \ref{chap:quasiquotation} and~\ref{chap:metasyntacticvariables} to ``Related Concepts''.    
    

\item
  Watch out for sentences that begin with ``But''.

\item
  Write enough of Chapters \ref{proto1}--\ref{proto4} that we can decide
  what material is needed in Chapters~\ref{chap:relatedWork}.

\item
  Rewrite \S~\ref{sec:exec-models-prolog} as suggested at the top of that section. 

\item
  Abstract is too long.  (Guidelines limit it to 150 words.  Currently
  at 275-ish.)



\item [\textbf{David}]
\item Clean up the non-exclusive license page in unbcthesis.cls
\item Review Chapter 7
\item Review Chapter 8
\item ``\textit{IO}'' or ``\textit{I/O}''?  Check!
\item Check spaces before opening parentheses.
\item Search for variants on ``\textit{metasyntactic}'', such as
  \begin{compactitem}
  \item
    ``\textit{meta syntactic}''
  \item
    ``\textit{meta-syntactic}''
  \end{compactitem}
\item Check for ``add on'' versus ``add-on'' and decide which is correct.
\end{enumerate}

\section{Thoughts on Chapter~\ref{proto1}}\label{sec:thoughts-chapt-proto1}

\subsection{Thoughts on Chapter~\ref{proto1}
  code}\label{sec:thoughts-chapt-proto1-code}

\begin{itemize}
\item I think that a lot of code cleanup ought to happen.  For instance, in
  Listing~\vref{tab:varsToDictM}, \haskellConstruct{variableExtractor} and
  \haskellConstruct{variableSet} are not used in the subsequent code.

\item 
  Without some actual tests we are missing the code that shows how a test
  case is built.

\item
  Names like \haskellConstruct{f1} are hacking.
\end{itemize}


\section{A List of Colloquialisms}\label{sec:list-colloquialisms}
\begin{enumerate}
\item
  ``\textit{throws light on}''
\item
  ``\textit{another new kid on the block}''.
\item
  Some expressions indicating change over time, for instance:
  \begin{compactitem}
  \item
    ``\textit{day by day}''
  \item
    ``\textit{over the years}''
  \item
    ``\textit{many times}''
  \item
    ``\textit{from time to time}''
    \dots
  \end{compactitem}
  are not in themselves problematic, but can be replaced by
  more precise language.
\item
  ``\textit{flipping the coin to the other side}''---perhaps ``\textsl{conversely}''? 
\item 
  ``\textit{bringing \dots{} onto the same plate.}''
\item
  ``\textit{job at hand}''
\item
  ``\textit{hot topic}''
\item
  ``\textit{faded out}''  (when used metaphorically).
\item
  ``\textit{adding fuel to the fire }''
\item
  ``\textit{push the ideas forward}'' (do you mean popularize?).
\end{enumerate}

\end{scope}
