%%  This file is really -*-LaTeX-*-
%%          Project:  ms280690/Thesis-Proposal
%%             File:  toDo.tex
%%       Created by:  David Casperson
%%          Created:  Sun Oct  4 11:34:28 2015
%%
%%      Description:  Work to do.

\begin{scope}
\nolinenumbers
% ---------------------------------------------------------------------------
\chapter{Editing to do}\label{chap:to-do}
% ---------------------------------------------------------------------------

\textit{\color{red} This Chapter needs to be removed from the final
  work.}

\textbf{Meeting on 5th Novemeber 2015} 
\begin{enumerate}
\item Write about this chapter and chapter conclusion for all chapters

\item Till haskell why haskell chapter 11 wait for feedback

\item In the remaining chapters write according to flow == move around stuff or add new content.
\end{enumerate}

\begin{enumerate}
\item [\textbf{2015-10-29}]
\item [\textbf{Either}]
\item
  We need a convention for what words to capitalize in chapter and
  section titles.

\item [\textbf{Mehul}]

\item
  Abstract is too long.  (Guidelines limit it to 150 words.  Currently
  at 275-ish.)

\item
  Remove sentences that begin with ``And''.
  You may have two in the entire thesis.
  No more!\label{item:no-and}
  Also watch out for sentences that begin with ``But''.

\item
  Search for all latin abbreviations (\textit{e.g.,} ``i.e.''), and be
  sure that you have appropriate punctuation before and after.
  If you are unsure of the punctuation to use, try substituting plain
  English, \yyy{\textit{e.g.,}}{for instance} using ``that is'' in place
  of ``i.e.''.

\item
  Justify the use of capitals in ``Functional styles of programming''.
  Minimally, say what rules you are using.

\item
  Chapter 13.5 needs fleshing out.

\item
  Write enough of Chapters \ref{proto1}--\ref{proto4} that we can decide
  what material is needed in Chapters~ \ref{chap:quasiquotation},
  and~\ref{chap:metasyntacticvariables}.

\item
  I am not sure that I agree with the use of ``reflective'' on
  \textit{p}~8 (\textit{l}~25).  Reflection often means run-time
  introspection (for instance the Java \Verb!.getClass()! method).
  In computer science, reflection is the ability of a computer program to examine (see type introspection) and modify its own structure and behavior (specifically the values, meta-data, properties and functions) at runtime.
  

\item [\textbf{David}]
\item Clean up the non-exclusive license page in unbcthesis.cls
\item Review Chapter 4
\item Review Chapter 5
\item Review Chapter 6
\item Review Chapter 7
\item Review Chapter 8
\item Review Chapter 18
\item ``\textit{IO}'' or ``\textit{I/O}''?  Check!
\item Check spaces before opening parentheses.
\item Search for variants on ``\textit{metasyntactic}'', such as
  \begin{compactitem}
  \item
    ``\textit{meta syntactic}''
  \item
    ``\textit{meta-syntactic}''
  \end{compactitem}
\item Check for ``add on'' versus ``add-on'' and decide which is correct.
\end{enumerate}

\section{Editing suggestions from David}\label{sec:edit-sugg-david}

\paragraph{Thoughts on Chapter~\ref{proto2.1}}\mbox{}

\begin{itemize}
\item
  Do not use naked \Verb!\ref!s:
  ``\textit{the generic methodology from \ref{proto1}}''
  should be
  ``\textsl{the generic methodology from \textbf{\upshape
      Chapter}~\ref{proto1}}''.
\item
  You should say more about \cite{prolog-lib}, either here or in an
  earlier section and reference that discussion here.  For instance, it
  isn't clear that \Verb!prolog-0.2.0.1! comes from~ \cite{prolog-lib}.
\item
  Line~7 on p~55 is not a complete sentence.
\item
  I suspect that \S~\ref{proto2.1}.2 should start with a sentence like 
  \begin{quote}\color[rgb]{0.3,0,0.6}\small\singlespacing
    The \Verb!prolog-0.2.0.1! (\cite{prolog-lib}) was written by Indira
    Ghandi and consists of 718 \progLang{Haskell} files.
    It implements data base assertions and cuts but lacks any IO
    facilities\dots
  \end{quote}
  and then go on to discuss the syntax.
\end{itemize}

\paragraph{Thoughts on Chapter~\ref{proto1}}

I am looking at what are currently lines 145--\textit{on} in
\Verb!proto1.tex!, and I am not sure whether 
\begin{compactenum}
\item
  the text should be loose---as you have it, or floated to a figure.
\item
  I am also not sure whether I like the inlined code, or whether I would
  prefer to have it \Verb!\inputminted! from a \progLang{Haskell} file.
  I suppose that this depends on your work-flow.  Thoughts?
\end{compactenum}

I am not sure what conventions you are following with respect to code in
text.
At some point you have \Verb!FlatTerm! in italics (\'a la
\textit{FlatTerm}); at other points you have it typeset in straight
double quotes ("FlatTerm") and I don't know what the different
typesetting implies.

Just above Section~\ref{proto1}.5 you mention a generic function
\Verb[formatcom=\color{blue}]!map!,
which for \progLang{Standard ML} and \progLang{Haskell} readers likely
means the function with signature
\Verb!(a -> b) -> ([a] -> [b])!\,.
Why not 
\Verb[formatcom=\color{blue}]!fmap!?

I am not sure what the point of the \P{} before Section~\ref{proto1}.5
is. 

\paragraph{Thoughts on 1.1}

We need to firmly fix in mind who the target audience is.  Some
possibilities
\begin{enumerate}
\item Undergraduate Physics students
\item Undergraduate Computer Science students
\item
  Future graduate students of Casperson who have just begun their
  thesis work.
\item
  Simon Peyton-Jones.
\end{enumerate}
If we assume (3), then the material in the first paragraph and part of
the second are unnecessary.

\paragraph{Thoughts on 1.4}

Rename to ``Thesis Organization''.

\paragraph{Thoughts on Chapter 2}

Here are some potential keywords from Chapter 2:
\begin{inparaitem}
\item Hindley-Milner type systems
\item Horn clauses
\item \(\lambda\)-calculi
\item \textsc{Haskell}
\item \textsc{Scala}
\item declarative programming languages
\item foreign function interfaces
\item functional programming
\item implementing Prolog in other languages
\item language embedding
\item language families
\item language paradigms
\item logic programming
\item meta-programming
\item monads
\item paradigm integration
\item quasi-quotation
\item the typed \(\lambda\)-calculus
\item the untyped \(\lambda\)-calculus
\end{inparaitem}.

What is the overall message?

\paragraph{A List of Colloquialisms}\label{sec:list-colloquialisms}
\begin{enumerate}
\item
  ``throws light on''
\item
  another new kid on the block.
\item
  ``Recap'' should be ``recapitulation''.
\item
  Some expressions indicating change over time, for instance:
  \begin{compactitem}
  \item
    day by day
  \item
    over the years
  \item
    many times
  \item
    from time to time
  \item
    \dots
  \end{compactitem}
  are not in themselves problematic, but sometimes can be replaced by
  more precise language.
\item
  ``flipping the coin to the other side''---perhaps ``conversely''?
\item 
  bringing \dots{} onto the same plate.
\item
  job at hand
\item
  hot topic
\item
  faded out (when used metaphorically).
\item
  to play with (when meaning ``to experiment with'').
\item
  adding fuel to the fire 
\item
  push the ideas forward (do you mean popularize?).
\end{enumerate}

\end{scope}
