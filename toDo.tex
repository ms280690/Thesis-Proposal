%%  This file is really -*-LaTeX-*-
%%          Project:  ms280690/Thesis-Proposal
%%             File:  toDo.tex
%%       Created by:  David Casperson
%%          Created:  Sun Oct  4 11:34:28 2015
%%
%%      Description:  Work to do.

\begin{scope}
\nolinenumbers
% ---------------------------------------------------------------------------
\chapter{Editing To Do}\label{chap:to-do}
% ---------------------------------------------------------------------------

\textit{\color{red} This chapter needs to be removed from the final
  work.}

\textbf{Meeting on 5th Novemeber 2015} 
\begin{enumerate}
\item Write about this chapter and chapter conclusion for all chapters

\item Till haskell why haskell chapter 11 wait for feedback

\item In the remaining chapters write according to flow == move around stuff or add new content.
\end{enumerate}

\begin{enumerate}
\item [\textbf{2015-10-29}]
\item [\textbf{Either}]
\item
  We need a convention for what words to capitalize in chapter and
  section titles.

\item [\textbf{Mehul}]

\item
  We need to decide on which version of ghc this compiles against, and ensure that the code does indeed compile
  against that version (\Verb!ErrorT! versus \Verb!ExceptT! and the like).

\item
  Abstract is too long.  (Guidelines limit it to 150 words.  Currently
  at 275-ish.)

\item
  Watch out for sentences that begin with ``But''.

\item
  Search for all latin abbreviations (\textit{e.g.,} ``i.e.''), and be
  sure that you have appropriate punctuation before and after.
  If you are unsure of the punctuation to use, try substituting plain
  English, \yyy{\textit{e.g.,}}{for instance} using ``that is'' in place
  of ``i.e.''.

\item \label{item:5}
  Determine how you want to punctuate items in enumerated (and other)
  list.
  Generally speaking there should be a punctuation mark at the end of an
  item in a list.
  Frequently these are comsas or semi-colons; but they may also be
  periods if each of the list items are sentences or \P{}s.

  Also, there is usually a punctuation mark just before the first item
  of the list, and this is usually \textit{not} a comma.  Colons and
  periods are typical.

\item
  Write enough of Chapters \ref{proto1}--\ref{proto4} that we can decide
  what material is needed in Chapters~ \ref{chap:quasiquotation},
  and~\ref{chap:metasyntacticvariables}.


\item [\textbf{David}]
\item Clean up the non-exclusive license page in unbcthesis.cls
\item Review Chapter 4
\item Review Chapter 5
\item Review Chapter 6
\item Review Chapter 7
\item Review Chapter 8
\item ``\textit{IO}'' or ``\textit{I/O}''?  Check!
\item Check spaces before opening parentheses.
\item Search for variants on ``\textit{metasyntactic}'', such as
  \begin{compactitem}
  \item
    ``\textit{meta syntactic}''
  \item
    ``\textit{meta-syntactic}''
  \end{compactitem}
\item Check for ``add on'' versus ``add-on'' and decide which is correct.
\end{enumerate}

\section{Editing suggestions from David}\label{sec:edit-sugg-david}

\subsection{Thoughts on Chapter~\ref{proto2}}\label{subsec:thoughts-chapt-proto2}

\begin{itemize}
\item
  You should say more about \cite{prolog-lib}, either here or in an
  earlier section and reference that discussion here.  For instance, it
  isn't clear that \Verb!prolog-0.2.0.1! comes from~ \cite{prolog-lib}.
\item
  I suspect that \S~\ref{proto2}.2 should start with a sentence like 
  \begin{quote}\color[rgb]{0.3,0,0.6}\small\singlespacing
    The \Verb!prolog-0.2.0.1! (\cite{prolog-lib}) was written by Indira
    Ghandi and consists of 718 \progLang{Haskell} files.
    It implements data base assertions and cuts but lacks any IO
    facilities\dots
  \end{quote}
  and then go on to discuss the syntax.
\end{itemize}

\subsection{Thoughts on Chapter~\ref{proto1}}\label{subsec:thoughts-chapt-proto1}

I am looking at what are currently lines 145--\textit{on} in
\Verb!proto1.tex!, and I am not sure whether 
the text should be loose---as you have it, or floated to a figure.

I am not sure what conventions you are following with respect to code in
text.
At some point you have \Verb!FlatTerm! in italics (\'a la
\textit{FlatTerm}); at other points you have it typeset in straight
double quotes ("FlatTerm") and I don't know what the different
typesetting implies.


\paragraph{Thoughts on Chapter 2}

Here are some potential keywords from Chapter 2:
\begin{inparaitem}
\item Hindley-Milner type systems
\item Horn clauses
\item \(\lambda\)-calculi
\item \textsc{Haskell}
\item \textsc{Scala}
\item declarative programming languages
\item foreign function interfaces
\item functional programming
\item implementing Prolog in other languages
\item language embedding
\item language families
\item language paradigms
\item logic programming
\item meta-programming
\item monads
\item paradigm integration
\item quasi-quotation
\item the typed \(\lambda\)-calculus
\item the untyped \(\lambda\)-calculus
\end{inparaitem}.

What is the overall message?

\section{A List of Colloquialisms}\label{sec:list-colloquialisms}
\begin{enumerate}
\item
  ``\textit{throws light on}''
\item
  ``\textit{another new kid on the block}''.
\item
  Some expressions indicating change over time, for instance:
  \begin{compactitem}
  \item
    ``\textit{day by day}''
  \item
    ``\textit{over the years}''
  \item
    ``\textit{many times}''
  \item
    ``\textit{from time to time}''
    \dots
  \end{compactitem}
  are not in themselves problematic, but can be replaced by
  more precise language.
\item
  ``\textit{flipping the coin to the other side}''---perhaps ``\textsl{conversely}''? 
\item 
  ``\textit{bringing \dots{} onto the same plate.}''
\item
  ``\textit{job at hand}''
\item
  ``\textit{hot topic}''
\item
  ``\textit{faded out}''  (when used metaphorically).
\item
  ``\textit{adding fuel to the fire }''
\item
  ``\textit{push the ideas forward}'' (do you mean popularize?).
\end{enumerate}

\end{scope}
