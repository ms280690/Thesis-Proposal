%%  This file is really -*-LaTeX-*-
%%          Project:  ms280690/Thesis-Proposal
%%             File:  toDo.tex
%%       Created by:  David Casperson
%%          Created:  Sun Oct  4 11:34:28 2015
%%
%%      Description:  Work to do.

\begin{scope}
\nolinenumbers
% ---------------------------------------------------------------------------
\section{Editing to do}\label{sect:to-do}
% ---------------------------------------------------------------------------

\textit{\color{red} This Chapter needs to be removed from the final
  work.}

\begin{enumerate}
\item [\textbf{Either}]
\item Rename ``\Verb!proposal.*!'' to ``\Verb!thesis-solanki.*!''.
\item Switch the thesis style to UNBC thesis style.  (Not urgent, if
  this breaks other tools, we can do this last, but it would be nice to
  have a sense of what the thesis is going to look like.)
\item
  Check the rules for spacing in the bibliography to ensure that we have
  them right.
  
\item [\textbf{Mehul}]

\item
  \textcolor{red}{Rewrite (Section) Chapter 3.2}\,.
  You are now in a position to state what your contributions are.
  In some sense everything else flows around this.

\item
  Fix the reference at the bottom of page~2:\\
  \Verb!citewikipro- log,somogyi1995logic,website:prolog1000db.!  \textbf{SOLVED}
\item
  Write enough of Chapters \ref{proto1}--\ref{proto4} that we can decide
  what material is needed in Chapters~ \ref{sect:quasiquotation},
  \ref{sect:metasyntacticvariables},
  and~\ref{sect:relatedtermskeywords}.
\item
  {}[\TeX{}nical]
  Remove the \Verb!\paragraph{}!s from the running text.  \LaTeX{}
  ends a paragraph every time that it encounters two end-of-lines
  with only whitespace between them.  \Verb!\par! does the same thing.

  The \Verb!\paragraph! command is in the same family as \Verb!chapter!,
  \Verb!\section!, and so on.  For its correct use, see later in this
  file.

  If you don't like the shape of the paragraphs that you get without
  \Verb!paragraph!, use something like
  \begin{Verbatim}
\setlength{\parindent}{3em}
\setlength{\parskip}{2\baselineskip}
  \end{Verbatim}
  to adjust either the initial paragraph indent, or the inter-paragraph
  space.

\item
  Rewrite (Section) Chapter 3 in formal English.

\item
  Bump the sectioning levels up by one.  That is, what is currently a
  section should become a chapter, what is currently a subsection should
  become a section, and so on.  It may not make sense to do this until
  you have switch to \Verb!thesis.sty!\,.

\item
  ``re-curses'' means to swear again (\textit{p} 9). \textbf{Changed to recurs}
\item
  I am not sure that I agree with the use of ``reflective'' on
  \textit{p}~8 (\textit{l}~25).  Reflection often means run-time
  introspection (for instance the Java \Verb!.getClass()! method).
  In computer science, reflection is the ability of a computer program to examine (see type introspection) and modify its own structure and behavior (specifically the values, meta-data, properties and functions) at runtime.
  http://www2.parc.com/csl/groups/sda/projects/reflection96/docs/malenfant/malenfant.pdf 
  
\item [\textbf{David}]
\item Review Chapter 1
\item Review Chapter 2
\item Review Chapter 3
\item Review Chapter 4
\item Review Chapter 5
\item Review Chapter 6
\item Review Chapter 7
\item Review Chapter 8
\item Review Chapter 18

\end{enumerate}

\subsection{Editing suggestions from David}\label{sec:edit-sugg-david}

\paragraph{Thoughts on 1.1}

We need to firmly fix in mind who the target audience is.  Some
possibilities
\begin{enumerate}
\item Undergraduate Physics students
\item Undergraduate Computer Science students
\item
  Future graduate students of Casperson who have just begun their
  thesis work.
\item
  Simon Peyton-Jones.
\end{enumerate}
If we assume (3), then the material in the first paragraph and part of
the second are unnecessary.

\paragraph{Thoughts on 1.3}

I am unsure that I can summarize this subsection in two sentences.  I
don't know what the problem statement is at the end of it.

\paragraph{Thoughts on 1.4}

Rename to ``Thesis Organization''.

\paragraph{Thoughts on Chapter 2}

Here are some potential keywords from Chapter 2:
\begin{inparaitem}
\item Hindley-Milner type systems
\item Horn clauses
\item \(\lambda\)-calculi
\item \textsc{Haskell}
\item \textsc{Scala}
\item declarative programming languages
\item foreign function interfaces
\item functional programming
\item implementing Prolog in other languages
\item language embedding
\item language families
\item language paradigms
\item logic programming
\item meta-programming
\item monads
\item paradigm integration
\item quasi-quotation
\item the typed \(\lambda\)-calculus
\item the untyped \(\lambda\)-calculus
\end{inparaitem}.

What is the overall message?


\end{scope}
