\documentclass[proposal.tex]{subfiles} 


\begin{document}

\chapter{Multi Paradigm Languages (Functional Logic Languages)}\label{chap:multiparadigm}

Over the years another approach has branched off from embedding languages, to merge and/or integrate programming 
languages from different paradigms. Let us take an example of the \progLang{Scala} Programming Language \cite{website:scala}, a 
hybrid Object-Functional Programming Language which takes a leaf from each of the two books. In this thesis, the languages in 
question are \progLang{Haskell} and \progLang{Prolog}. This section takes a look at the literature on Multi Paradigm Languages, mainly Functional 
Logic Programming Languages that combine two of the most widespread Declarative Programming Styles.      

\par A peak into language classification reveals that it is not always a straight forward task to segregate languages according to their features and/or 
characteristics. Turns out that there are a number of notions which play a role in deciding where the language belongs. Many a times a language ends up 
being a part of almost all paradigms due extensive libraries. Simply speaking, a multi-paradigm programming language is a programming language that 
supports more than one programming paradigm \cite{Krishnamurthi:2008:TPL:1480828.1480846}, more over as Timothy Budd puts it 
\cite{website:wikimultiparadigm} "The idea of a multi paradigm language is to provide a framework in which programmers can work in a variety of styles, 
freely intermixing constructs from different paradigms."       



\begin{comment}
\subparagraph{}
In this section we talk about marrying or integrating the paradigms, multi paradigm programming language approach. Here we talk 
about combining the two most important and widely spread declarative paradigms, Functional and Logical Programming Paradigms. 
\end{comment}  

\section{The Informal Content from Blogs, Articles and Internet Discussions}
\begin{description}
  
\item[$\bullet$] Multi Paradigm Languages

\par A lot has been talked and discussed on coming to clear grounds about the classification of programming languages. If the conventional ideology is
 considered then the scope of each language is pretty much infinite as small extension modules replicate different feature sets which are not naturally 
native to the language itself. The definitions of multi paradigm languages across the web \cite{website:wikimultiparadigm,website:mdn,website:blogc2}
 converge to roughly the same thing that of providing a framework to work with different styles with a list of languages 
\cite{website:wikimpllist,website:dmoz} that ticks the boxes. Generally speaking, it does not feel all that hot or popular in programming circles; one 
reason could be that it is a very broad topic and specifying details can clear the fog.        

\begin{comment}
\begin{enumerate}
\item Wikipedia Multiparadigm Programming Languages
\\* \url{http://en.wikipedia.org/wiki/Multi-paradigm_programming_language#Multi-paradigm}
\\* \url{http://en.wikipedia.org/wiki/List_of_programming_languages_by_type#Multiparadigm_languages}

\item Mozilla Developer Network MDN,
\\* \url{https://developer.mozilla.org/en-US/docs/multiparadigmlanguage.html}

\item Some blog called c2,
\\* \url{http://c2.com/cgi/wiki?MultiParadigmProgrammingLanguage} 
\end{enumerate}
\end{comment}

\item[$\bullet$] Functional Logic Programming Languages

\par Continuing from the previous section, narrowing down the search by considering only multi paradigm declarative languages namely, Functional 
Logical programming languages. By doing so a large amount of information pops up, from articles that give brief description and mentions 
\cite{website:wikiflpl, website:wikiflpllist} to the implementing techniques \cite{website:imlpementingflpl} which give a brief overview of the aim and 
also the backdrop of publications. 

\par The jackpot however is the fact that there is a dedicated website \cite{website:funclogprog} for the history, research and development, existing languages, the literature, the contacts and everything else that one can think of for functional logic languages. As a matter of fact the holy grail of information is maintained by two of the most important people in the field Michael Hanus \cite{website:mhanus} and Sergio Antoy \cite{website:santoy}.      

\begin{comment}
\begin{enumerate}
\item FLPL Wikipedia,
\\* \url{http://en.wikipedia.org/wiki/Functional_logic_programming}
\\* \url{http://en.wikipedia.org/wiki/Category:Functional_logic_programming_languages}

\item Implementation of Functional Logic Languages
\\* \url{http://web.cecs.pdx.edu/~antoy/research/flp/}

\item Functional Logic Programming
\\* \url{http://www.informatik.uni-kiel.de/~mh/FLP/}
 
\end{enumerate}
\end{comment}
\end{description}


\begin{comment}
\section{People}
There are a lot of people working on this but, I found a lot of papers of two of them,
\begin{enumerate}
\item Michael Hanus,
\\* \url{http://www.informatik.uni-kiel.de/~mh/}
\item Sergio Antoy,
\\* \url{http://web.cecs.pdx.edu/~antoy/}
\item Uday S Reddy 
\\* \url{}
\end{enumerate}
\end{comment}

\section{Literature and Publications}
\begin{description}
\item[$\bullet$] Multi Paradigm Languages
\par Possibly one of the most important works towards bringing programming styles together is the book by C.A.R. Hoare \cite{hoare1998unifying} which points out that among the large number of programming paradigms and/or theories the unification theory serves as a complementary rather than a replacement to relate the universe. As as always since we are talking about \progLang{Haskell} we have to include monads and unifying theories using monads \cite{gibbons2013unifying}.
 

\item[$\bullet$] Functional Logic Programming Languages

\par A recent survey \cite{hanus2007multi} throws light on these hybrid languages. 

\par One of the most prominent multi paradigm languages in \progLang{Haskell} is \progLang{Curry} \cite{antoy2010functional}. Th syntax is borrowed from the parent language and so are a lot of the features. Taking a recap, a functional programming language works on the notion of mathematical functions while a logic programming language is based on predicate logic. The strong points of \progLang{Curry} are that the features or basis of the language are general and are visible in a number of languages like \cite{website:toy}. The language can play with problems from both worlds. In a problem where there are no unknowns and/or variables the language behaves like a functional language which is pattern matching the rules and execute the respective bodies. In the case of missing information, it behaves like \progLang{Prolog}; a sub-expression \textit{e} is evaluated on the conditions that it should satisfy which constraint the possible values of \textit{e}. This brings us to the first important feature of functional logic languages \textit{narrowing}. The expressions contain \textit{free variables}; simply speaking incomplete information that needs to be \textit{unified} to a value depending on the constraints of the problem. The language introduces only a few new constructs to support non determinism and choice. Firstly, \textit{narrowing} ($\mathtt{=:=}$), which deals with the expressions and unknown values and binds them with appropriate values. The next one is the \textit{choice} operator ($\mathtt{?}$) for non-deterministic operations. Lastly, for unifying variables and values under some conditions, ($\mathtt{\&}$) operator has been provided to add constraints to the equation. Putting it all together, it gives us the feel of a logic language for something that looks very much like \progLang{Haskell}. Unification is like two way pattern matching and with a similar analogy \progLang{Curry} is a \progLang{Haskell} that works both ways and hence variables can be on either sides. Although the language can do a lot but gaps do exist such as the improvement of narrowing techniques.       
 
 
\end{description}

\section{Some Multi Paradigm Languages}

The list of multi paradigm languages is huge, but in this thesis we will mostly stick to Functional Logical programming languages. Beginning with functional hybrids, a small project language called \progLang{Virgil} \cite{website:virgil}, combining objects to work with functions and procedures. On similar lines is \progLang{Common Object Lisp System (CLOS)} \cite{website:closwiki}. This can be justified as object oriented programming has been one of the most dominant styles of programming and hence even \progLang{Haskell} has one called \progLang{O'Haskell} \cite{website:ohaskell} though it last saw a release back in 2001. Another prominent implementation is \progLang{OCaml} \cite{website:ocamlwiki,website:ocamllang} which adds object oriented capabilities with a powerful type system and module support. This is the case with most of the languages in this section hardly a few have survived as the new ones incorporated the positives of the old.  As mentioned before one of the most poplar \cite{website:langpop} and widely usage both in academia and industry is the \progLang{Scala} \cite{website:scala} programming language stands out.       

\begin{comment}
\begin{enumerate}
\item \progLang{Scala}, Object Functional Programming Language.

\item Virgil, Object Functional Programming Language.

\item CLOS, Common Lisp Object System.

\item .......................????????
\end{enumerate}
\end{comment}

\section{Functional Logic Programming Languages}

Knowing that there is quite some amount of literature out there on these type of languages, it is fairly easy to say that there have been numerous 
attempts at specifications and/or implementations. Sadly though not many have survived leave alone being successful as a result of the competition. 
Only the ones that are easily available or have an implementation or have been cited or referred by other attempts have been included as the list is long 
and does not reflect the main intention of the document. Beginning with the ones from Australia, which seems to be a popular destination for fiddling 
with \progLang{Prolog} and merging paradigms. As of now there have been three popular ones, beginning with \progLang{Neu Prolog}, 
\cite{website:nue-prolog}, \progLang{Oz (Mozart Programming System)} \cite{website:oz-mozart} and \progLang{Mercury} \cite{website:mercury}. 
Delving deeper the languages feel more like extensions of \progLang{Prolog} rather than hybrids. Starting with \progLang{Mercury} which a boundary 
between deterministic and non-deterministic programs, similarly \progLang{Nue Prolog} has special support for functions while \progLang{Oz} gives  
concurrent constraint programming plus distributed support, with different function types for goal solving and expression rewriting.  \progLang{Escher} 
\cite{lloyd1999programming:escher} comes very close to \progLang{Haskell} with monads, higher order functions and lazy evaluation. Taking a look at 
\progLang{Prolog} variants, \progLang{Ciao} \cite{website:ciao}; a preprocessor to \progLang{Prolog} for functional syntax support, \progLang{$\lambda
$ Prolog} \cite{website:lambda-prolog} aims at modular higher order programming with abstract data types in a logical setting, \progLang{Babel} \cite{
website:babel,moreno1992logic, moreno1988babel} combines pure \progLang{Prolog} with a first order functional notation, \progLang{LIFE} \cite{
website:life} is for  Logic, Inheritance, Functions and Equations in \progLang{Prolog} syntax with currying and other features like functional languages 
and others \cite{bert1987lpg,malachi1984tablog}.

The functional language \progLang{Scheme} is a very popular choice for this sort of a thing. With a book \cite{friedman05reasoned} and an implementation to accompany \cite{website:kanren,website:minkanren} which seems to have translated into \progLang{Haskell}, 
\cite{website:haskellkanren,website:molog,website:minikanrent}. 

Finally talking about \progLang{Curry}, one of the most popular \progLang{Haskell} based multi paradigm languages with support for deterministic and non-deterministic computations. Contributing to the same there have been some predecessors \cite{website:alf,website:toy}.       



\begin{comment}
\section{Functional Logic Programming Language}
\begin{enumerate}
	\item The intergration of functions into Logic Programming : From Theory to Practice,
				\\* \url{http://www.informatik.uni-kiel.de/~mh/publications/papers/JLP94.html}
	\item Functional Logic Programming : From theory to curry,
				\\* \url{http://www.informatik.uni-kiel.de/~mh/papers/GanzingerFestschrift.pdf}
	\item Functional Logic Programming,
				\\* \url{http://dl.acm.org/citation.cfm?doid=1721654.1721675}
	\item A Higher Order Rewriting Logic for FLP,
				 \url{http://books.google.ca/books?hl=en\&lr=\&id=TSJDeaVpJyMC\&oi=fnd\&pg=PA153\&dq=functional+logic+programming\&ots=Ikp3Y-kZRV\&sig=j7XQq-Hi-utdeNG54ZFkE1BeBNw\#v=onepage\&q=functional%20logic%20programming&f=false}
	\item Toy a multiparadigm declarative system
	\item A unified computation model for functional and logic programming 
	\item Semantics and Types in Functional Logic Programming 
	\item Polymorphic Types in FLP
	\item A general Computation Scheme for Constraint Logic Programming
\end{enumerate}

\subparagraph{}
List of Functional Logic Languages
\begin{enumerate}
	\item Mercury \cite{website:mercury} provides a boundary, so either deterministic / non-deterministic at call time
	\item Curry \cite{website:curry} 
	\item Escher \cite{lloyd1999programming:escher} lazy eval with monads like haskell and simple 
	\item ALF \cite{website:alf} another release from the Michael Hanus, predecessor to curry, works on a modified wam
  	\item Babel \cite{website:babel,moreno1992logic, moreno1988babel} BABEL combines pure PROLOG with a first order functional notation.
  	\item Ciao \cite{website:ciao} preprocessor for supporting functional syntax which is then converts the code to Prolog
  	\item Life \cite{website:life} LIFE is an interpreted logic programming language, related to Prolog, with features for functional and object-oriented programming. Intended mainly as a research vehicle, LIFE integrates inheritance, functional, and constraint rule programming styles into a logic programming framework.
  	The syntax of LIFE is similar to that of Prolog, and its logic semantics are also related to Prolog. Prolog uses indexed Herbrand terms as its primary data element, but LIFE uses extensible psi-terms. Primitive data types supported by LIFE include booleans, integers, floats, symbols, strings, and sorts (a sort is a kind of type descriptor). In addition to psi-terms, LIFE supports lists with the same syntax as Prolog. As a functional language, LIFE supports higher-order functions, currying, and similar operations. It does not support lazy evaluation.
  	An implementation of the LIFE language is available for free download, it is written in C and works on Unix systems. A programmer's manual and various scholarly papers are also available. The name LIFE is an acronym for Logic, Inheritance, Functions, and Equations.
  	One of the most novel aspects of LIFE, as a logic programming language, is its notion of sorts. Sorts are like types, and like the types in most object-oriented languages, have inheritance relationships.
  	Several different institutions have hosted LIFE development: MCC in Texas, DEC's Paris Research Lab, and most recently Simon Frasier U in Canada. 
  	\item LPG \cite{bert1987lpg} algebraic + logic programming language
  	\item NUE-Prolog \cite{website:nue-prolog} like prolog but with special support for functions
  	\item Oz (Mozart Programming System) \cite{website:oz-mozart} concurrent constraint programming + distributed support, with different function types for goal solving and expression rewriting
  	\item TOY \cite{website:toy} very close to Curry with better constraint handling but no good libs
  	\item $\lambda$ Prolog \cite{website:lambda-prolog} modular higher order programming with abstract data types in a logical setting
	\item Visual Prolog (They claim it is a OOFLPL) \cite{website:visual-prolog} a version of prolog which supports not only func but alos oops which can be used to develop windows applications
	\item TABLOG \cite{malachi1984tablog} TABLOG (Tableau Logic Programming Language) is a language based on first-order predicate logic with equality that combines functional and logic programming. TABLOG incorporate advantages of LISP and PROLOG. A program in TABLOG is a list of formulas in a first-order logic (including equality, negation, and equivalence) that is more general and more expressive than PROLOG?s Horn Clauses. Whereas PROLOG programs must be relational, TABLOG programs may define either relations or functions. While LISP programs yield results of a computation by returning a single output value, TABLOG programs can be relations and can produce several results simultaneously through their arguments. TABLOG employs the Manna-Waldinger deductive-tableau proof system as an interpreter in the same way that PROLOG uses a resolution-based proof system. Unification is used by TABLOG to match a call with a line in the program and to bind arguments. The basic rules of deduction used for computing are nonclausal resolution and rewriting by means of equality and equivalence. A pilot interpreter for the language has been implemented.
	\item Caledon Language \cite{website:caledon} Caledon is a dependently typed, polymorphic, higher order logic programming language in Haskell
	\item HAL \cite{website:hal} constraint solving capabilities to functional programming
	\item Omega System ($\Omega$mega) \cite{website:omega} Haskell theorem prover
	\item MiniKanrenT \cite{website:minikanrent} implementation from reasoned schemer for haskell
	\item Haskell Kanren \cite{website:haskellkanren} implementation from reasoned schemer for haskell
	\item Molog	 \cite{website:molog} implementation from reasoned schemer for haskell
	\item Mini Kanren \cite{website:minkanren} implementation from reasoned schemer to extend scheme with logic capabilities, few new constructs into scheme replicates the essence of prolog.
	\item Kanren \cite{website:kanren}, scheme library from the book 
	\end{enumerate}

\end{comment}


\end{document}
