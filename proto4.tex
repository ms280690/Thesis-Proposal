\documentclass[thesis-solanki.tex]{subfiles}


\ifMain
\externaldocument{thesis-solanki}
\fi
\begin{document}

\chapter{Prototype 4}{\label{proto4}}


\section{About this chapter}
The aim of this prototype is to embed \languageConstruct{IO} operations within the definition of an eDSL so as to allow the chaining and 
control of operations of the language irrespective of them being pure or impure.

\section{\progLang{Haskell} \languageConstruct{IO} is pure}

The discussion in this section is mainly paraphrased from \cite{website:ioispurechristaylor}.

\progLang{Haskell} calls itself a pure functional programming language. Every function in \progLang{Haskell} is a function in the mathematical sense (i.e., ``pure'').
 Even side-effecting \languageConstruct{IO} operations are but a description of what to do, produced by pure code. There are no statements or instructions, 
 only expressions which cannot mutate variables (local or global) nor access state like time or random numbers \cite{website:haskellorg}. Consider the
 example in Listing~\ref{tab:haskellgetline} describing the \haskellConstruct{getLine} function in \progLang{Haskell}. 

\begin{code-list}[H]
\begin{singlespace}
\inputminted{haskell}{haskell-proto4-haskell-getLine.hs}
\end{singlespace}
\caption{\progLang{Haskell} \haskellConstruct{getLine}}
\label{tab:haskellgetline}
\end{code-list}

\languageConstruct{IO} actions can be embedded by building up data structures which can then be executed to cause side-effects, but until that point they 
are pure. Consider the Listing~\ref{tab:ioactiondatatype} describing an example for the same.

\begin{code-list}[H]
\begin{singlespace}
\inputminted{haskell}{haskell-proto4-ioaction-datatype.hs}
\end{singlespace}
\caption{\languageConstruct{IO} action data type taken from \cite{website:ioispurechristaylor}}
\label{tab:ioactiondatatype}
\end{code-list}

\haskellConstruct{IOAction} is one of the following three types:
\begin{enumerate}
\item A container for a value of type \haskellConstruct{a},

\item A container holding a \haskellConstruct{String} to be printed to \haskellConstruct{stdout},
  followed by another \haskellConstruct{IOAction a}, or

\item A container holding a function from \haskellConstruct{String} \Verb!->! \haskellConstruct{IOAction a}, which can be applied
  to whatever \haskellConstruct{String} is read from \haskellConstruct{stdin}. 
\end{enumerate}

The \haskellConstruct{Return} constructor is the terminal operation for any program written in \haskellConstruct{IOAction}. 

Some simple actions include the one that prints to \haskellConstruct{stdout} before returning \Verb!()!:

\mint{haskell}|put s = Put s (Return ())|

and the action that reads from \haskellConstruct{stdin} and returns the string unchanged:

\mint{haskell}|get = Get (\s -> Return s)|

A program is a sequence of actions. Operators for chaining actions and then performing them in a particular order would be required to execute
a program. We could have the second \haskellConstruct{IOAction} depend on the return value of the first one. Consider the \haskellConstruct{seqio}
operator described in Listing~\ref{tab:seqioop}.

\begin{code-list}[H]
\begin{singlespace}
\inputminted{haskell}{haskell-proto4-seqio-op.hs}
\end{singlespace}
\caption{\haskellConstruct{seqio} operation}
\label{tab:seqioop}
\end{code-list}


We want to take the \haskellConstruct{IOAction} \haskellConstruct{a} supplied in the first argument, get its return value (which is of type 
\haskellConstruct{a}) and feed that to the function in 
the second argument, getting an \haskellConstruct{IOAction} \haskellConstruct{b} out, which can be sequenced with the first \haskellConstruct{IOAction} 
\haskellConstruct{a}. Listing ~\ref{tab:ioactionexample} describes an example of chaining \haskellConstruct{IOActions} and 
Listing~\ref{tab:ioactionexampleoutput} shows the output.


\begin{code-list}[H]
\begin{singlespace}
\inputminted{haskell}{haskell-proto4-ioaction-example.hs}
\end{singlespace}
\caption{Example operation with \haskellConstruct{IOAction}s}
\label{tab:ioactionexample}
\end{code-list}

Although this looks like imperative code, it's really a value of type \haskellConstruct{IOAction} \Verb!()!. In \progLang{Haskell},
code can be data and data can be code.

\begin{code-list}[H]
\begin{singlespace}
\inputminted{haskell}{haskell-proto4-ioaction-example-output.hs}
\end{singlespace}
\caption{Output of example operation}
\label{tab:ioactionexampleoutput}
\end{code-list}

\haskellConstruct{IOAction} is a monad. Listing~\ref{tab:ioactionmonad} shows the instance for the same.

\begin{code-list}[H]
\begin{singlespace}
\inputminted{haskell}{haskell-proto4-ioaction-monad.hs}
\end{singlespace}
\caption{\haskellClass{IOAction Monad}}
\label{tab:ioactionmonad}
\end{code-list}



The main benefit of doing this is that we can now sequence actions using \progLang{Haskell}'s
\haskellConstruct{do} notation. Listing~\ref{tab:ioactionexampledo} describes the example from Listing~\ref{tab:ioactionexample}:

\begin{code-list}[H]
\begin{singlespace}
\inputminted{haskell}{haskell-proto4-ioaction-example-do.hs}
\end{singlespace}
\caption{Example operation using \haskellConstruct{do} notation}
\label{tab:ioactionexampledo}
\end{code-list}

Since no code is executed, till this the above example is pure and side-effect free.

To see the effects, we need to define a function that takes an \haskellConstruct{IOAction} \haskellConstruct{a} and converts it into a 
value of type \haskellConstruct{IO} \haskellConstruct{a}, which can then be executed by the interpreter or the runtime system. 
Listing~\ref{tab:ioactionrun} shows the \haskellConstruct{run} function for \haskellConstruct{IOAction}.


\begin{code-list}[H]
\begin{singlespace}
\inputminted{haskell}{haskell-proto4-ioaction-run.hs}
\end{singlespace}
\caption{\haskellConstruct{run} function for \haskellConstruct{IOAction}}
\label{tab:ioactionrun}
\end{code-list}

Listing~\ref{tab:ioactionrunoutput} shows the output for the \haskellConstruct{run} function.

\begin{code-list}[H]
\begin{singlespace}
\inputminted{haskell}{haskell-proto4-ioaction-run-output.hs}
\end{singlespace}
\caption{Output for \haskellConstruct{run} function}
\label{tab:ioactionrunoutput}
\end{code-list}

\haskellConstruct{IOAction} is a mini-language for doing impure, side-effecting code. It restricts the language constructs to only reading 
from \haskellConstruct{stdin} and writing to \haskellConstruct{stdout} in effect creating a safe embedded domain specific language.  

\begin{comment}
\begin{code-list}[H]
\begin{singlespace}
  \inputminted[linenos]{haskell}{haskell-proto4-purvey-wincer.hs}
\end{singlespace}
\caption{\protect\haskellConstruct{IOAction} definitions}
\label{lis:IOAction}
\end{code-list}
\end{comment}


\section{Approach 2}

\begin{comment}
So when the program is getting interpreted the interpreter encounters an IO operation which then gets "interpreted" to the above and it 
continues normally.

The interpreted program is still pure since the IO actions have not been executed 

if the running is done inside a monad then the IO still is pure.
\end{comment}


Listing ~\ref{tab:prologresultdatatype} shows a \progLang{Prolog}-like language encapsulating impure actions for input and output 
operations. The \haskellConstruct{OneBinding} constructor represents a \prologConstruct{unifier} as a pairs of variables. For termination 
we use \haskellConstruct{NoResult}. 


\begin{code-list}[H]
\begin{singlespace}

  \inputminted[linenos, firstline=7, lastline=16]{haskell}{haskell-proto4-platen-winkel.hs}
\end{singlespace}
\caption{\progLang{Prolog}-like language with \languageConstruct{IO} constructors}
\label{tab:prologresultdatatype}
\end{code-list}

\section{IOSketchM}

\begin{code-list}[H]
\begin{singlespace}
\inputminted[linenos, firstline=31, lastline=58]{haskell}{IOSketchM.hs}
\end{singlespace}
\caption{Language data type}
\label{tab:iosketchmlanguagedatatype}
\end{code-list}

\begin{code-list}[H]
\begin{singlespace}
\inputminted[linenos, firstline=74, lastline=111]{haskell}{IOSketchM.hs}
\end{singlespace}
\caption{\haskellConstruct{run} functions}
\label{tab:iosketchmrunfunctions}
\end{code-list}


\begin{code-list}[H]
\begin{singlespace}
\inputminted[linenos, firstline=130, lastline=152]{haskell}{IOSketchM.hs}
\end{singlespace}
\caption{Functionality for \haskellConstruct{XM Monad}}
\label{tab:iosketchmxmmonadfunctionality}
\end{code-list}

\begin{code-list}[H]
\begin{singlespace}
\inputminted[linenos, firstline=161, lastline=184]{haskell}{IOSketchM.hs}
\end{singlespace}
\caption{Language interpretation}
\label{tab:iosketchmlanguageinterpretation}
\end{code-list}


\begin{code-list}[H]
\begin{singlespace}
\inputminted[linenos, firstline=201, lastline=228]{haskell}{IOSketchM.hs}
\end{singlespace}
\caption{Sample program}
\label{tab:iosketchmsampleprogram}
\end{code-list}

\begin{code-list}[H]
\begin{singlespace}
\begin{minted}[linenos]{haskell}
runIO $ runProg testProg 
a?  1
b?  2
c?  3
b?  4
[1,2,3,-4,-14]
\end{minted}
\end{singlespace}
\caption{Sample program output}
\label{tab:iosketchmsampleprogramoutput}
\end{code-list}


\section{Chapter recapitulation}

\ifMain
\begin{scope}
  \nolinenumbers
  \enotesize
  \par
  \begin{singlespace}
  \setlength{\parskip}{12pt plus 2pt minus 1pt}
  \theendnotes
  \par
  \end{singlespace}
\end{scope}
\unbcbibliography{bibliography}
\fi

\end{document}
