\documentclass[thesis-solanki.tex]{subfiles}


\ifMain
\externaldocument{thesis-solanki}
\fi
\begin{document}

\chapter{Prototype 4}{\label{proto4}}


\section{About this chapter}
The aim of this prototype is to embed \languageConstruct{IO} operations within the definition of an eDSL so as to
allow the chaining and control of operations of the language irrespective of them being pure or impure.

\section{\progLang{Haskell} \languageConstruct{IO} is pure}

The discussion in this section is mainly paraphrased from \cite{website:ioispurechristaylor}.

\progLang{Haskell} calls itself a pure functional programming language.
Every function in \progLang{Haskell} is a function in the mathematical sense (i.e., ``pure'').
Even side-effecting \languageConstruct{IO} operations are but a description of what to do, produced by pure code.
There are no statements or instructions, only expressions which cannot mutate variables (local or global) nor
access state like time or random numbers \cite{website:haskellorg}.
Consider the example in Listing~\ref{tab:haskellgetline} describing the \haskellConstruct{getLine} function in
\progLang{Haskell}.

\begin{code-list}[H]
\begin{singlespace}
\inputminted{haskell}{haskell-proto4-haskell-getLine.hs}
\end{singlespace}
\caption{\progLang{Haskell} \haskellConstruct{getLine}}
\label{tab:haskellgetline}
\end{code-list}

\languageConstruct{IO} actions can be embedded by building up data structures which can then be executed to cause
side-effects, but until that point they are pure.
Consider the Listing~\ref{tab:ioactiondatatype} describing an example for the same.

\begin{code-list}[H]
\begin{singlespace}
\inputminted{haskell}{haskell-proto4-ioaction-datatype.hs}
\end{singlespace}
\caption{\languageConstruct{IO} action data type taken from \cite{website:ioispurechristaylor}}
\label{tab:ioactiondatatype}
\end{code-list}

\haskellConstruct{IOAction} is one of the following three types:
\begin{enumerate}
\item
  A container for a value of type \haskellConstruct{a},
\item
  A container holding a \haskellConstruct{String} to be printed to \haskellConstruct{stdout},
  followed by another \haskellConstruct{IOAction a}, or
\item
  A container holding a function from \haskellConstruct{String} \Verb!->! \haskellConstruct{IOAction a}, which can
  be applied to whatever \haskellConstruct{String} is read from \haskellConstruct{stdin}.
\end{enumerate}

The \haskellConstruct{Return} constructor is the terminal operation for any program written in
\haskellConstruct{IOAction}.

Some simple actions include the one that prints to \haskellConstruct{stdout} before returning \Verb!()!:

\mint{haskell}|put s = Put s (Return ())|

\noindent and the action that reads from \haskellConstruct{stdin} and returns the string unchanged:

\mint{haskell}|get = Get (\s -> Return s)|

A program is a sequence of actions. Operators for chaining actions and then performing them in a particular order would be required to execute
a program. We could have the second \haskellConstruct{IOAction} depend on the return value of the first one. Consider the \haskellConstruct{seqio}
operator described in Listing~\ref{tab:seqioop}.

\begin{code-list}[H]
\begin{singlespace}
\inputminted{haskell}{haskell-proto4-seqio-op.hs}
\end{singlespace}
\caption{\haskellConstruct{seqio} operation}
\label{tab:seqioop}
\end{code-list}


We want to take the \haskellConstruct{IOAction} \haskellConstruct{a} supplied in the first argument, get its return
value (which is of type \haskellConstruct{a}) and feed that to the function in the second argument, getting an
\haskellConstruct{IOAction} \haskellConstruct{b} out, which can be sequenced with the first
\haskellConstruct{IOAction} \haskellConstruct{a}.
Listing ~\ref{tab:ioactionexample} describes an example of chaining \haskellConstruct{IOActions} and
Listing~\ref{tab:ioactionexampleoutput} shows the output.


\begin{code-list}[H]
\begin{singlespace}
\inputminted{haskell}{haskell-proto4-ioaction-example.hs}
\end{singlespace}
\caption{Example operation with \haskellConstruct{IOAction}s}
\label{tab:ioactionexample}
\end{code-list}

Although this looks like imperative code, it's really a value of type \haskellConstruct{IOAction} \Verb!()!.
In \progLang{Haskell}, code can be data and data can be code.

\begin{code-list}[H]
\begin{singlespace}
\inputminted{haskell}{haskell-proto4-ioaction-example-output.hs}
\end{singlespace}
\caption{Output of example operation}
\label{tab:ioactionexampleoutput}
\end{code-list}

\haskellConstruct{IOAction} is a monad. Listing~\ref{tab:ioactionmonad} shows the instance for the same.

\begin{code-list}[H]
\begin{singlespace}
\inputminted{haskell}{haskell-proto4-ioaction-monad.hs}
\end{singlespace}
\caption{\haskellClass{IOAction Monad}}
\label{tab:ioactionmonad}
\end{code-list}



The main benefit of doing this is that we can now sequence actions using \progLang{Haskell}'s \haskellConstruct{do}
notation.
Listing~\ref{tab:ioactionexampledo} describes the example from Listing~\ref{tab:ioactionexample}:

\begin{code-list}[H]
\begin{singlespace}
\inputminted{haskell}{haskell-proto4-ioaction-example-do.hs}
\end{singlespace}
\caption{Example operation using \haskellConstruct{do} notation}
\label{tab:ioactionexampledo}
\end{code-list}

Since no code is executed, till this the above example is pure and side-effect free.

To see the effects, we need to define a function that takes an \haskellConstruct{IOAction} \haskellConstruct{a} and
converts it into a value of type \haskellConstruct{IO} \haskellConstruct{a}, which can then be executed by the
interpreter or the runtime system.
Listing~\ref{tab:ioactionrun} shows the \haskellConstruct{run} function for \haskellConstruct{IOAction}.


\begin{code-list}[H]
\begin{singlespace}
\inputminted{haskell}{haskell-proto4-ioaction-run.hs}
\end{singlespace}
\caption{\haskellConstruct{run} function for \haskellConstruct{IOAction}}
\label{tab:ioactionrun}
\end{code-list}

Listing~\ref{tab:ioactionrunoutput} shows the output for the \haskellConstruct{run} function.

\begin{code-list}[H]
\begin{singlespace}
\inputminted{haskell}{haskell-proto4-ioaction-run-output.hs}
\end{singlespace}
\caption{Output for \haskellConstruct{run} function}
\label{tab:ioactionrunoutput}
\end{code-list}

\haskellConstruct{IOAction} is a mini-language for doing impure, side-effecting code.
It restricts the language constructs to only reading from \haskellConstruct{stdin} and writing to
\haskellConstruct{stdout} in effect creating a safe embedded domain specific language.

\begin{comment}
\begin{code-list}[H]
\begin{singlespace}
  \inputminted[linenos]{haskell}{haskell-proto4-purvey-wincer.hs}
\end{singlespace}
\caption{\protect\haskellConstruct{IOAction} definitions}
\label{lis:IOAction}
\end{code-list}
\end{comment}

\newpage

\section{Prototype architecture}
Figure~\ref{fig:proto4architecture} shows architecture for prototype 4.

\begin{figure}[H]
  \centering
  \includegraphics[width=1\textwidth]{proto4_architecture.jpeg}
%\vspace*{1cm}
  \caption{Prototype 4 architecture}
  \label{fig:proto4architecture}
\end{figure}

This architecture adopts a two stage interpretation procedure with \haskellConstruct{runProg} and \haskellConstruct{runIO}. Consider the 
\progLang{Prolog}-like language described in Listing~\ref{tab:langwithpureandimpureconstructors}.

\begin{code-list}[H]
\begin{singlespace}
\inputminted{haskell}{haskell-proto4-prlg-lang.hs}
\end{singlespace}
\caption{Language with pure and impure constructors}
\label{tab:langwithpureandimpureconstructors}
\end{code-list}

This abstract grammar encapsulates constructors which represent pure and impure actions. The first stage of the interpretation, i.e., 
\haskellConstruct{runProg} takes a program and returns a \haskellConstruct{PrologResult} as shown in Listing~\ref{tab:prologresultdatatype}.

\begin{code-list}[H]
\begin{singlespace}
  \inputminted[linenos, firstline=7, lastline=11]{haskell}{haskell-proto4-platen-winkel.hs}
\end{singlespace}
\caption{\progLang{Prolog}-like language with \languageConstruct{IO} constructors}
\label{tab:prologresultdatatype}
\end{code-list}

The \haskellConstruct{NoResult} is used for termination. If a most general unifier is reached then a \haskellConstruct{Cons} construct is
returned. The last two constructors are for read and write operations respectively.

Listing~\ref{tab:runprogtype} describes the type signature of \haskellConstruct{runProg}.
\begin{code-list}[H]
\begin{singlespace}
\mint{haskell}|runProg :: Prolog -> PrologResult| 
\end{singlespace}
\caption{\haskellConstruct{runProg} type signature}
\label{tab:runprogtype}
\end{code-list}

Till this point in the interpretation of the program no \languageConstruct{IO} operations have been executed. As described in the previous 
section we construct functions which upon execution will produce results and hence the partially interpreted program is still pure.

The second stage of interpretation, i.e., \haskellConstruct{runIO} involves executing the impure actions within the \haskellConstruct{IO Monad} to produce the desired side
effects. In this prototype, the impure actions as seen in Listing~\ref{tab:prologresultdatatype} are:

\begin{enumerate}
\item \haskellConstruct{IOIn (IO String) (String -> PrologResult)}, and

\item \haskellConstruct{IOIn (IO String) (String -> PrologResult)} 
\end{enumerate} 
 

Listing~\ref{tab:runiotype} describes the type signature of \haskellConstruct{runIO}.
\begin{code-list}[H]
\begin{singlespace}
\mint{haskell}|runIO :: PrologResult -> IO [Unifier]| 
\end{singlespace}
\caption{\haskellConstruct{runIO} type signature}
\label{tab:runiotype}
\end{code-list}

\haskellConstruct{runIO} accepts as input the result from \haskellConstruct{runProg}, i.e., \haskellConstruct{PrologResult} and executes
each action in the \haskellConstruct{IO Monad}.

\section{Chapter recapitulation}
Recapitulating, this prototype gives an architecture for a two stage interpretation strategy for an embedded domain specific language. In 
this process the first stage produces a pure interpreted program while the latter executes each action to produce output. This approach 
provides modularity and control over the side effecting actions for execution.   


\ifMain\ifDraft
\begin{scope}
  \nolinenumbers
  \enotesize
  \par
  \begin{singlespace}
  \setlength{\parskip}{12pt plus 2pt minus 1pt}
  \theendnotes
  \par
  \end{singlespace}
\end{scope}
\unbcbibliography{bibliography}
\fi\fi
\end{document}
