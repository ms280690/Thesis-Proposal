\documentclass[thesis-solanki.tex]{subfiles}


\begin{document}

\chapter{Prototype 4}{\label{proto4}}


\section{What is this chapter about}

-----------------------------------------------------------------------------

Our aim to embedd IO into the DSL

So something like a \yyy{"}{``}data\yyy{"}{''} declaration for IO operations

\section{I/O is pure}

\cite{website:ioispurechristaylor}

A common question amongst people learning Haskell is whether I/O is pure or not. Haskell advertises itself as a purely functional 
programming language, but I/O looks like it’s inherently impure - for example, the function getLine, which gets a line from stdin, returns 
a different result depending on what the user types:

\begin{minted}[linenos]{haskell}
Prelude> x <- getLine
Hello
Prelude> x
"Hello"
\end{minted}

How can this possibly be pure?

In this post I want to explain exactly why I/O in Haskell is pure. I’ll do it by building up data structures that represent blocks of code. 
These data structures can later be executed, and they cause effects to occur - but until that point we’ll always work with pure functions, 
never with effects.

Let’s look at a simplified form of I/O, where we only care about reading from stdin, writing to stdout and returning a value. We can model 
this with the IOAction data type.

That is, an IOAction is one of the following three things:
\begin{enumerate}
\item A container for a value of type a,

\item A container holding a String to be printed to stdout, followed by another IOAction a, or

\item A container holding a function from String -> IOAction a, which can be applied to whatever String is read from stdin.
\end{enumerate}

Notice that the only terminal constructor is Return – that means that any IOAction must be a combination of Get and Put constructors, 
finally ending in a Return.

Some simple actions include the one that prints to stdout before returning ():

\mint{haskell}|put s = Put s (Return ())|

and the action that reads from stdin and returns the string unchanged:

\mint{haskell}|get = Get (\s -> Return s)|

To build up a language for doing I/O we need to be able to combine and sequence actions. We want the ability to perform an IOAction a 
followed by an IOAction b, and return some result.

In fact, we could have the second IOAction depend on the return value of the first one - that is, we need a sequencing combinator of the 
following type:

\mint{haskell}|seqio :: IOAction a -> (a -> IOAction b) -> IOAction b|

We want to take the IOAction a supplied in the first argument, get its return value (which is of type a) and feed that to the function in 
the second argument, getting an IOAction b out, which can be sequenced with the first IOAction a.

That’s a bit of a mouthful, but writing this combinator isn’t too hard. When we reach the final Return, we apply the function f to get a 
new action. For the other constructors, we keep the form of the action the same, and just thread seqio through the seqio constructor.

Using seqio we can define the action that gets input from stdin and immediately prints it to the screen:

%\mint{haskell}|echo = get $\backslash$ seqio $\backslash$ put|

or even more complicated actions:
\begin{minted}[linenos]{haskell}
hello = put "What is your name?"      `seqio` \_    ->
        get                           `seqio` \name ->
        put "What is your age?"       `seqio` \_    ->
        get                           `seqio` \age  ->
        put ("Hello " ++ name ++ "!") `seqio` \_    ->
        put ("You are " ++ age ++ " years old")
\end{minted}
Although this looks like imperative code (admittedly with pretty unpleasant syntax), it’s really a value of type IOAction (). In Haskell, 
code can be data and data can be code.

In the gist I’ve defined a function to convert an IOAction to a String, which allows them to be printed, so you can load the file into GHCi 
and verify that hello is in fact just data:

\begin{minted}[linenos]{haskell}
*Main> print hello
Put "What is your name?" (
  Get ($0 -> 
    Put "What is your age?" (
      Get ($1 -> 
        Put "Hello $0!" (
          Put "You are $1 years old" (
            Return ()
          )
        )
      )
    )
  )
)
\end{minted}
It will surprise no one to learn that IOAction is a monad. In fact we’ve already defined the necessary bind operation in seqio, so getting 
the Monad instance is trivial:
\begin{minted}[linenos]{haskell}
instance Monad IOAction where
    return = Return
    (>>=)  = seqio
\end{minted}

The main benefit of doing this is that we can now sequence actions using Haskell’s do notation, which desugars into calls to (>>=), and 
hence to seqio. Our earlier hello example can now be written as:

\begin{minted}[linenos]{haskell}
hello2 = do put "What is your name?"
            name <- get
            put "What is your age?"
            age <- get
            put ("Hello, " ++ name ++ "!")
            put ("You are " ++ age ++ " years old!")
\end{minted}
Remember though, that this is still just defining a value of type IOAction () - no code is executed, and no effects occur! So far, this post is 100 \% pure.

To see the effects, we need to define a function that takes an IOAction a and converts it into a value of type IO a, which can then be executed by the interpreter or the runtime system. It’s easy to write such a function just by turning it into the approprate calls to putStrLn and getLine.

\begin{minted}[linenos]{haskell}
run :: IOAction a -> IO a
run (Return a) = return a
run (Put s io) = putStrLn s >> run io
run (Get g)    = getLine >>= \s -> run (g s)
\end{minted}

You can now load up GHCi and apply run to any action – a value of type IO a will be returned, and then immediately executed by the 
interpreter:

\begin{minted}[linenos]{haskell}
*Main> run hello
What is your name?
Chris
What is your age?
29
Hello Chris!
You are 29 years old

\end{minted}
Is there any practical use to this?

Yes - an IOAction is a mini-language for doing I/O. In this mini language you are restricted to only reading from stdin and writing to stdout - there is no accessing files, spawning threads or network I/O.

In effect we have a “safe” domain-specific language. If a user of your program or library supplies a value of type IOAction a, you know that you are free to convert it to an IO a using run and execute it, and it will never do anything except reading from stdin and writing to stdout (not that those things aren’t potentially dangerous in themselves, but…)


\begin{minted}[linenos]{haskell}
-- http://chris-taylor.github.io/blog/2013/02/09/io-is-not-a-side-effect/

data IOAction a = 
-- A container for a value of type a.
                  Return a 
-- A container holding a String to be printed to stdout, followed by another IOAction a.
                | Put String (IOAction a) 
-- A container holding a function from String -> IOAction a, which can be applied to whatever String is read from stdin.
                | Get (String -> IOAction a)
{--

Return 1

Put "hello" (Return ())
Put "hello" (
  Return ()
)

Put "hello" (Return 1)
Put "hello" (
  Return 1
)

Put "hello" (get)
Put "hello" (
  Get ($0 -> 
    Return "$0"
  )
)

Get put
Get ($0 -> 
  Put "$0" (
    Return ()
  )
)

--}

-- Read and return
get :: IOAction String
get   = Get Return
{--

Get ($0 -> 
  Return "$0"
)

--}

-- Print and return.
put :: String -> IOAction ()
put s = Put s (Return ())
{--

put "hello"
Put "hello" (
  Return ()
)

--}

-- (>>=) Action sequencer and combiner :- read -> write -> read -> write -> ........
seqio :: IOAction a -> (a -> IOAction b) -> IOAction b
--      (First action   (Take and perform                       
--      which generates  next action)
--      value a) 
seqio (Return a) f = f a
seqio (Put s io) f = Put s (seqio io f)
seqio (Get g)    f = Get (\s -> seqio (g s) f)

--Take input and print.
echo :: IOAction ()
echo = get `seqio` put
{--

Get ($0 -> 
  Put "$0" (
    Return ()
  )
)

--}

hello :: IOAction ()
hello = put "What is your name?"      `seqio` \_    ->
        get                           `seqio` \name ->
        put "What is your age?"       `seqio` \_    ->
        get                           `seqio` \age  ->
        put ("Hello " ++ name ++ "!") `seqio` \_    -> 
        put ("You are " ++ age ++ " years old")
{--

Put "What is your name?" (
  Get ($0 -> 
    Put "What is your age?" (
      Get ($1 -> 
        Put "Hello $0!" (
          Put "You are $1 years old" (
            Return ()
          )
        )
      )
    )
  )
)

run hello
What is your name?
Mehul
What is your age?
25
Hello Mehul!
You are 25 years old

--}

-- hello in "do" block since IOAction is a Monad
hello2 :: IOAction ()
hello2 = do put "What is your name?"
            name <- get
            put "What is your age?"
            age <- get
            put ("Hello, " ++ name ++ "!")
            put ("You are " ++ age ++ " years old!")
{--

Put "What is your name?" (
  Get ($0 -> 
    Put "What is your age?" (
      Get ($1 -> 
        Put "Hello, $0!" (
          Put "You are $1 years old!" (
            Return ()
          )
        )
      )
    )
  )
)

run hello2
What is your name?
Mehul
What is your age?
25
Hello, Mehul!
You are 25 years old!

--}

-- where the effects happen.
-- "Real" IO functions like return, putStrLn, getLine.
run :: IOAction a -> IO a
run (Return a) = return a
run (Put s io) = putStrLn s >> run io
run (Get f)    = getLine >>= run . f
{--

run (Return 1)
1

run (Put "hello" get)
hello
1
"1"

run (Get put)
1
1

--}


-- Glue code that makes everything play nice --

instance Monad IOAction where
    return = Return
    (>>=)  = seqio

instance Show a => Show (IOAction a) where
  show io = go 0 0 io
    where
      go m n (Return a) = ind m "Return " ++ show a
      go m n (Put s io) = ind m "Put " ++ show s ++ " (\n" ++ go (m+2) n io ++ "\n" ++ ind m ")"
      go m n (Get g)    = let i = "$" ++ show n
                          in ind m "Get (" ++ i ++ " -> \n" ++ go (m+2) (n+1) (g i) ++ "\n" ++ ind m ")"

      ind m s = replicate m ' ' ++ s

-- IOAction is also a Functor --

mapio :: (a -> b) -> IOAction a -> IOAction b
mapio f (Return a) = Return (f a)
mapio f (Put s io) = Put s (mapio f io)
mapio f (Get g)    = Get (\s -> mapio f (g s))
{--

mapio (+1) (Return 1)
Return 2

mapio (id) (Put "hello" get)
Put "hello" (
  Get ($0 -> 
    Return "$0"
  )
)

mapio (id) (Get put)
Get ($0 -> 
  Put "$0" (
    Return ()
  )
)

--}

instance Functor IOAction where
    fmap = mapio



\end{minted}

\section{Dr. Casperson Pure IO}
\begin{minted}[linenos]{haskell}
-- Prolog IO

{--
FREE MONADS
In general, a structure is called free when it is left-adjoint to a forgetful functor.
In this specific instance, the Term data type is a higher-order functor that maps
a functor f to the monad Term f ; this is illustrated by the above two instance
definitions. This Term functor is left-adjoint to the forgetful functor from monads
to their underlying functors.
--}

data Term f a = Pure a 
			  | Impure (f (Term f a))

main 							= undefined

instance Functor f => Functor (Term f) where
	fmap f (Pure x ) 			= Pure (f x )
	fmap f (Impure t) 			= Impure (fmap (fmap f ) t)

instance Functor f => Monad (Term f) where
	return x 					= Pure x
	(Pure x ) 	>>= 	f 		= f x
	(Impure t) 	>>= 	f 		= Impure (fmap (>>= f ) t)

\end{minted}

\section{Mehul Pure IO}

So when the program is getting interpreted the interpreter encounters an IO operation which then gets "interpreted" to the above and it 
continues normally.

The interpreted program is still pure since the IO actions have not been executed 

if the running is done inside a monad then the IO still is pure.


\begin{minted}[linenos]{haskell}
import Data.Traversable
import Control.Monad
import Data.Functor
import Control.Applicative
import System.IO

data PrologResult
   = NoResult
   | Cons OneBinding PrologResult 
   | IOIn (IO String) (String -> PrologResult)
   | IOOut (IO ()) PrologResult



data OneBinding = Pair VariableName VariableName


--data MiniLang a = MyData a | Empty | Input   

--runInIO :: PrologResult -> IO [OneBinding]


data PrologIO a = Input (IO a) | Output (a -> IO ()) | PrologData a | Empty 
--				deriving (Show, Eq, Ord)
{--
instance Functor (PrologIO) where
	fmap f Empty 					= Empty
	fmap f (Input (IO a)) 			= Input (IO (f a))
--	fmap f (Output (a -> IO ()))	= Output (a -> IO ())
--	fmap f (PrologData a)			= PrologData (f a)
--}

instance Monad PrologIO where
	 	return a = PrologData a
--	 	(Input i) >>= (Output o) = i >>= (\a -> (o a))

instance (Show a) => Show (PrologIO a) where
	show (Empty) 		= show "No result"
	show (PrologData a) = show a
--	show (Input f)		= show (f ++ "")
--	show (Output )		


-- (>>=) Action sequencer and combiner :- read -> write -> read -> write -> ........
seqio :: PrologIO a -> (a -> PrologIO b) -> PrologIO b
--      (First action   (Take and perform                       
--      which generates  next action)
--      value a) 
seqio (PrologData a) 	f 	= f a
--seqio (Output o) 		f 	= \a -> o a
--seqio (Input i)    		f 	= \s -> (seqio (i s) f) --				Get (\s -> seqio (g s) f)



{--
instance Applicative PrologIO where
	func =

instance Traversable PrologIO where
	traverse f Empty 					= Empty
	traverse f (Input (IO a)) 			= Input (IO (f a))
	traverse f (Output (a -> IO ()))	= Output ((a) -> IO ())
	traverse f (PrologData a)			= PrologData (f a)
--}


concate :: PrologIO t -> PrologIO t -> IO ()
concate (Input f1) (Output f2) = do
	x <- f1
	f2 x
{--
concate (Input getLine) (Output putStrLn)
Loading package list-extras-0.4.1.4 ... linking ... done.
Loading package syb-0.5.1 ... linking ... done.
Loading package array-0.5.0.0 ... linking ... done.
Loading package deepseq-1.3.0.2 ... linking ... done.
Loading package containers-0.5.5.1 ... linking ... done.
Loading package transformers-0.4.3.0 ... linking ... done.
Loading package mtl-2.2.1 ... linking ... done.
Loading package logict-0.6.0.2 ... linking ... done.
Loading package unification-fd-0.10.0.1 ... linking ... done.
1
1
--}

\end{minted}


\section{Chapter Recapitulation}


\end{document}
