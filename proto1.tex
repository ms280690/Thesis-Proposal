\documentclass[proposal.tex]{subfiles} 


\begin{document}

\section{Prototype 1}{\label{proto1}}

This chapter looks into solving the issue of conflicting type systems of the languages in question. \progLang{Haskell} is a strong 
statically typed language requiring type signature for programming constructs at compile time while \progLang{Prolog} is strong dynamically 
typed which lets through untyped programs. This prototype throws light on the process of tackling the issues involved in creating 
a data type to replicate the target language type system while conforming to the host language restrictions and also utilizing the 
benefits.       


\subsection{Creating a data type}

A type system consists of a set of rules to define a "type" to different constructs in a programming language such as variables, functions 
and so on. A static type system requires types to be attached to the programming constructs before hand which results in finding errors at 
compile time and thus increase the reliability of the program. The other end is the dynamic type system which passes through code which 
would not have worked in former environment, it comes of as less rigid.

The advantages of static typing \cite{meijer2004static}
\begin{enumerate}
\item Earlier detection of errors
\item Better documentation in terms of type signatures
\item More opportunities for compiler optimizations
\item Increased run-time efficiency
\item Better developer tools 
\end{enumerate}          

For dynamic typing
\begin{enumerate}
\item Less rigid
\item Ideal for prototyping / unknown / changing requirements or unpredictable behaviour 
\item Re-usability  
\end{enumerate}

\paragraph{Transitional paragraph}
This prototype 


To start with, replicating the single type "term" in \progLang{Prolog} one must consider the distinct constructs it can be associated to 
such as  complex structures (for example, predicated clauses etc.), don't cares, cuts, variables and so on.

\begin{minted}[linenos]{haskell}	
--david-0.2.0.2

data VariableName = VariableName Int String
      deriving (Eq, Data, Typeable, Ord)

data Atom         = Atom      !String
                  | Operator  !String
      deriving (Eq, Ord, Data, Typeable)
      
data Term = Struct Atom [Term]
          | Var VariableName
          | Wildcard
          | PString   !String
          | PInteger  !Integer
          | PFloat    !Double
          | Flat [FlatItem]
          | Cut Int
      deriving (Eq, Data, Typeable)

data Clause = Clause { lhs :: Term, rhs_ :: [Goal] }
            | ClauseFn { lhs :: Term, fn :: [Term] -> [Goal] }
      deriving (Data, Typeable)

type Program = [Sentence]

type Body    = [Goal]

data Sentence = Query   Body
              | Command Body
              | C Clause
      deriving (Data, Typeable)

\end{minted}

Even though \textit{Term} has a number of constructors the resulting construct has a single type. Hence, a function would still be untyped 
/ singly typed,
\begin{minted}{haskell}
append :: [Term] -> [Term] -> [Term]
\end{minted} 

The above data type is recursive as seen in the constructor,
\mint{haskell}|Struct Atom [Term]|

One of the issues with the above is that it is not possible to distinguish the structure of the data from the data type itself 
\cite{sheard2004two}. Consider the following, a reduced version of the above data type,

\begin{minted}[linenos]{haskell}
type Atom         = String

data VariableName = VariableName Int String
      deriving (Eq, Data, Typeable, Ord)

data Term = Struct Atom [Term]
          | Var VariableName
          | Wildcard -- Don't cares 
          | Cut Int
      deriving (Eq, Data, Typeable)
\end{minted}

To split a data type into two levels, a single recursive data type is replaced by two related data types. Consider the following,
\begin{minted}[linenos]{haskell}
data FlatTerm a = 
		 Struct Atom [a]
	|	Var VariableName
	|	Wildcard
	|	Cut Int deriving (Show, Eq, Ord)
\end{minted}

One result of the approach is that the non-recursive type \textit{FlatTerm} is modular and generic as the structure "FlatTerm" is separate 
from it's type which is "a". Simply speaking we can have something like 
\mint{haskell}|FlatTerm Bool|

and a generic fuinction like,
\mint{haskell}|map :: (a -> b) -> FlatTerm a -> FlatTerm b|


\subsection{Working with the language}
Creating instances,
\begin{minted}[linenos]{haskell}
instance Functor (FlatTerm) where
	fmap = T.fmapDefault

instance Foldable (FlatTerm) where
 	foldMap = T.foldMapDefault

instance Traversable (FlatTerm) where
  	traverse f (Struct atom x)	=	Struct atom <$> 
  				sequenceA (Prelude.map f x)
  	traverse _ (Var v)	=	pure (Var v)
  	traverse _ Wildcard	=	pure (Wildcard)
  	traverse _ (Cut i)	= 	pure (Cut i)

instance Unifiable (FlatTerm) where
	zipMatch (Struct al ls) (Struct ar rs) = 
		if (al == ar) && (length ls == length rs) 
			then Struct al <$> 
				pairWith (\l r -> Right (l,r)) ls rs  		
			else Nothing
	zipMatch Wildcard _ = Just Wildcard
	zipMatch _ Wildcard = Just Wildcard
	zipMatch (Cut i1) (Cut i2) = if (i1 == i2) 
		then Just (Cut i1) 
		else Nothing

instance Applicative (FlatTerm) where
	pure x = Struct "" [x] 
	_ <*> Wildcard	= 	Wildcard
	_ <*> (Cut i) 	= 	Cut i
	_ <*> (Var v)	=	(Var v)
	(Struct a fs) <*> (Struct b xs) = Struct (a ++ b) [f x | f <- fs, x <- xs] 
 
\end{minted}

After flattening do fixing,
 

Opening up the language somehow so as to accommodate your own variables .

 




\subsection{Black box}


\end{document}