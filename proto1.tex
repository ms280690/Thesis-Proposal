
\documentclass[thesis-solanki.tex]{subfiles}


\begin{document}

\chapter{Prototype 1}{\label{proto1}}



\section{About this chapter}
This chapter throws light on what \progLang{Prolog} does to resolve a given query via \textit{unification} and this can be replicated in
the host language along with the challenges.  

This chapter discusses the aspects of opening a language while preserving the original structure of a closed recursive structure in 
\progLang{Haskell}. Also discussed are the issues related to customizing certain aspects such as meta-syntactic variables.

\section{How Prolog works ?}
Looking at how \progLang{Prolog} works \cite{webiste:learnprolognow}.

Most \progLang{Prolog} distributions have three types of terms:
\begin{enumerate}
\item Constants.

\item Variables.

\item Complex terms.
\end{enumerate}

Two terms can be unified if they are the same or the variables can be assigned to terms such that the resulting terms are equal.

The possibilities could be,
\begin{enumerate}
\item If term1 and term2 are constants, then term1 and term2 unify if and only if they are the same atom, or the same number.
\begin{minted}[linenos]{prolog}
?-  =(mia,mia).
yes
\end{minted}

\item If term1 is a variable and term2 is any type of term, then term1 and term2 unify, and term1 is instantiated to term2 . Similarly, if 
term2 is a variable and term1 is any type of term, then term1 and term2 unify, and term2 is instantiated to term1 . (So if they are both 
variables, they’re both instantiated to each other, and we say that they share values.)
\begin{minted}[linenos]{prolog}
?-  mia  =  X.
X  =  mia 
yes
\end{minted}

\begin{minted}[linenos]{prolog}
?-  X  =  Y. 
yes
\end{minted}

\item If term1 and term2 are complex terms, then they unify if and only if:

\begin{enumerate}
\item They have the same functor and arity, and

\item all their corresponding arguments unify, and

\item the variable instantiations are compatible.
\end{enumerate}
\begin{minted}[linenos]{prolog}
?-  k(s(g),Y)  =  k(X,t(k)).
X  =  s(g) 
Y  =  t(k) 
yes
\end{minted}


\item Two terms unify if and only if it follows from the previous three clauses that they unify.
\end{enumerate} 

\newpage

For example, consider the append function 

\begin{minted}[linenos]{prolog}
append([],L,L). 
append([H|T],L2,[H|L3])  :-  append(T,L2,L3).
\end{minted}


\begin{figure}[h]
\centering
\includegraphics[scale = 0.5]{PrologAppendWorking.png}
\caption{Trace for append \cite{webiste:learnprolognowappend}}
\label{fig:Trace for append}
\end{figure}  

\begin{comment}
This chapter looks into solving the issue of conflicting type systems of the languages in question. \progLang{Haskell} is a strong 
statically typed language requiring type signature for programming constructs at compile time while \progLang{Prolog} is strong dynamically 
typed which lets through untyped programs. This prototype throws light on the process of tackling the issues involved in creating 
a data type to replicate the target language type system while conforming to the host language restrictions and also utilizing the 
benefits.       
\end{comment}

\section{What we do in this Prototype}
This prototype throws light on the process of tackling the issues involved in creating 
a data type to replicate the target language type system while conforming to the host language restrictions and also utilizing the 
benefits. 


We have a \progLang{Prolog} like language in \progLang{Haskell} defined via \textit{data}.

The language defined is recursive in nature. 

We convert it into a non recursive data type.


Basically we do Unification monadically.

  


\section{Creating a data type}

A type system consists of a set of rules to define a "type" to different constructs in a programming language such as variables, functions 
and so on. A static type system requires types to be attached to the programming constructs before hand which results in finding errors at 
compile time and thus increase the reliability of the program. The other end is the dynamic type system which passes through code which 
would not have worked in former environment, it comes of as less rigid.

The advantages of static typing \cite{meijer2004static}
\begin{enumerate}
\item Earlier detection of errors
\item Better documentation in terms of type signatures
\item More opportunities for compiler optimizations
\item Increased run-time efficiency
\item Better developer tools 
\end{enumerate}          

For dynamic typing
\begin{enumerate}
\item Less rigid
\item Ideal for prototyping / unknown / changing requirements or unpredictable behaviour 
\item Re-usability  
\end{enumerate}

\paragraph{Transitional paragraph}
An ideal case would would be something that is ........ dont know what to write


To start with, replicating the single type "term" in \progLang{Prolog} one must consider the distinct constructs it can be associated to 
such as complex structures (for example predicates, clauses etc.), don't cares, cuts, variables and so on.

Consider the language below,

\begin{minted}[linenos]{haskell}	
data VariableName = VariableName Int String
      deriving (Eq, Data, Typeable, Ord)
data Atom         = Atom      !String
                  | Operator  !String
      deriving (Eq, Ord, Data, Typeable)
data Term = Struct Atom [Term]
          | Var VariableName
          | Wildcard
          | PString   !String
          | PInteger  !Integer
          | PFloat    !Double
          | Flat [FlatItem]
          | Cut Int
      deriving (Eq, Data, Typeable)
data Clause = Clause { lhs :: Term, rhs_ :: [Goal] }
            | ClauseFn { lhs :: Term, fn :: [Term] -> [Goal] }
      deriving (Data, Typeable)
type Program = [Sentence]
type Body    = [Goal]
data Sentence = Query   Body
              | Command Body
              | C Clause
      deriving (Data, Typeable)
\end{minted}

Even though \textit{Term} has a number of constructors the resulting construct has a single type. Hence, a function would still be untyped 
/ singly typed,
\begin{minted}{haskell}
append :: [Term] -> [Term] -> [Term]
\end{minted} 

The above data type is recursive as seen in the constructor,
\mint{haskell}|Struct Atom [Term]|

One of the issues with the above is that it is not possible to distinguish the structure of the data from the data type itself 
\cite{sheard2004two}. Consider the following, a reduced version of the above data type,

\begin{minted}[linenos]{haskell}
type Atom         = String
data VariableName = VariableName Int String
      deriving (Eq, Data, Typeable, Ord)
data Term = Struct Atom [Term]
          | Var VariableName
          | Wildcard -- Don't cares 
          | Cut Int
      deriving (Eq, Data, Typeable)
\end{minted}

Also one cannot create Quantifiers plus logic 

To split a data type into two levels, a single recursive data type is replaced by two related data types. Consider the following,
\begin{minted}[linenos]{haskell}
data FlatTerm a = 
		 Struct Atom [a]
	|	Var VariableName
	|	Wildcard
	|	Cut Int deriving (Show, Eq, Ord)
\end{minted}

One result of the approach is that the non-recursive type \textit{FlatTerm} is modular and generic as the structure "FlatTerm" is separate 
from it's type which is "a". Simply speaking we can have something like 
\mint{haskell}|FlatTerm Bool|

and a generic fuinction like,
\mint{haskell}|map :: (a -> b) -> FlatTerm a -> FlatTerm b|


\section{Working with the language}
Creating instances,
\begin{minted}[linenos]{haskell}
instance Functor (FlatTerm) where
	fmap = T.fmapDefault
instance Foldable (FlatTerm) where
 	foldMap = T.foldMapDefault
instance Traversable (FlatTerm) where
  	traverse f (Struct atom x)	=	Struct atom <$> 
  				sequenceA (Prelude.map f x)
  	traverse _ (Var v)	=	pure (Var v)
  	traverse _ Wildcard	=	pure (Wildcard)
  	traverse _ (Cut i)	= 	pure (Cut i)
instance Unifiable (FlatTerm) where
	zipMatch (Struct al ls) (Struct ar rs) = 
		if (al == ar) && (length ls == length rs) 
			then Struct al <$> 
				pairWith (\l r -> Right (l,r)) ls rs  		
			else Nothing
	zipMatch Wildcard _ = Just Wildcard
	zipMatch _ Wildcard = Just Wildcard
	zipMatch (Cut i1) (Cut i2) = if (i1 == i2) 
		then Just (Cut i1) 
		else Nothing
instance Applicative (FlatTerm) where
	pure x = Struct "" [x] 
	_ <*> Wildcard	= 	Wildcard
	_ <*> (Cut i) 	= 	Cut i
	_ <*> (Var v)	=	(Var v)
	(Struct a fs) <*> (Struct b xs) = Struct (a ++ b) [f x | f <- fs, x <- xs] 
\end{minted}

After flattening do fixing,
 

Opening up the language somehow so as to accommodate your own variables.

\section{Black box}


\section{Something about unification-fd and Monadic Unification}
Library \cite{unification-fd-lib}


Tutorial 1 \cite{website:unification-fd-lib-tutorial1}


Tutorial 2 \cite{website:unification-fd-lib-tutorial2}


\begin{enumerate}
\item What library provides ?

This module provides first-order structural unification over general structure types. It also provides the standard suite of functions 
accompanying unification (applying bindings, getting free variables, etc.).

The implementation makes use of numerous optimization techniques. First, we use path compression everywhere (for weighted path compression 
see Control.Unification.Ranked). Second, we replace the occurs-check with visited-sets. Third, we use a technique for aggressive 
opportunistic observable sharing; that is, we track as much sharing as possible in the bindings (without introducing new variables), so 
that we can compare bound variables directly and therefore eliminate redundant unifications.


\item Unifiable stuff

The basic class for generating, reading, and writing to bindings stored in a monad. These three functionalities could be split apart, but are combined in order to simplify contexts. Also, because most functions reading bindings will also perform path compression, there's no way to distinguish "true" mutation from mere path compression.

The superclass constraints are there to simplify contexts, since we make the same assumptions everywhere we use BindingMonad.


In order to use our T data type with the rest of the API, we'll need to give a Unifiable instance for it. Before we do that we'll have to give Functor, Foldable, and Traversable instances. These are straightforward and can be automatically derived with the appropriate language pragmas.

The Unifiable class gives one step of the unification process. Just as we only need to specify one level of the ADT (i.e., T) and then we can use the library's UTerm to generate the recursive ADT, so too we only need to specify one level of the unification (i.e., zipMatch) and then we can use the library's operators to perform the recursive unification, subsumption, etc.

The zipMatch function takes two arguments of type t a. The abstract t will be our concrete T type. The abstract a is polymorphic, which ensures that we can't mess around with more than one level of the term at once. If we abandon that guarantee, then you can think of it as if a is UTerm T v. Thus,t a means T (UTerm T v); and T (UTerm T v) is essentially the type UTerm T v with the added guarantee that the values aren't in fact variables. Thus, the arguments to zipMatch are non-variable terms.

The zipMatch method has the rather complicated return type: Maybe (t (Either a (a,a))). Let's unpack this a bit by thinking about how 
unification works. When we try to unify two terms, first we look at their head constructors. If the constructors are different, then the 
terms aren't unifiable, so we return Nothing to indicate that unification has failed. Otherwise, the constructors match, so we have to 
recursively unify their subterms. Since the T structures of the two terms match, we can return Just t0 where t0 has the same T structure as 
both input terms. Where we still have to recursively unify subterms, we fill t0 with Right(l,r) values where l is a subterm of the left 
argument to zipMatch and r is the corresponding subterm of the right argument. Thus, zipMatch is a generalized zipping function for 
combining the shared structure and pairing up substructures. And now, the implementation:

\begin{minted}[linenos]{haskell}

instance Unifiable T where
    zipMatch (T m ls) (T n rs)
        | m /= n    = Nothing
        | otherwise =
            T n <$> pairWith (\l r -> Right(l,r)) ls rs
\end{minted}
Where list-extras:Data.List.Extras.Pair.pairWith is a version of zip which returns Nothing if the lists have different lengths. So, if the 
names m and n match, and if the two arguments have the same number of subterms, then we pair those subterms off in order; otherwise, either 
the names or the lengths don't match, so we return Nothing.


\item UTerm stuff

The type of terms generated by structures t over variables v. The structure type should implement Unifiable and the variable type should 
implement Variable.

The Show instance doesn't show the constructors, in order to improve legibility for large terms.

All the category theoretic instances (Functor, Foldable, Traversable,...) are provided because they are often useful; however, beware that 
since the implementations must be pure, they cannot read variables bound in the current context and therefore can create incoherent 
results. Therefore, you should apply the current bindings before using any of the functions provided by those classes.


\item STVar stuff

This module defines an implementation of unification variables using the ST monad.



\item IntVar stuff


This module defines a state monad for functional pointers represented by integers as keys into an IntMap. This technique was independently 
discovered by Dijkstra et al. This module extends the approach by using a state monad transformer, which can be made into a backtracking 
state monad by setting the underlying monad to some MonadLogic (part of the logict library, described by Kiselyov et al.).

Atze Dijkstra, Arie Middelkoop, S. Doaitse Swierstra (2008) Efficient Functional Unification and Substitution, Technical Report 
UU-CS-2008-027, Utrecht University.

Oleg Kiselyov, Chung-chieh Shan, Daniel P. Friedman, and Amr Sabry (2005) Backtracking, Interleaving, and Terminating Monad Transformers, 
ICFP

A "mutable" unification variable implemented by an integer. This provides an entirely pure alternative to truly mutable alternatives (like 
STVar), which can make backtracking easier.

N.B., because this implementation is pure, we can use it for both ranked and unranked monads.

\item Binding Monad Stuff

A monad for handling STVar bindings.

Run the ST ranked binding monad. N.B., because STVar are rank-2 quantified, this guarantees that the return value has no such references. 
However, in order to remove the references from terms, you'll need to explicitly apply the bindings and ground the term.

\item U.unify stuff

Unify two terms, or throw an error with an explanation of why unification failed. Since bindings are stored in the monad, the two input 
terms and the output term are all equivalent if unification succeeds. However, the returned value makes use of aggressive opportunistic 
observable sharing, so it will be more efficient to use it in future calculations than either argument.

\item U.unifyOccurs

A variant of unify which uses occursIn instead of visited-sets. This should only be used when eager throwing of occursFailure errors is 
absolutely essential (or for testing the correctness of unify). Performing the occurs-check is expensive. Not only is it slow, it's 
asymptotically slow since it can cause the same subterm to be traversed multiple times.

\clearpage

\item Translation stuff

\begin{figure}
\begin{minted}[linenos,frame=single]{haskell}
monadicUnification :: (BindingMonad FlatTerm (STVar s FlatTerm) (ST.STBinding s)) => (forall s. ((Fix FlatTerm) -> (Fix FlatTerm) -> 
  ErrorT (UT.UFailure (FlatTerm) (ST.STVar s (FlatTerm)))
           (ST.STBinding s) (UT.UTerm (FlatTerm) (ST.STVar s (FlatTerm)),
            Map VariableName (ST.STVar s (FlatTerm)))))
monadicUnification t1 t2 = do
--  let
--    t1f = termFlattener t1
--    t2f = termFlattener t2
  (x1,d1) <- lift . translateToUTerm $ t1
  (x2,d2) <- lift . translateToUTerm $ t2
  x3 <- U.unify x1 x2
  --get state from somehwere, state -> dict
  return $! (x3, d1 `Map.union` d2)


goUnify ::
  (forall s. (BindingMonad FlatTerm (STVar s FlatTerm) (ST.STBinding s))
  =>
      (ErrorT
          (UT.UFailure FlatTerm (ST.STVar s FlatTerm))
          (ST.STBinding s)
          (UT.UTerm FlatTerm (ST.STVar s FlatTerm),
             Map VariableName (ST.STVar s FlatTerm)))
     )
  -> [(VariableName, Prolog)]
goUnify test = ST.runSTBinding $ do
  answer <- runErrorT $ test --ERROR
  case answer of
    (Left _)            -> return []
    (Right (_, dict))   -> f1 dict


f1 ::
  (BindingMonad FlatTerm (STVar s FlatTerm) (ST.STBinding s))
  => (forall s. Map VariableName (STVar s FlatTerm)
      -> (ST.STBinding s [(VariableName, Prolog)])
     )
f1 dict = do
  let ld1 = Map.toList dict
  ld2 <- Control.Monad.Error.sequence [ v1 | (k,v) <- ld1, let v1 = UT.lookupVar v]
  let ld3 = [ (k,v) | ((k,_),Just v) <- ld1 `zip` ld2]
      ld4 = [ (k,v) | (k,v2) <- ld3, let v = translateFromUTerm dict v2 ]
  return ld4
\end{minted}
  \vspace*{-1.0\baselineskip}
  \caption{A sample Minted figure}
  \label{fig:sample}
\end{figure}
\end{enumerate}


\section{Chapter Recap}


\end{document}
