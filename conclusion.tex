\documentclass[thesis-solanki.tex]{subfiles}


\ifMain
\externaldocument{thesis-solanki}
\fi
\begin{document}

% ---------------------------------------------------------------------------
\chapter{Conclusion}\label{chap:conclusion}
% ---------------------------------------------------------------------------
\paragraph{}
This thesis was set out to explore and study the various approaches used for bringing features of multiple languages into a single 
programming environment and has identified embedding and paradigm integration as being the major ones. This study has also sought to improve
the existing work on implementing \progLang{Prolog} in \progLang{Haskell}. One of the main contributions of this thesis was to build a 
\progLang{Prolog}-like language in \progLang{Haskell} which not only is closer to a \progLang{Prolog} distribution but also provides 
logic programming functionality as natively as possible in the host language.

During this process we have also thrown light on the subject of eDSLs in \progLang{Haskell} and the support for the same. Moreover, we
have provided a methodology for replicating results achieved in this thesis.

The benefits of \progLang{Haskell} as a tool for embedding domain specific languages have been shown to be effective in proving it to be
a suitable target language.


\ifMain
\begin{scope}
  \nolinenumbers
  \enotesize
  \par
  \begin{singlespace}
  \setlength{\parskip}{12pt plus 2pt minus 1pt}
  \theendnotes
  \par
  \end{singlespace}
\end{scope}
\unbcbibliography{bibliography}
\fi

\end{document}
