\documentclass[thesis-solanki.tex]{subfiles}


\ifMain
\externaldocument{thesis-solanki}
\fi
\begin{document}

% ---------------------------------------------------------------------------
\chapter{Conclusion and Outcomes}\label{chap:conclusion}
% ---------------------------------------------------------------------------

This thesis looked into the problem of embedding \progLang{Prolog} in \progLang{Haskell}. More generally it gives insights into the 
different approaches of merging programming languages namely paradigm integration and embedding. We provide a literature reviews for them
and discuss the topic of functional logic programming languages. We review the state of embedding \progLang{Prolog} in \progLang{Haskell}
and improve on the current work. More generally speaking we look into adopting features provided by \progLang{Haskell} for eDSLs.

This thesis provides an approach to opening up an eDSL in \progLang{Haskell} and a monadic approach for \progLang{Prolog}-like unification
in \progLang{Haskell}. We made it modular so that it can be incorporated with any eDSl defined by an recur abstract syntax and generic to 
support multiple search strategies. We also provide an approach to define an interpretation approach to deal with eDSLs incorporating 
\languageConstruct{IO} operations.


\ifMain
\begin{scope}
  \nolinenumbers
  \enotesize
  \par
  \begin{singlespace}
  \setlength{\parskip}{12pt plus 2pt minus 1pt}
  \theendnotes
  \par
  \end{singlespace}
\end{scope}
\unbcbibliography{bibliography}
\fi

\end{document}
