
\documentclass[thesis-solanki.tex]{subfiles}

\ifMain
\externaldocument{thesis-solanki}
\fi
\begin{document}

%----------------------------------------------------------------------------
\chapter{Background}\label{chap:background}
%----------------------------------------------------------------------------

\section{What this chapter is about}

\noindent\rule{\textwidth}{0.5pt}
%-----------------------------------------------------------------------------

\section{Languages and their classification}
Programming languages fall into different categories also known as ``paradigms''.
They exhibit different characteristics according to the paradigm they fall into.
It has been argued \cite{Krishnamurthi:2008:TPL:1480828.1480846} that rather than classifying a language into a
particular paradigm, it is more accurate that a language exhibits a set of characteristics from a number of
paradigms.
The broader the scope of a language, the broader is the versatility in solving problems.

Programming languages that fall into the same family, in our case declarative programming languages, can be of
different paradigms and can have very contrasting, conflicting characteristics and behaviours.
The two most important ones in the family of declarative languages are the functional and
logical style of programming.

Functional programming, \cite{hughes1989functional} gets its name as the fundamental concept is
to apply mathematical functions to arguments to get results.
A program itself consists of functions and functions only which when applied to arguments produce results without
changing the state that is values on variables and so on.
Higher order functions allow functions to be passed as arguments to other functions.
The roots lie in $\lambda$-calculi \cite{website:lambdacalculuswiki}, a formal system in mathematical logic and
computer science for expressing computation based on function abstraction and application using variable binding
and substitution.
It can be thought as the smallest programming language \cite{rojas2004tutorial}, a single rule and a single
function definition scheme.
In particular there are typed and untyped $\lambda$-calculi.
In untyped $\lambda$-calculi functions have no predetermined type, whereas typed $\lambda$-calculi puts restriction
on what sort (type) of data can a function work with.
\progLang{Scheme} is based on the untyped variant, while \progLang{ML} and \progLang{Haskell} are based on typed
$\lambda$-calculi.
Most typed $\lambda$-calculi langauges are based on the Hindley-Milner (or Damas-Milner or Damas-Hindley-Milner,
\cite{hindley1969principal,milner1978theory,website:hdmtypesystemwiki}) type system.
The Hindley-Milner-like type systems have the ability to give a most general type to a program
without any annotations.
The algorithm \cite{website:hdmtypesystem} works by initially assigning undefined types to all inputs, next check
the body of the function for operations that impose type constraints and go on mapping the types of each of the 
variables, lastly unifying all of the constraints giving the type of the result.
This is, in fact, an instance of the unification algorithm that we discuss in a different context in Chapter~\ref{proto1}\yyy{}{\Large.}

Logic programming, \cite{spivey1995introduction} on the other hand is based on formal logic.
A program is a set of rules and formul\ae{} in symbolic logic that are used to derive new formulas from the old
ones.
This is done until the one which gives the solution is not derived.

\section{\progLang{Haskell} and \progLang{Prolog}}
In this thesis we aim to merge two languages
together and produce a result which exhibits hybrid
properties.
The languages in question are \progLang{Haskell} and \progLang{Prolog}.\endnote{%
  Replace
  \vspace*{-1\baselineskip}
  \begin{quote}\itshape
    In this thesis we aim to merge two languages
    together and produce a result which exhibits hybrid
    properties.
    The languages in question are \progLang{Haskell} and \progLang{Prolog}.
  \end{quote}
  \vspace*{-1\baselineskip}
  with
  \vspace*{-1\baselineskip}
  \begin{quote}\slshape
    In this thesis we aim to merge two languages---%
    \progLang{Haskell} and \progLang{Prolog}---%
    together and produce a result which exhibits hybrid
    properties.
  \end{quote}}
%
These two languages come from respective the functional programming and logical programming branches of the declarative language group.
Some of the dissimilarities between the languages are:
\begin{enumerate}
\item \progLang{Haskell}
  uses \haskellConstruct{pattern matching} while \progLang{Prolog} uses
  \prologConstruct{unification}. 
\item \progLang{Haskell} is all about functions while \progLang{Prolog} is on Horn clause logic.
\end{enumerate} 

\progLang{Prolog} \cite{wikiprolog}, being one of the most dominant logic programming languages, has spawned a
number of distributions and is present from academia to industry.

\progLang{Haskell} is one the most popular \cite{website:langpop} functional languages around and is the first
language to incorporate monads \cite{wadler1992comprehending} for safe input and output.
\haskellConstruct{Monads}\endnote{%
  Here ``monad'' is an (admittedly technical) English word.
  It doesn't need special marking.
  \mehul{changed}
}
can be 
described as composable computation descriptions \cite{website:monadshaskellorg}.
Each monad
consists of a description of what has action has to be executed, how the action has to be run and how to combine
such computations.
An action can describe an impure or side-effecting computation, for example, input and output can be performed
outside the language but can be brought together with pure functions inside in a program resulting in a separation
and maintaining safety with practicality.
\progLang{Haskell} computes results lazily and is strongly typed.

\progLang{Prolog} and \progLang{Haskell} are contrasting in nature, and bringing them into the same environment is
tricky.
The differences in typing, execution, working among others lead to an altogether mixed bag of properties.


The selection of the target language is not uncommon: \progLang{Prolog} seems to be the all time favourite for
``let's implement \progLang{Prolog} in the language X for proving its power and expressibility''.
The \progLang{Prolog} language has been partially implemented \cite{swipembedd} in other languages such as
\progLang{Scheme} \cite{racklog}, \progLang{Lisp}
\cite{komorowski1982qlog,robinson1982loglisp,robinson1980loglisp}, \progLang{Java} \cite{wikiprolog, jlog},
\progLang{JavaScript} \cite{jscriptlog} and the list \cite{yieldprolog} goes on.

\section{Approaches to integration}

The technique of embedding is a shallow one.
It is as if the embedded language floats over the host language.
Other approaches provide deeper integration between the target and the host.
Here we look at a few.
Over time there has been an approach that branches out, which is paradigm integration.
The First International Symposium on \textit{Unifying the Theories of Programming}
    \textup{(\cite{DBLP:conf/utp/2006})}
    produced a lot of work
    \textup{\cite{DBLP:conf/utp/2008,DBLP:conf/utp/2010,DBLP:conf/utp/2012,hoare1998unifying,
      gibbons2013unifying}}.
Hybrid languages that have characteristics from more than one paradigm are coming into the mainstream.
One of the more successful attempts is \progLang{Scala} \cite{website:scala}.
Simply speaking it is like a \textit{functional} \progLang{Java} providing side-effect free programming environment
along with \progLang{Java}-like features.

Before moving on, let us take a look at some terms related to the content above.
To begin with foreign function interfaces (FFI) \cite{website:ffiwiki} provide a mechanism by which a program
written in one programming language can make use of services written in another programming language.
For example, a function
written in \progLang{C} can be called within a program written in \progLang{Haskell} and vice versa through the FFI
mechanism.
Currently the \progLang{Haskell} foreign function interface works only for one language.
Another notable example is the common foreign function interface (CFFI) \cite{website:commonlisp} for
\progLang{Lisp} which provides fairly complete support for \progLang{C} functions and data.
As yet another example, \progLang{Java} provides the java native interface(JNI) for the working with other
languages.
Moreover there are services that provide a common platform for multiple languages to work with each other.
They can be termed as multilingual runtimes
which lay down a common layer for languages to use each others functions.
An example for this is the Microsoft common language runtime (CLR) \cite{website:clrwiki} which is an
implementation of the common language infrastructure (CLI) standard \cite{website:cliwiki}.

Another important concept is meta programming \cite{website:metaprogwiki}, which involves writing computer programs
that write or manipulate other programs.
The language used to write meta programs is known as the meta language while the the language in which the program
to be modified is written is the object language.
Sometimes the meta language and the object language are the same.

\progLang{Haskell} programs can be modified using Template \progLang{Haskell} \cite{website:templatehaskell}, an
extension to the language which provides services to jump between the two types of programs.
The abstract syntax tree used to define a grammar is in the form of a \progLang{Haskell} data type. This allows us to play with the code and go back and forth betwwen the meta program and the object program.

A specific tool used in meta programming is quasi-quotation \cite{mainland2007s,haskellquasi,wikiquasi}, which
permits \progLang{Haskell} expressions and patterns to be constructed using domain specific,
pro\-gram\-mer-de\-fined concrete syntax.
For example, consider a particular application that requires a complex data type.
To accommodate the data type it must be represented
using \progLang{Haskell} syntax and performing pattern matching may turn into a tedious task.
So having the option of using specific syntax reduces the programmer from this burden and this is where a
quasi-quoter comes into the picture.
Template \progLang{Haskell} provides the facilities mentioned above.
For example, consider the code in Figure~\ref{fig:append-prolog}
in \progLang{Prolog} to append two lists,
\begin{code-list}[h]
\begin{minted}[linenos,frame=lines]{prolog}
append([], X, X).
append([X|Xs], Ys, [X|Zs]) :- append(Xs, Ys, Zs).
\end{minted}    
\vspace*{-0.8\baselineskip}
\caption{Code to ``append'' in \protect\progLang{Prolog}.}
\label{fig:append-prolog}
\end{code-list}
going through the code, the first rule says that and empty list appended with any list results in the list itself.
The second predicate matches the head of the first and the resulting lists and then recurses on the tails.
The same in \progLang{Haskell} would be of the form:\endnote{%
  \david{Is the point here that templating helps, or what?}\newline
  \mehul{it is just one of the many ways to bring the languages together}\newline
  \david{Say something in the text.}
}
%\begin{minted}[linenos]{haskell}
%append(Ps, Qs, Rs) = (Ps = [] & Qs = Rs) ||
%	(∃ X, Xs, Ys -> Ps = [X|Xs] & 
%		Rs = [X|Ys] & 
%			append(Xs, Qs, Ys))
%\end{minted}  


\begin{code-list}[th]
  \begin{singlespace}
    \inputminted[linenos]{haskell}{append.pl}
  \end{singlespace}
  \caption{\progLang{Prolog} append function translated to \progLang{Haskell}.}
\label{tab:prlgappndhskl}
\end{code-list}

\begin{code-list}[th]
  \begin{singlespace}
    \inputminted[linenos]{haskell}{append_quasi_quote.hs}
  \end{singlespace}
  \caption{\progLang{Prolog} append function translated to \progLang{Haskell} using quasi quotation.}
\label{tab:prlgappndhsklqq}
\end{code-list}



Listing~\ref{tab:prlgappndhskl} shows \yyy{}{\LARGE !}\endnote{%
  Shows what?
}

Consider the object functional programming language, \progLang{Scala}
\cite{website:scala}.
It is purely functional but with objects and classes.
With the above in mind, coming back to the problem of implementing \progLang{Prolog} in \progLang{Haskell}.\endnote{%
  Sentence fragment.
}
There have been quite a few attempts to ``merge'' these two programming languages that are from different programming
paradigms.
The attempts fall into two categories as follows:

\begin{enumerate}
\item
  Embedding, where \progLang{Prolog} is merely translated to the host language \progLang{Haskell} or a foreign
  function interface.

\item
  Paradigm integration, developing a hybrid programming language that is a functional logic programming language
  with a set of characteristics derived from both the participating languages.
\end{enumerate}

The approaches of embedding and paradigm integration are discussed in Chapter~\ref{chap:embedding} and Chapter~\ref{chap:multiparadigm} 
respectively.  

\section{Chapter Recapitulation}

\ifMain
\begin{scope}
  \nolinenumbers
  \enotesize
  \par
  \begin{singlespace}
  \setlength{\parskip}{12pt plus 2pt minus 1pt}
  \theendnotes
  \par
  \end{singlespace}
\end{scope}
\unbcbibliography{bibliography}
\fi

\end{document}
