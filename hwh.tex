\documentclass[thesis-solanki.tex]{subfiles}

\ifMain
\externaldocument{thesis-solanki}
\fi
\begin{document}

% ---------------------------------------------------------------------------
\chapter{\progLang{Haskell}}\label{chap:hwh}
% ---------------------------------------------------------------------------


\section{About this chapter}
This chapter discusses \progLang{Haskell} as a functional language and its features which assist in embedding domain specific languages.
\progLang{Haskell} as a functional programming language \progLang{Haskell} is an advanced purely-functional programming
  language.
  In particular, it is a polymorphically statically typed, lazy, purely functional language
  \cite{website:haskellwiki}.
  It is one of the popular functional programming languages \cite{website:langpop}.
  \progLang{Haskell} is widely used in the industry \cite{website:haskellinindustry}.
  
\section{Functional programming languages}
Functional programming revolves around the concept of functions being applied to arguments to get results. In functional programming
functions are first class citizens and a main program itself is defined in terms of
other functions, which in turn are defined in terms of still more functions, until
at the bottom level the functions are language primitives. Programs contain no
assignment statements, so variables, once given a value, never change and hence contain no side-effects \cite{hughes1989functional}.


\section{Embedded domain specific languages}
  Shifting a bit to embedded domain specific languages (eDSLs) such as \progLang{Emacs LISP}.
  Opting for embedding provides a ``shortcut'' to create a language which may be designed to provide specific
  functionality.
  Designing a language from scratch would require writing a parser, code generator / interpreter and possibly a
  debugger, not to mention all the routine stuff that every language
  needs such as variables, control structures and 
  arithmetic types.
  All of the aforementioned are provided by the host language; in this case \progLang{Haskell}.
  Examples for the same can be found here \cite{jones2001composing, meyer2008eiffel} which talk about introducing
  combinator libraries for custom functionality.

  The flip side of the coin is that the host language enforces certain aspects and properties of the eDSL and hence
  might not be exact to specification, all required constructs cannot be implemented due to constraints, programs
  could be difficult to debug since it happens at the host level and so on.


\section{Monads}
Control flow defines the order/ manner of execution of statements in a pro\-gram \cite{website:controlflowwiki}.
The specification is set by the programming language.
Generally, in the case of imperative languages the control flow is sequential while for a functional language is
recursion \cite{website:controlflowdalhousie}.
For example, \progLang{Java} has a top down sequential execution approach.
The declarative style consists of defining components of programs i.e.,
computations not a control flow \cite{website:declarativeprogrammingwiki}.

This is where \progLang{Haskell} shines by providing something called a \textit{monad}.
Functional programming languages
define computations which then need to be ordered in some way to form a
combination \cite{website:monadascomputation}.
A monad gives a bubble within the language to allow modification of control flow without affecting the rest of the
universe.
This is especially useful while handling side effects.

A related topic would be of persistence languages, architectures and data structures.
Persistent programming is concerned with creating and manipulating data in a manner that is independent of its
lifetime \cite{morrison1990persistent}.
A persistent data structure supports access to multiple versions which may arise after modifications
\cite{driscoll1986making, website:persistentdatastructuresmit}.
A structure is partially persistent if all versions can be accessed but only the current can be modified and fully
persistent if all of them can be modified.

Coming back to control flow; for example, implementing backtracking in an imperative language would mean undoing
side effects which even \progLang{Prolog} is not able to do since the asserts and retracts cannot be undone.
In \progLang{Haskell}, a monad defines a model for control flow and how side effects would propagate through a
computation from step to step or modification to modification.
\progLang{Haskell} allows creation of custom monads relieving the burden of dealing with a fixed model of the
host language.

\section{Monads by example: state monad}

In this section we try and replicate a dictionary of variables. The operations such as adding and removing entries to and from the
dictionary are implemented so as to replicate modification. Listing~\ref{tab:hskllmndworkngdatatype} shows the structure of the 
\haskellConstruct{IntegerDictionary} for storing and modifying the \haskellConstruct{Variables}. Each \haskellConstruct{Variable} is a \haskellConstruct{VariableName}, \haskellConstruct{Value} pair.
 
\begin{code-list}[H]
  \begin{singlespace}
    \inputminted[linenos, firstline=22, lastline=35]{haskell}{haskell-monad-working-2.hs}
  \end{singlespace}
  \caption{Haskell Monad Working: Data Types}
\label{tab:hskllmndworkngdatatype}
\end{code-list}

Listing~\ref{tab:hskllmndworkngfunctions} shows the insertion and removal in a variable dictionary and the \haskellConstruct{run} function 
for applying the operation on the dictionary.

\begin{code-list}[H]
  \begin{singlespace}
    \inputminted[linenos, firstline=37
      , lastline=71
    ]{haskell}{haskell-monad-working-2.hs}
  \end{singlespace}
  \caption{Haskell Monad Working: Functions}
\label{tab:hskllmndworkngfunctions}
\end{code-list}

Listing~\ref{tab:hskllmndworkngexamples} shows a combination of the operation of finding a product of the two variables and 
storing back the result in the dictionary.


\begin{code-list}[H]
  \begin{singlespace}
    \inputminted[linenos, firstline=73, lastline=86]{haskell}{haskell-monad-working-2.hs}
  \end{singlespace}
  \caption{Haskell Monad Working: Examples}
\label{tab:hskllmndworkngexamples}
\end{code-list}

Listing~\ref{tab:hskllmndworkngexamplesoutput} shows the output of \haskellConstruct{runExampleOperation} function.

\begin{code-list}[H]
  \begin{singlespace}
%    \inputminted[linenos, firstline=86, lastline=89]{haskell}{haskell-monad-working-2.hs} % does not colour as it is in comments in .hs 
%file
\begin{minted}[linenos]{haskell}
runExampleOperation init_dictionary ("x" <-- 10) ("y" <-- 20)
-- output
{[product <-- 200,y <-- 20,x <-- 10,x0 <-- 0,x1 <-- 1,x2 <-- 2,
    x3 <-- 3,x4 <-- 4,x5 <-- 5]}
\end{minted}
  \end{singlespace}
  \caption{Haskell Monad Working: Example output}
\label{tab:hskllmndworkngexamplesoutput}
\end{code-list}


\section{Lazy evaluation}
Another property of \progLang{Haskell} is laziness or lazy evaluation which means that nothing is evaluated until
it is necessary.
This results in the ability to define infinite data structures because at execution only a fragment is used
\cite{website:haskelllazinesswiki}.


\section{Quasiquotation and \progLang{Haskell}}\label{hwh:quasiquotationandhaskell}
\subsection{Quasiquotation}

Quotation is a device for exactly specifying some text \cite{website:quotationstanford}.
Thus ``not p'' refers to the expression consisting of the word not followed by the letter p.
Quasi-quotation, or Quine quotation, is a metalinguistic device for referring to the form of an expression containing
variables without referring to the symbols for those variables.
Thus $\left\ulcorner \mathop\text{not} p \right\urcorner$ refers to the form of any expression consisting of the word not followed by any
value of the variable $p$ \cite{website:quasiquotationfreedictionary}.
Quasi-quotation facilitates rigorous but terse formulation of general rules about expressions
\cite{wikiquasi}.

\subsection{Quasiquotaion in \progLang{Haskell}}

Quasiquoting allows programmers to use custom, domain-specific syntax to construct fragments of their program. Along with 
\progLang{Haskell}'s existing support for domain specific languages, you are now free to use new syntactic forms for your EDSLs. Working 
with complex data types can impose a significant syntactic burden; extensive applications of nested data constructors are often required to 
build values of a given data type, or, worse yet, to pattern match against values. Allow \progLang{Haskell} expressions and patterns to be 
constructed using domain specific, programmer-defined concrete syntax \cite{haskellquasi, mainland2007s}.



\section{Chapter recapitulation}
Recapitulating, this chapter provided information on \progLang{Haskell} as a tool for embedding domain specific languages.


\ifMain
\begin{scope}
  \nolinenumbers
  \enotesize
  \par
  \begin{singlespace}
  \setlength{\parskip}{12pt plus 2pt minus 1pt}
  \theendnotes
  \par
  \end{singlespace}
\end{scope}
\unbcbibliography{bibliography}
\fi


\end{document}
