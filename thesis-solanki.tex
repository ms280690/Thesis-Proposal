%
% proposal.tex
%
% Dissertation Proposal Template.
% School of Computing
% Clemson University
%
\documentclass[12pt,pagewise,right,nofinal]{unbcthesis}

\usepackage{comment}
\usepackage{subfiles}

\usepackage[htt]{hyphenat}
\usepackage{amssymb}
\usepackage{amsmath}
\usepackage{amsthm}
\usepackage{booktabs}
\usepackage{color}
\usepackage{etoolbox}
\usepackage{fancyvrb}
\usepackage{float}      %% added dgc 2015-12-07
\usepackage{graphicx}
\usepackage[pagewise,right]{lineno}
\usepackage{minted}
\usepackage{paralist}
\usepackage{tcolorbox}
\usepackage{thesis-macros}
\usepackage{varioref}   %% added dgc 2016-02-01
\usepackage{xr}
\usepackage{hyperref}      

%% added dgc 2015-12-07
\floatstyle{ruled}%
\floatname{code-list}{Listing}%
\newfloat{code-list}{tp}{lis}[chapter]%


\hyphenation{pro-gram-ming pa-ra-digm pa-ra-digms}


%% Do we need the external document cross-references??
%- \externaldocument{background}
%- \externaldocument{conclusion}
%- \externaldocument{embedding}
%- \externaldocument{introduction}
%- \externaldocument{miscrelcon}
%- \externaldocument{multiparadigm}
%- \externaldocument{prologin} 
%- \externaldocument{prologinhaskell}
%- \externaldocument{proposedwork}
%- \externaldocument{pwp}
%- \externaldocument{quasiquotation}
%- \externaldocument{metasyntacticvariables}
%- \externaldocument{relatedtermskeywords}
%- \externaldocument{relatedWork}
%- \externaldocument{ummcpp}
%- \externaldocument{workCompleted}

%\newcommand{\chapterReference}[1]{\textcolor{blue} {\nameref{#1} } }
\renewcommand\makeenmark{\hbox{\textcolor{red}{$^{\theenmark}$}}}

% This is needed to include figures
% \usepackage{graphicx}

% Use any additional packages you might need

\AtBeginEnvironment{minted}{\singlespacing}
%\BeforeBeginEnvironment{minted}{\begin{tcolorbox}}%
%\AfterEndEnvironment{minted}{\end{tcolorbox}}%
%
% Give values to the variables used in this document
%
\setTitle{Embedding Programming Languages: \textsc{Prolog} in \textsc{Haskell}}
\setMajor{Computer Science}
\edef\calMonth{\ifcase \the \month\or
  January   \or
  February  \or
  March     \or
  April     \or
  May       \or
  June      \or
  July      \or
  August    \or
  September \or
  October   \or
  November  \or
  December  \fi
}
\setMonth{\calMonth}
\setYear{\the \year}
\setAuthor{Mehul Chandrakant Solanki}
\setDegree{Master of Science}


%
% PDF Setup -- You should not need to change this
%
\hypersetup{
    colorlinks,
    linkcolor={black},
    citecolor={black},
    filecolor={black},
    urlcolor={black},
    pdftitle={\theTitle},
    pdfauthor={\theAuthor},
    pdfkeywords={University of Northern British Columbia, \theAuthor},
    pdfstartpage={1}
}



\newcommand{\haskellscript}[2]{
\begin{itemize}
\item[]\lstinputlisting[caption=#2,label=#1]{#1.hs}
\end{itemize}
}

\newif\ifMain
\csname @ifl@aded\endcsname{cls}{subfiles}{\Maintrue}{\Mainfalse}

\begin{document}
%   ==========================================================================
%   Begin front matter (pages are numbered with roman numerals)
%   ==========================================================================
    \begin{frontmatter}
      \unbcthesistitle{\theTitle
      }{\theAuthor}{\theDegree}{\theMajor}{\theMonth}{\theYear}{%
      %% Supply your credentials 
      {B.Eng, Mumbai University, 2012}}
      
      

\begin{unbcabstract}

This thesis focuses on combining the two most important and wide spread declarative programming paradigms, 
functional and logical programming. This will include playing with languages from each paradigm, 
\progLang{Haskell} from the functional side and \progLang{Prolog} from the logical side. The proposed approach 
aims at adding logic programming features which are native to \progLang{Prolog} onto \progLang{Haskell} by 
developing an extension which replicates the target language and utilizes the advanced features of the host for an 
efficient implementation.      

The thesis  aims to provide insights into merging two declarative languages namely, \progLang{Haskell} and
\progLang{Prolog} by embedding the latter into the former and analyzing the result of doing so as they have
conflicting characteristics.
The finished product will be something like a \textit{haskellised} \progLang{Prolog} which has logical programming
like capabilities.

\end{unbcabstract}

%
      \unbctableofcontents
      \newpage
      \unbclistoftables
      \newpage
      \unbclistoffigures
      \newpage
      \unbclistofcode
      \newpage
      % Generate the abstract
    \end{frontmatter}

\linenumbers

%   ==========================================================================
%   Begin main matter (pages are numbered with arabic numerals)
%   ==========================================================================
    \doublespacing     % Text should be double spaced
 
    %
    % I use a file for every section.  Each of these corresponds to a file
    % with the specified name ending in '.tex' (e.g., introduction.tex).
    %
   
   \nocite{claessen2000typed,
   chin2003type,
   hinze1998prological,
   spivey2000functional,
   erwig2004escape,
   spivey1999embedding,
   seres1999algebra,
   seres2001algebra,
   seres2000optimisation,
   seres2001higher,
   spivey1999algebra,
   hoare1998unifying,
   gibbons2013unifying,
   friedman05reasoned,
   krishnamurthi2007programming,
   website:lambda-the-ultimate,
   website:lambda-the-ultimate-2,
   website:lambda-the-ultimate-3,
   website:takashi-workplace,
   website:mini-prolog-hugs98,
   website:logic-programming-haskell,
   website:stackoverflow,
   website:haskell-choice,
   website:prolog-steam,
   website:prolog-death,
   website:prolog-killer,
   somogyi1995logic,
   nanoprolog-lib,
   prolog-lib,
   cspm-To-Prolog-lib,
   prolog-graph-lib,
   prolog-graph-lib-lib,
   hswip-lib,
   logict-lib,
   logict2-lib,
   logic-classes-lib,proplogic-lib,
   cflp-lib,
   logic-grows-on-trees-lib,
   unification-fd-lib,
   cmu-lib,
   peg-lib,
   monadiccp-lib,
   monadiccp-gecode-lib,
   csp-lib,
   liquid-fix-point-lib,
   ramsey2003embedding,
   hudak1996building,
   benton2005embedded,
   Augustsson98cayenne--,
   barzilay2004foreign,
   pinto2003dot,
   reppy2006application,
   ait1999warren,
   castorc++,
   gnuprolog,
   jlog,
   jscriptlog,
   lang2001embedding,
   audklangembedd,
   swipembedd,
   quintusprolog,
   yieldprolog,
   komorowski1982qlog,
   robinson1982loglisp,
   robinson1980loglisp,
   ummlisp,
   racklog}
   
   
   
   \nocite{DBLP:conf/utp/2006,
   DBLP:conf/utp/2008,
   DBLP:conf/utp/2010,
   DBLP:conf/utp/2012,
   wikiquasi,
   haskellquasi,
   hughes1989functional,
   spivey1995introduction,
   Krishnamurthi:2008:TPL:1480828.1480846,
   wadler1992comprehending,
   website:logicprogexamplehaskell,
   wikiprolog,
   website:scala,
   website:virgil,
   website:closwiki,
   website:mercury,
   website:curry,
   lloyd1999programming:escher,
   website:alf,
   website:babel,
   moreno1992logic,
   moreno1988babel,
   website:ciao,
   website:life,
   bert1987lpg,
   website:nue-prolog,
   website:oz-mozart,
   website:oz-mozart,
   website:toy,
   website:lambda-prolog,
   website:visual-prolog,
%   website:tablog,               %% FIXME !
   website:haskellwiki}
    
	\subfile{introduction}
	\clearpage	
	
	\subfile{background}
	\clearpage	
	
	\subfile{proposedWork}
	\clearpage
	
	\subfile{accomplishedWork}
	\clearpage
					
	\subfile{embedding}
	\clearpage
			
	\subfile{multiparadigm}
	\clearpage	

%	\subfile{prologin}
%	\clearpage
			
%	\subfile{prologinhaskell}
%	\clearpage
		
%	\subfile{ummcpp}
%	\clearpage
		
%	\subfile{quasiquotation}
%	\clearpage
	
%	\subfile{metasyntacticvariables}
%	\clearpage
	
%	\subfile{relatedtermskeywords}
%	\clearpage
		
	\subfile{hwh}
	\clearpage
		
	\subfile{pwp}
	\clearpage
			  	
	\subfile{relatedWork}
	\clearpage
				
%	\subfile{miscrelcon}
%	\clearpage
  	
  \subfile{proto1}
  \clearpage
  	
  \subfile{proto2}
  \clearpage
  	
%  \subfile{proto2.2}
%  \clearpage
  	
  \subfile{proto3}
  \clearpage
  	
  \subfile{proto4}
  \clearpage
  	
  %\subfile{workCompleted}
	%\clearpage	
	
	%\subfile{results}
	% \clearpage
	
\subfile{futureScope}
\clearpage	
			
	\subfile{conclusion}
	\clearpage

  %%  This file is really -*-LaTeX-*-
%%          Project:  ms280690/Thesis-Proposal
%%             File:  toDo.tex
%%       Created by:  David Casperson
%%          Created:  Sun Oct  4 11:34:28 2015
%%
%%      Description:  Work to do.

\begin{scope}
\nolinenumbers
% ---------------------------------------------------------------------------
\chapter{Editing to do}\label{chap:to-do}
% ---------------------------------------------------------------------------

\textit{\color{red} This Chapter needs to be removed from the final
  work.}

\begin{enumerate}
\item [\textbf{Either}]
\item Rename ``\Verb!proposal.*!'' to ``\Verb!thesis-solanki.*!''.
\item Switch the thesis style to UNBC thesis style.  (Not urgent, if
  this breaks other tools, we can do this last, but it would be nice to
  have a sense of what the thesis is going to look like.)
\item
  Check the rules for spacing in the bibliography to ensure that we have
  them right.
  
\item [\textbf{Mehul}]

\item
  \textcolor{red}{Rewrite (Section) Chapter 3.2}\,.
  You are now in a position to state what your contributions are.
  In some sense everything else flows around this.

\item
  Fix the reference at the bottom of page~2:\\
  \Verb!citewikipro- log,somogyi1995logic,website:prolog1000db.!  \textbf{SOLVED}
\item
  Write enough of Chapters \ref{proto1}--\ref{proto4} that we can decide
  what material is needed in Chapters~ \ref{sect:quasiquotation},
  \ref{sect:metasyntacticvariables},
  and~\ref{sect:relatedtermskeywords}.
\item
  {}[\TeX{}nical]
  Remove the \Verb!\paragraph{}!s from the running text.  \LaTeX{}
  ends a paragraph every time that it encounters two end-of-lines
  with only whitespace between them.  \Verb!\par! does the same thing.

  The \Verb!\paragraph! command is in the same family as \Verb!chapter!,
  \Verb!\section!, and so on.  For its correct use, see later in this
  file.

  If you don't like the shape of the paragraphs that you get without
  \Verb!paragraph!, use something like
  \begin{Verbatim}
\setlength{\parindent}{3em}
\setlength{\parskip}{2\baselineskip}
  \end{Verbatim}
  to adjust either the initial paragraph indent, or the inter-paragraph
  space.

\item
  Rewrite (Section) Chapter 3 in formal English.

\item
  Bump the sectioning levels up by one.  That is, what is currently a
  section should become a chapter, what is currently a subsection should
  become a section, and so on.  It may not make sense to do this until
  you have switch to \Verb!thesis.sty!\,.

\item
  ``re-curses'' means to swear again (\textit{p} 9). \textbf{Changed to recurs}
\item
  I am not sure that I agree with the use of ``reflective'' on
  \textit{p}~8 (\textit{l}~25).  Reflection often means run-time
  introspection (for instance the Java \Verb!.getClass()! method).
  In computer science, reflection is the ability of a computer program to examine (see type introspection) and modify its own structure and behavior (specifically the values, meta-data, properties and functions) at runtime.
  
\item
  Supply your credentials in the front material (what degrees do you
  have?).
  (Search for \Verb!%% Supply your credentials! in \Verb!proposal1.tex!\,.)

\item [\textbf{David}]
\item Clean up the non-exclusive license page in unbcthesis.cls
\item Incorporate unbcthesis.cls into Mehul's work.
\item Review Chapter 2
\item Review Chapter 3
\item Review Chapter 4
\item Review Chapter 5
\item Review Chapter 6
\item Review Chapter 7
\item Review Chapter 8
\item Review Chapter 18

\end{enumerate}

\section{Editing suggestions from David}\label{sec:edit-sugg-david}

\paragraph{Thoughts on 1.1}

We need to firmly fix in mind who the target audience is.  Some
possibilities
\begin{enumerate}
\item Undergraduate Physics students
\item Undergraduate Computer Science students
\item
  Future graduate students of Casperson who have just begun their
  thesis work.
\item
  Simon Peyton-Jones.
\end{enumerate}
If we assume (3), then the material in the first paragraph and part of
the second are unnecessary.

\paragraph{Thoughts on 1.3}

I am unsure that I can summarize this subsection in two sentences.  I
don't know what the problem statement is at the end of it.

\paragraph{Thoughts on 1.4}

Rename to ``Thesis Organization''.

\paragraph{Thoughts on Chapter 2}

Here are some potential keywords from Chapter 2:
\begin{inparaitem}
\item Hindley-Milner type systems
\item Horn clauses
\item \(\lambda\)-calculi
\item \textsc{Haskell}
\item \textsc{Scala}
\item declarative programming languages
\item foreign function interfaces
\item functional programming
\item implementing Prolog in other languages
\item language embedding
\item language families
\item language paradigms
\item logic programming
\item meta-programming
\item monads
\item paradigm integration
\item quasi-quotation
\item the typed \(\lambda\)-calculus
\item the untyped \(\lambda\)-calculus
\end{inparaitem}.

What is the overall message?


\end{scope}

  \clearpage

\begin{scope}
  \nolinenumbers
  \enotesize
  \par
  \begin{singlespace}
  \setlength{\parskip}{12pt plus 2pt minus 1pt}
  \begin{scope}
    \def\enoteheading{\section*{\notesname}%
      \addtocounter{section}{1}%
      \phantomsection
      \addcontentsline{toc}{section}{\numberline\thesection{End Notes}}}
    \theendnotes
  \end{scope}
  %% dgc 2016-01-09.  If you see an error message that
  %% \begin{scope} is ended by \end{singlespace}, then
  %% there is an unended soope somewhere in an \endnote.
  \par
  \end{singlespace}
\end{scope}


%   ==========================================================================
%   Wrap up the document with the Bibliography (looks for the specified .bib)
%   ==========================================================================
   	

\unbcbibliography{bibliography}
\end{document}
