%
% proposal.tex
%
% Dissertation Proposal Template.
% School of Computing
% Clemson University
%
\documentclass[12pt,pagewise,right,nofinal]{unbcthesis}

\usepackage{comment}
\usepackage{subfiles}

\usepackage[htt]{hyphenat}
\usepackage{amssymb}
\usepackage{amsmath}
\usepackage{amsthm}
\usepackage{booktabs}
\usepackage{color}
\usepackage{etoolbox}
\usepackage{fancyvrb}
\usepackage{float}      %% added dgc 2015-12-07
\usepackage{graphicx}
\usepackage[pagewise,right]{lineno}
\usepackage{minted}
\usepackage{paralist}
\usepackage{tcolorbox}
\usepackage{thesis-macros}
\usepackage{varioref}   %% added dgc 2016-02-01
\usepackage{xr}
\usepackage{hyperref}      

%% added dgc 2015-12-07
\floatstyle{ruled}%
\floatname{code-list}{Listing}%
\newfloat{code-list}{tp}{lis}[chapter]%


\hyphenation{pro-gram-ming pa-ra-digm pa-ra-digms}


%% Do we need the external document cross-references??
%- \externaldocument{background}
%- \externaldocument{conclusion}
%- \externaldocument{embedding}
%- \externaldocument{introduction}
%- \externaldocument{miscrelcon}
%- \externaldocument{multiparadigm}
%- \externaldocument{prologin} 
%- \externaldocument{prologinhaskell}
%- \externaldocument{proposedwork}
%- \externaldocument{pwp}
%- \externaldocument{quasiquotation}
%- \externaldocument{metasyntacticvariables}
%- \externaldocument{relatedtermskeywords}
%- \externaldocument{relatedWork}
%- \externaldocument{ummcpp}
%- \externaldocument{workCompleted}

%\newcommand{\chapterReference}[1]{\textcolor{blue} {\nameref{#1} } }
\renewcommand\makeenmark{\hbox{\textcolor{red}{$^{\theenmark}$}}}

% This is needed to include figures
% \usepackage{graphicx}

% Use any additional packages you might need

\AtBeginEnvironment{minted}{\singlespacing}
%\BeforeBeginEnvironment{minted}{\begin{tcolorbox}}%
%\AfterEndEnvironment{minted}{\end{tcolorbox}}%
%
% Give values to the variables used in this document
%
\setTitle{Embedding Programming Languages: \textsc{Prolog} in \textsc{Haskell}}
\setMajor{Computer Science}
\edef\calMonth{\ifcase \the \month\or
  January   \or
  February  \or
  March     \or
  April     \or
  May       \or
  June      \or
  July      \or
  August    \or
  September \or
  October   \or
  November  \or
  December  \fi
}
\setMonth{\calMonth}
\setYear{\the \year}
\setAuthor{Mehul Chandrakant Solanki}
\setDegree{Master of Science}


%
% PDF Setup -- You should not need to change this
%
\hypersetup{
    colorlinks,
    linkcolor={black},
    citecolor={black},
    filecolor={black},
    urlcolor={black},
    pdftitle={\theTitle},
    pdfauthor={\theAuthor},
    pdfkeywords={University of Northern British Columbia, \theAuthor},
    pdfstartpage={1}
}



\newcommand{\haskellscript}[2]{
\begin{itemize}
\item[]\lstinputlisting[caption=#2,label=#1]{#1.hs}
\end{itemize}
}

\newif\ifMain
\csname @ifl@aded\endcsname{cls}{subfiles}{\Maintrue}{\Mainfalse}

\begin{document}
%   ==========================================================================
%   Begin front matter (pages are numbered with roman numerals)
%   ==========================================================================
    \begin{frontmatter}
      \unbcthesistitle{\theTitle
      }{\theAuthor}{\theDegree}{\theMajor}{\theMonth}{\theYear}{%
      %% Supply your credentials 
      {B.Eng, Mumbai University, 2012}}
      
      

\begin{unbcabstract}

This thesis focuses on combining the two most important and wide spread declarative programming paradigms, 
functional and logic programming. The proposed approach 
aims at adding logic programming features which are native to \progLang{Prolog} onto \progLang{Haskell}. We
develop extensions which replicate the target language by utilizing advanced features of the host language
for an efficient implementation.      

The thesis aims to provide insights into merging two declarative languages namely, \progLang{Haskell} and
\progLang{Prolog} by embedding the latter into the former and analyzing the results of doing so as the two languages have
conflicting characteristics.
The finished products will be something similar to a \textit{haskellised} \progLang{Prolog} which has logic programming-like capabilities.

\end{unbcabstract}

%
      \unbctableofcontents
      \newpage
      \unbclistoftables
      \newpage
      \unbclistoffigures
      \newpage
      \unbclistofcode
      \newpage
      % Generate the abstract
    \end{frontmatter}

\linenumbers

%   ==========================================================================
%   Begin main matter (pages are numbered with arabic numerals)
%   ==========================================================================
    \doublespacing     % Text should be double spaced
 
    %
    % I use a file for every section.  Each of these corresponds to a file
    % with the specified name ending in '.tex' (e.g., introduction.tex).
    %
   
   \nocite{claessen2000typed,
   chin2003type,
   hinze1998prological,
   spivey2000functional,
   erwig2004escape,
   spivey1999embedding,
   seres1999algebra,
   seres2001algebra,
   seres2000optimisation,
   seres2001higher,
   spivey1999algebra,
   hoare1998unifying,
   gibbons2013unifying,
   friedman05reasoned,
   krishnamurthi2007programming,
   website:lambda-the-ultimate,
   website:lambda-the-ultimate-2,
   website:lambda-the-ultimate-3,
   website:takashi-workplace,
   website:mini-prolog-hugs98,
   website:logic-programming-haskell,
   website:stackoverflow,
   website:haskell-choice,
   website:prolog-steam,
   website:prolog-death,
   website:prolog-killer,
   somogyi1995logic,
   nanoprolog-lib,
   prolog-lib,
   cspm-To-Prolog-lib,
   prolog-graph-lib,
   prolog-graph-lib-lib,
   hswip-lib,
   logict-lib,
   logict2-lib,
   logic-classes-lib,proplogic-lib,
   cflp-lib,
   logic-grows-on-trees-lib,
   unification-fd-lib,
   cmu-lib,
   peg-lib,
   monadiccp-lib,
   monadiccp-gecode-lib,
   csp-lib,
   liquid-fix-point-lib,
   ramsey2003embedding,
   hudak1996building,
   benton2005embedded,
   Augustsson98cayenne--,
   barzilay2004foreign,
   pinto2003dot,
   reppy2006application,
   ait1999warren,
   castorc++,
   gnuprolog,
   jlog,
   jscriptlog,
   lang2001embedding,
   audklangembedd,
   swipembedd,
   quintusprolog,
   yieldprolog,
   komorowski1982qlog,
   robinson1982loglisp,
   robinson1980loglisp,
   ummlisp,
   racklog}
   
   
   
   \nocite{DBLP:conf/utp/2006,
   DBLP:conf/utp/2008,
   DBLP:conf/utp/2010,
   DBLP:conf/utp/2012,
   wikiquasi,
   haskellquasi,
   hughes1989functional,
   spivey1995introduction,
   Krishnamurthi:2008:TPL:1480828.1480846,
   wadler1992comprehending,
   website:logicprogexamplehaskell,
   wikiprolog,
   website:scala,
   website:virgil,
   website:closwiki,
   website:mercury,
   website:curry,
   lloyd1999programming:escher,
   website:alf,
   website:babel,
   moreno1992logic,
   moreno1988babel,
   website:ciao,
   website:life,
   bert1987lpg,
   website:nue-prolog,
   website:oz-mozart,
   website:oz-mozart,
   website:toy,
   website:lambda-prolog,
   website:visual-prolog,
%   website:tablog,               %% FIXME !
   website:haskellwiki}
    
	\subfile{introduction}
	\clearpage	
	
	\subfile{background}
	\clearpage	
	
	\subfile{proposedWork}
	\clearpage
	
	\subfile{accomplishedWork}
	\clearpage
					
	\subfile{embedding}
	\clearpage
			
	\subfile{multiparadigm}
	\clearpage	

%	\subfile{prologin}
%	\clearpage
			
%	\subfile{prologinhaskell}
%	\clearpage
		
%	\subfile{ummcpp}
%	\clearpage
		
%	\subfile{quasiquotation}
%	\clearpage
	
%	\subfile{metasyntacticvariables}
%	\clearpage
	
%	\subfile{relatedtermskeywords}
%	\clearpage
		
	\subfile{hwh}
	\clearpage
		
	\subfile{pwp}
	\clearpage
			  	
	\subfile{relatedWork}
	\clearpage
				
%	\subfile{miscrelcon}
%	\clearpage
  	
  \subfile{proto1}
  \clearpage
  	
  \subfile{proto2}
  \clearpage
  	
%  \subfile{proto2.2}
%  \clearpage
  	
  \subfile{proto3}
  \clearpage
  	
  \subfile{proto4}
  \clearpage
  	
  %\subfile{workCompleted}
	%\clearpage	
	
	%\subfile{results}
	% \clearpage
	
\subfile{futureScope}
\clearpage	
			
	\subfile{conclusion}
	\clearpage

  %%  This file is really -*-LaTeX-*-
%%          Project:  ms280690/Thesis-Proposal
%%             File:  toDo.tex
%%       Created by:  David Casperson
%%          Created:  Sun Oct  4 11:34:28 2015
%%
%%      Description:  Work to do.

\begin{scope}
\nolinenumbers
% ---------------------------------------------------------------------------
\chapter{Editing To Do}\label{chap:to-do}
% ---------------------------------------------------------------------------

\textit{\color{red} This chapter needs to be removed from the final
  work.}

\textbf{Meeting on 5th Novemeber 2015} 
\begin{enumerate}
\item Write about this chapter and chapter conclusion for all chapters

\begin{enumerate}
\item [\textbf{Meeting on 5th Novemeber 2015}]
  (\textit{Items merged into list below.})

\item [\textbf{Both}]

\item

\item [\textbf{Mehul}]

\item Remove the remaining non-\textsc{ascii} characters.

\item
  We need to decide on which version of ghc this compiles against, and ensure that the code does indeed compile
  against that version (\Verb!ErrorT! versus \Verb!ExceptT! and the like).

\item

\item
  Watch out for sentences that begin with ``But''.

\item
  There are (as of 2016-02-01 07:45:33) around 30 \macroName{xxx} and \macroName{yyy} corrections.  Fix these, and
  for each one see if there are other changes in the same pattern that I haven't caught.

\item
  Replace \S\S~\ref{sec:thes-impr-contr}--\ref{sec:work-in-points} with
  paragraphs.  Call the result ``\textsl{Thesis Contributions}''.

\item
  Write enough of Chapters \ref{proto1}--\ref{proto4} that we can decide
  what material is needed in Chapters~\ref{chap:quasiquotation}
  and~\ref{chap:metasyntacticvariables}.

\item
  Rewrite \S~\ref{sec:exec-models-prolog} as suggested at the top of that section. 

\item
  Abstract is too long.  (Guidelines limit it to 150 words.  Currently
  at 275-ish.)



\item [\textbf{David}]
\item Clean up the non-exclusive license page in unbcthesis.cls
\item Review Chapter 7
\item Review Chapter 8
\item ``\textit{IO}'' or ``\textit{I/O}''?  Check!
\item Check spaces before opening parentheses.
\item Search for variants on ``\textit{metasyntactic}'', such as
  \begin{compactitem}
  \item
    ``\textit{meta syntactic}''
  \item
    ``\textit{meta-syntactic}''
  \end{compactitem}
\item Check for ``add on'' versus ``add-on'' and decide which is correct.
\end{enumerate}

\section{Editing suggestions from David}\label{sec:edit-sugg-david}

\subsection{Thoughts on Chapter~\ref{proto2}}\label{subsec:thoughts-chapt-proto2}

\begin{itemize}
\item
  You should say more about \cite{prolog-lib}, either here or in an
  earlier section and reference that discussion here.  For instance, it
  isn't clear that \Verb!prolog-0.2.0.1! comes from~ \cite{prolog-lib}.
\item
  I suspect that \S~\ref{proto2}.2 should start with a sentence like 
  \begin{quote}\color[rgb]{0.3,0,0.6}\small\singlespacing
    The \Verb!prolog-0.2.0.1! (\cite{prolog-lib}) was written by Indira
    Ghandi and consists of 718 \progLang{Haskell} files.
    It implements data base assertions and cuts but lacks any IO
    facilities\dots
  \end{quote}
  and then go on to discuss the syntax.
\end{itemize}

\subsection{Thoughts on Chapter~\ref{proto1}}\label{subsec:thoughts-chapt-proto1}

I am looking at what are currently lines 145--\textit{on} in
\Verb!proto1.tex!, and I am not sure whether 
the text should be loose---as you have it, or floated to a figure.

I am not sure what conventions you are following with respect to code in
text.
At some point you have \Verb!FlatTerm! in italics (\'a la
\textit{FlatTerm}); at other points you have it typeset in straight
double quotes ("FlatTerm") and I don't know what the different
typesetting implies.


\paragraph{Thoughts on Chapter 2}

Here are some potential keywords from Chapter 2:
\begin{inparaitem}
\item Hindley-Milner type systems
\item Horn clauses
\item \(\lambda\)-calculi
\item \textsc{Haskell}
\item \textsc{Scala}
\item declarative programming languages
\item foreign function interfaces
\item functional programming
\item implementing Prolog in other languages
\item language embedding
\item language families
\item language paradigms
\item logic programming
\item meta-programming
\item monads
\item paradigm integration
\item quasi-quotation
\item the typed \(\lambda\)-calculus
\item the untyped \(\lambda\)-calculus
\end{inparaitem}.

What is the overall message?

\section{A List of Colloquialisms}\label{sec:list-colloquialisms}
\begin{enumerate}
\item
  ``\textit{throws light on}''
\item
  ``\textit{another new kid on the block}''.
\item
  Some expressions indicating change over time, for instance:
  \begin{compactitem}
  \item
    ``\textit{day by day}''
  \item
    ``\textit{over the years}''
  \item
    ``\textit{many times}''
  \item
    ``\textit{from time to time}''
    \dots
  \end{compactitem}
  are not in themselves problematic, but can be replaced by
  more precise language.
\item
  ``\textit{flipping the coin to the other side}''---perhaps ``\textsl{conversely}''? 
\item 
  ``\textit{bringing \dots{} onto the same plate.}''
\item
  ``\textit{job at hand}''
\item
  ``\textit{hot topic}''
\item
  ``\textit{faded out}''  (when used metaphorically).
\item
  ``\textit{adding fuel to the fire }''
\item
  ``\textit{push the ideas forward}'' (do you mean popularize?).
\end{enumerate}

\end{scope}

  \clearpage

\begin{scope}
  \ifDraft
  \nolinenumbers
  \enotesize
  \par
  \begin{singlespace}
  \setlength{\parskip}{12pt plus 2pt minus 1pt}
  \begin{scope}
    \def\enoteheading{\section*{\notesname}%
      \addtocounter{section}{1}%
      \phantomsection
      \addcontentsline{toc}{section}{\numberline\thesection{End Notes}}}
    \theendnotes
  \end{scope}
  %% dgc 2016-01-09.  If you see an error message that
  %% \begin{scope} is ended by \end{singlespace}, then
  %% there is an unended soope somewhere in an \endnote.
  \par
  \end{singlespace}
  \fi
\end{scope}


%   ==========================================================================
%   Wrap up the document with the Bibliography (looks for the specified .bib)
%   ==========================================================================
   	

\unbcbibliography{bibliography}
\end{document}
