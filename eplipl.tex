
\documentclass[thesis-solanki.tex]{subfiles}


\begin{document}

%-----------------------------------------------------------------------------
\chapter{Embedding a Programming Language into another Programming Language}\label{chap:eplipl}
%-----------------------------------------------------------------------------

\section{What is this chapter about}

-----------------------------------------------------------------------------



Embedding a language into another language has been explored with a variety of languages. Attempts have been made to build Domain Specific Languages from the host languages \cite{hudak1996building}, Foriegn Function Interfaces \cite{barzilay2004foreign}

Creating a programming language from scratch is a tedious task requiring ample amount of programming, not to mention the effort required in
designing. A typical procedure would consist of formulating characteristics and properties based on the following points,

\begin{enumerate}
\item Syntax
\item Semamtics
\item Standard Library
\item Runtime Sytsem
\item Parsers
\item Code Generators
\item Interpreters
\item Debuggers
\end{enumerate}

A lot of the above can be skipped or taken from the base language if an embedding approach is chosen.
For an embedded domain specific language the functionality is translated and written as an add on.
The result can be thought of as a library.
But the difference between an ordinary library and an eDSl is the feature set provided and the degree of embedding
\cite{website:eDSLhaskellwiki}.
For example, reading a file and parsing its contents to perform certain operations to return \textit{string}
results is a shallow form of embedding as the generation of code, results is not native nor are the functions
processing them dealing with embedded data types as such.
On the other hand, building data structures in the base language which represent the target language expression
would be called a deep embedding approach.

The snippet of \progLang{Haskell} code below describes \progLang{Prolog} entities,

\begin{minted}[linenos]{haskell}
data Term = Struct Atom [Term]
          | Var VariableName
          | Wildcard
          | PString   !String
          | PInteger  !Integer
          | PFloat    !Double
          | Flat [FlatItem]
          | Cut Int
      deriving (Eq, Data, Typeable)
\end{minted}

The above can be described as concrete syntax for the \yyy{"}{``}new\yyy{"}{''} language and can be used to write a program.


As discussed in the

\section{Theory}
\begin{enumerate}
%------------------------------------------------------------------------------------------------------------------------------------
\item Papers
\begin{enumerate}
\item Embedding an interpreted language using higher-order functions, \cite{ramsey2003embedding}
\item Building domain-specific embedded languages, \cite{hudak1996building}
\item Embedded interpreters, \cite{benton2005embedded}
\item Cayenne -- a Language With Dependent Types, \cite{Augustsson98cayenne--}
\item Foreign interface for PLT Scheme, \cite{barzilay2004foreign}
\item Dot-Scheme: A PLT Scheme FFI for the .NET framework, \cite{pinto2003dot}
\item Application-specific foreign-interface generation, \cite{reppy2006application}
\item Embedding S in other languages and environments, \cite{lang2001embedding}
\end{enumerate}
%------------------------------------------------------------------------------------------------------------------------------------
\item Books
\begin{enumerate}
\item ?????????
\end{enumerate}
%------------------------------------------------------------------------------------------------------------------------------------
\item Articles / Blogs / Discussions
\begin{enumerate}
\item Embedding one language into another, \cite{website:lambda-the-ultimate-2}
\item Application-specific foreign-interface generation, \cite{website:lambda-the-ultimate-3}
\item Linguistic Abstraction, \cite{audklangembedd}
\item LISP, Unification and Embedded Languages, \cite{ummlisp}
\end{enumerate}
%------------------------------------------------------------------------------------------------------------------------------------
\item Websites
\begin{enumerate}
\item Embedding SWI-Prolog in other applications, \cite{swipembedd}
\end{enumerate}
%------------------------------------------------------------------------------------------------------------------------------------
\end{enumerate}

\section{Implementations}
\begin{enumerate}
\item Lots of them I guess
\end{enumerate}

\section{Important People}
\begin{enumerate}
\item ????
\end{enumerate}

\section{Miscellaneous / Possibly Related Content}
\begin{enumerate}
\item ????
\end{enumerate}


\section{Chapter Recap}


\end{document}
