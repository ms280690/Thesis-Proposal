

\begin{unbcabstract}

This document looks at the problem of combining programming languages with contrasting and conflicting 
characteristics which mostly belong to different pro\-gram\-ming pa\-ra\-digms. The purpose to be fulfilled here is that 
rather than moulding a problem to fit in the chosen language it must be the other way around that the language 
adapts to the problem at hand. Moreover, it reduces the need for jumping between different  languages. The aim is 
achieved either by embedding a target language  whose features are desirable or to be captured into the host 
language which is the base on to which the mapping takes place which can be carried out by creating a module or 
library as an extension to the host language or developing a hybrid programming language that accommodates the 
best of both worlds.  

This thesis focuses on combining the two most important and wide spread declarative programming paradigms, 
functional and logical programming. This will include playing with languages from each paradigm, 
\progLang{Haskell} from the functional side and \progLang{Prolog} from the logical side. The proposed approach 
aims at adding logic programming features which are native to \progLang{Prolog} onto \progLang{Haskell} by 
developing an extension which replicates the target language and utilizes the advanced features of the host for an 
efficient implementation.      

The thesis  aims to provide insights into merging two declarative languages namely, \progLang{Haskell} and
\progLang{Prolog} by embedding the latter into the former and analyzing the result of doing so as they have
conflicting characteristics.
The finished product will be something like a \textit{haskellised} \progLang{Prolog} which has logical programming
like capabilities.


       
\end{unbcabstract}

