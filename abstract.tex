

\begin{unbcabstract}

This thesis focuses on combining the two most important and wide spread declarative programming paradigms, 
functional and logical programming. This will include playing with languages from each paradigm, 
\progLang{Haskell} from the functional side and \progLang{Prolog} from the logical side. The proposed approach 
aims at adding logic programming features which are native to \progLang{Prolog} onto \progLang{Haskell} by 
developing an extension which replicates the target language and utilizes the advanced features of the host for an 
efficient implementation.      

The thesis  aims to provide insights into merging two declarative languages namely, \progLang{Haskell} and
\progLang{Prolog} by embedding the latter into the former and analyzing the result of doing so as they have
conflicting characteristics.
The finished product will be something like a \textit{haskellised} \progLang{Prolog} which has logical programming
like capabilities.

\end{unbcabstract}

