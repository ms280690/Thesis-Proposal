
\documentclass[thesis-solanki.tex]{subfiles}

\ifMain
\externaldocument{thesis-solanki}
\fi
\begin{document}
\chapter{Results}\label{sect:results}


\section{What this chapter is about}

\noindent\rule{\textwidth}{0.5pt}
%-----------------------------------------------------------------------------

\section{Types}


One of the major differences between \progLang{Prolog} and \progLang{Haskell} is how each language handles types.
\progLang{Prolog} is an untyped language meaning any operation can be performed on the data irrespective of its
type.
\progLang{Haskell} on the other hand is strongly typed i.e.
each operation requires a signature stating what types of data it can work with.
Moreover, the \progLang{Haskell} type system is static.


\progLang{Prolog} like any other language can work with some basic data types like numbers, characters, strings
among others.
Using these one can make terms like \textit{Atoms}, \textit{Clauses}, \textit{Constants}, \textit{Strings},
\textit{Characters}, \textit{Predicates}, \textit{Structures}, \textit{Special Characters} and so on.
These need to be incorporated into the implementation so as to give a palette for writing programs.

Our preliminary implementation is as follows,
\begin{minted}{haskell}
type Atom = String

data VariableName = VariableName Int String  deriving (Show,Eq,Ord)

data FlatTerm a = 
		Struct Atom [a]
	| 	Var VariableName
	|	Wildcard
	|	Cut Int deriving (Show,Eq,Ord)

{--
Output :-

Struct "a" [Var (VariableName 0 "x"),Cut 0,Wildcard,Struct "b" []]

--}
\end{minted}
        
which in \progLang{Prolog} would look like,

\begin{minted}{prolog}
a(X, !, b).
\end{minted}





% % % % % % % % % % % % % % % % % % % % % % % % % % % % % % % % % % % % % % % % % % % % % % % % % % % % % % % % % % % % % % % % %









% % % % % % % % % % % % % % % % % % % % % % % % % % % % % % % % % % % % % % % % % % % % % % % % % % % % % % % % % % % % % % % % %
\section{Lazy Evaluation}



% % % % % % % % % % % % % % % % % % % % % % % % % % % % % % % % % % % % % % % % % % % % % % % % % % % % % % % % % % % % % % % % %




% % % % % % % % % % % % % % % % % % % % % % % % % % % % % % % % % % % % % % % % % % % % % % % % % % % % % % % % % % % % % % % % %
\section{Opening up the Language}

\subparagraph{Flattening}

\subparagraph{Fixing}

\subparagraph{MetaSyntactic Variables}

% % % % % % % % % % % % % % % % % % % % % % % % % % % % % % % % % % % % % % % % % % % % % % % % % % % % % % % % % % % % % % % % %





% % % % % % % % % % % % % % % % % % % % % % % % % % % % % % % % % % % % % % % % % % % % % % % % % % % % % % % % % % % % % % % % %
\section{Quasi Quotation}


% % % % % % % % % % % % % % % % % % % % % % % % % % % % % % % % % % % % % % % % % % % % % % % % % % % % % % % % % % % % % % % % % 





% % % % % % % % % % % % % % % % % % % % % % % % % % % % % % % % % % % % % % % % % % % % % % % % % % % % % % % % % % % % % % % % %
\section{Template Haskell}

% % % % % % % % % % % % % % % % % % % % % % % % % % % % % % % % % % % % % % % % % % % % % % % % % % % % % % % % % % % % % % % % %


% % % % % % % % % % % % % % % % % % % % % % % % % % % % % % % % % % % % % % % % % % % % % % % % % % % % % % % % % % % % % % % % %
\section{Higher Order Functions}

\begin{minted}{prolog}
% Mehul Solanki.

% Higher Order Functions.

% The following library contains the maplist function.
:- use_module(library(apply)).

% The maplist function takes a function and a list to apply the 
% function.
% The function write is passes which will print out the elements 
% of the list.
higherOrder(X) :- maplist(write,X).

/*
higherOrder([1,2,3,4]).
1234
true
*/
\end{minted}

% % % % % % % % % % % % % % % % % % % % % % % % % % % % % % % % % % % % % % % % % % % % % % % % % % % % % % % % % % % % % % % % %



% % % % % % % % % % % % % % % % % % % % % % % % % % % % % % % % % % % % % % % % % % % % % % % % % % % % % % % % % % % % % % % % % 
\section{I/O}

\begin{minted}{haskell}
data Result = Ordinary ____________ --No I/O required
| SideEffect (IO __________) 	-- Requiring Output
| ReadEffect (IO _____ -> Result) 	-- Requiring Input	

\end{minted}


% % % % % % % % % % % % % % % % % % % % % % % % % % % % % % % % % % % % % % % % % % % % % % % % % % % % % % % % % % % % % % % % % 


% % % % % % % % % % % % % % % % % % % % % % % % % % % % % % % % % % % % % % % % % % % % % % % % % % % % % % % % % % % % % % % % %
\section{Mutability}



% % % % % % % % % % % % % % % % % % % % % % % % % % % % % % % % % % % % % % % % % % % % % % % % % % % % % % % % % % % % % % % % % %

% % % % % % % % % % % % % % % % % % % % % % % % % % % % % % % % % % % % % % % % % % % % % % % % % % % % % % % % % % % % % % % % % %

\section{Unification}

% % % % % % % % % % % % % % % % % % % % % % % % % % % % % % % % % % % % % % % % % % % % % % % % % % % % % % % % % % % % % % % % % %


% % % % % % % % % % % % % % % % % % % % % % % % % % % % % % % % % % % % % % % % % % % % % % % % % % % % % % % % % % % % % % % % % %
%% TODO 2016-01-16.  remove the embedded space in the title.
\section{ Monads}


% % % % % % % % % % % % % % % % % % % % % % % % % % % % % % % % % % % % % % % % % % % % % % % % % % % % % % % % % % % % % % % % % %



\section{Chapter Recapitulation}

\ifMain
\begin{scope}
  \nolinenumbers
  \enotesize
  \par
  \begin{singlespace}
  \setlength{\parskip}{12pt plus 2pt minus 1pt}
  \theendnotes
  \par
  \end{singlespace}
\end{scope}
\unbcbibliography{bibliography}
\fi

\end{document}
