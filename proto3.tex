\documentclass[thesis-solanki.tex]{files}


\begin{document}

\chapter{Prototype 3}{\label{proto3}}

This chapter discusses the procedure to infuse multiple search strategies into a \progLang{Prolog} query resolver
with monadic unification.
The base implementation for this prototype is \codeLibrary{Mini \progLang{Prolog}}
\cite{website:mini-prolog-hugs98}.


\section{\codeLibrary{Mini \progLang{Prolog}} \cite{website:mini-prolog-hugs98} architecture}
\codeLibrary{Mini \progLang{Prolog}} is based on an older implementation of \progLang{Haskell} called
\codeLibrary{Hugs 98}.
The architecture of the library is described in the Figure~\ref{fig:miniprlgarchitecture}.
The main components are as follows:

\begin{enumerate}
\item the language itself,

\item multiple search strategies used by the query resolver,

\item a parser,

\item a unification mechanism,

\item an interpreter,

\item a knowledge base, and

\item a REPL.  
\end{enumerate}

The main highlight of this implementation is the fact that the query resolver can work with multiple search
strategies decided at compile time.
A query request will consist of the query itself i.e., the terms to be unified, a knowledge base storing the
clauses, the unification procedure and finally a user provided search strategy.


\begin{figure}[H]
  \centering
  \includegraphics[width=0.7\textwidth]{miniprologhugs_architecture.png}
%\vspace*{1cm}
  \caption{Mini \progLang{Prolog} architecture}
  \label{fig:miniprlgarchitecture}
\end{figure}


\section{Prototype architecture}

The focus of this prototype is to embed the language modification procedure and monadic unification into
\cite{website:mini-prolog-hugs98} so to further prove the generality and modularity of the approach from the
previous prototypes.
The architecture for this prototype is beautifully illustrated by Figure~\ref{fig:architecture-proto-3}.

\begin{figure}[H]
  \includegraphics[width=1\textwidth]{proto3_architecture_revised.png}
\vspace*{-1cm}
  \caption{Architecture of Prototype 3}
  \label{fig:architecture-proto-3}
\end{figure}

Since we are aiming for modularity, most components in the Figure~\ref{fig:miniprlgarchitecture} are untouched.
The abstract syntax is modified to conform to the \codeLibrary{unification-fd} library \cite{unification-fd-lib}.
Looking at the center of the figure you will find \textit{query}.
This component takes as input the terms to be unified in modified language form, a search strategy, the knowledge
base and the monadic unifier, and returns a list of substitutions as required by \codeLibrary{Mini
  \progLang{Prolog}} library \cite{website:mini-prolog-hugs98}.
Each of the components in Figure~\ref{fig:architecture-proto-3} will be discussed in the sections to come.


\section{Engines (search strategies)}
This section corresponds to the top left component of Figure~\ref{fig:architecture-proto-3}.
Below are the description of the various engines (these are called search strategies in
\ref{sec:exec-models-prolog}).

\subsection{The Stack engine}
The stack based engine works on a stack of triples \haskellConstruct{(s,goal,alts)}
corresponding to backtrack choice points, where: 
\begin{enumerate}
\item \haskellConstruct{s} is the substitution at that point,

\item \haskellConstruct{goal} is the outstanding goal and  

\item \haskellConstruct{alts} is a list of possible ways of extending the current proof to find a solution.   
\end{enumerate}
Each member of \haskellClass{alts} is a pair \haskellConstruct{(tp,u)} where: 
\begin{enumerate}
\item \haskellConstruct{tp} is a new goal that must be proved and 
\item \haskellConstruct{u} is a unifying substitution that must be combined with the substitution \haskellConstruct{s}.
\end{enumerate}

The list of relevant clauses at each step in the execution is produced by attempting to unify the head of the
current goal with a suitably renamed clause from the database.

Listing~\ref{tab:stackengineminiprlg} represents the Stack engine.

\begin{code-list}[H]
\begin{singlespace}
\inputminted[linenos, firstline=29, lastline=56]{haskell}{haskell-proto3-sudsy-woe.hs}
\end{singlespace}
\caption{Stack engine from \protect{\codeLibrary{Mini \progLang{Prolog}}} \cite{website:mini-prolog-hugs98}}
\label{tab:stackengineminiprlg}
\end{code-list}

\subsection{The Pure engine}
The pure engine works on \haskellConstruct{Prooftree}s. Each node in a \haskellConstruct{Prooftree} corresponds to:
\begin{enumerate}
\item
  Either, a solution to the current goal, represented by \haskellConstruct{Done s}, where \haskellConstruct{s} is
  the required substitution, or,
\item
  a choice between a number of trees \haskellConstruct{ts}, each corresponding to a proof of a
  \haskellConstruct{goal} of the current goal, represented by \haskellConstruct{Choice ts}.
  The proof tree corresponding to an unsolvable goal is \haskellConstruct{Choice []}. 
\end{enumerate}

Listing~\ref{tab:pureengineminiprlg} represents the Pure engine.

\begin{code-list}[H]
\begin{singlespace}
\inputminted[linenos, firstline=26, lastline=46]{haskell}{haskell-proto3-absurd-silicon.hs}
\end{singlespace}
\caption{Pure engine from \protect{\codeLibrary{Mini \progLang{Prolog}}} \cite{website:mini-prolog-hugs98}}
\label{tab:pureengineminiprlg}
\end{code-list}

\subsection{The Andorra engine}
This inference engine implements a variation of the Andorra Principle for logic programming adapted from
\cite{haridi1990kernel}.
The main difference here is to select a relatively deterministic goal and not the first one.
Upon selecting a goal:
\begin{enumerate}
\item
  for each goal g in the list of goals, calculate the resolvents that would result from selecting g, and
\item
  then choose a g which results in the lowest number of resolvents.
\end{enumerate}

If some g results in no resolvents then it is regarded as a failure. For instance,

\mint{prolog}|?- append(A,B,[1,2,3]),equals(1,2).)|

\progLang{Prolog} would not perform this optimization and would instead search and backtrack wastefully.
If some g results in a single resolvent then that g will get selected; by selecting and resolving g, bindings are
propagated sooner, and useless search can be avoided, since these bindings may prune away choices for other
clauses.
For example:

\mint{prolog}|?- append(A,B,[1,2,3]),B=[].|

\noindent Listing~\ref{tab:andorraengineminiprlg} represents the Andorra engine.

\begin{code-list}[H]
\begin{singlespace}
\inputminted[linenos, firstline=29, lastline=64]{haskell}{haskell-proto3-diatomic-unbank.hs}
\end{singlespace}
\caption{Andorra engine from \protect{\codeLibrary{Mini \progLang{Prolog}}} \cite{website:mini-prolog-hugs98}}
\label{tab:andorraengineminiprlg}
\end{code-list}


\section{Language}
\subsection{Current language}
Listing~\ref{tab:miniprlglang} shows the abstract syntax of \codeLibrary{Mini \progLang{Prolog}}
\cite{website:mini-prolog-hugs98}.
A term can either be a variable or a complex term with an atom as a head.
A \haskellConstruct{Clause} consists of a head of type \haskellConstruct{Term} and a body of type
\haskellConstruct{[Term]}.

\begin{code-list}[H]
\begin{singlespace}
  \inputminted[linenos, firstline=24, lastline=37]{haskell}{haskell-proto3-butter-chicken.hs}
\end{singlespace}
\caption{Current abstract syntax grammar in \protect{\codeLibrary{Mini \progLang{Prolog}}} \cite{website:mini-prolog-hugs98}}
\label{tab:miniprlglang}
\end{code-list}

\subsection{Language modification}
Listing~\ref{tab:miniprlglangmod} describes the necessary modifications required to adapt the language for monadic
unification.
This procedure consists of opening the language and adding the necessary instances for \codeLibrary{unification-fd}
\cite{unification-fd-lib} compatibility.

\begin{code-list}[H]
\begin{singlespace}
  \inputminted[linenos, firstline=64, lastline=88]{haskell}{haskell-proto3-uplift-apart.hs}
\end{singlespace}
\caption{Language modification}
\label{tab:miniprlglangmod}
\end{code-list}

\section{Unification from \protect{\codeLibrary{Mini \progLang{Prolog}}} \cite{website:mini-prolog-hugs98}}

\subsection{Current unification}

Listing~\ref{tab:miniprlgunif} describes the current unification mechanism which works on substitutions.
The \haskellConstruct{unify} function compares the two terms and returns a list of substitutions as with the base
implementations from the previous prototypes.

\begin{code-list}[H]
\begin{singlespace}
  \inputminted[linenos, firstline=67, lastline=95]{haskell}{haskell-proto3-pentyl-skater.hs}
\end{singlespace}
\caption{Current unification procedure in \protect{\codeLibrary{Mini \progLang{Prolog}}} \cite{website:mini-prolog-hugs98}}
\label{tab:miniprlgunif}
\end{code-list}

The shortcomings are translated from the language to the unification as it is based on basic pattern matching.

\subsection{Monadic unification}
Listing~\ref{tab:miniprlgmonadicunif} shows the procedure for monadic unification. Most components of the procedure remain similar to the ones
in previous prototypes.
\begin{code-list}[H]
\begin{singlespace}
  \inputminted[linenos, firstline=1, lastline=42]{haskell}{haskell-proto3-bevy-icebox.hs}
\end{singlespace}
\caption{Monadic unification}
\label{tab:miniprlgmonadicunif}
\end{code-list}

The major changes occur in \haskellConstruct{substConvertor} used to convert the result of monadic unification into
the original \haskellConstruct{Subst} form and \haskellConstruct{unify}.

\section{Summary}
Recapitulating, this chapter provided us with a working implementation of a \progLang{Prolog}-like interpreter with
the option to change the search strategy further proving the modularity and genericity of the language modification
and monadic unification procedure.

\end{document}
