\documentclass[thesis-solanki.tex]{files}


\begin{document}

\chapter{Prototype 3}{\label{proto3}}


\section{What is this chapter about}

-----------------------------------------------------------------------------

When two terms are to be unified we can use \ref{proto1} ,

term1 and term2 are matched and an assignment is the result 

now this may be a part of a query resolution procedure

to reach the point where two terms need to unified will happen through some sort of search strategy

and our approach is independent of that, and this prototype is a proof of concept to implementing query resolution using unification with
variable search strategy


\section{Unification}
The first, \yyy{"}{``}unification,\yyy{"}{''} regards how terms are matched and variables assigned to make terms match. \cite{website:prologunification}



\section{Resolution}
this where the complete procedure takes place after the query is passed along with the knowledge 

the resolver searches to create and a list of  goals and then tries to achieve each one.

\cite{website:prologresolution}

\cite{website:resolutionlogicwiki}




\section{Search strategies}
The base implementation used for this prototype  is \cite{website:mini-prolog-hugs98} and below are the search strategies 
\section{Stack Engine}
\begin{singlespace}
\inputminted[linenos]{haskell}{haskell-proto3-sudsy-woe.hs}
\end{singlespace}

\section{Pure Engine}
\begin{singlespace}
\inputminted[linenos]{haskell}{haskell-proto3-absurd-silicon.hs}
\end{singlespace}

\section{Andorra Engine}
\begin{singlespace}
\inputminted[linenos]{haskell}{haskell-proto3-diatomic-unbank.hs}
\end{singlespace}

\section{Current Unification}
\begin{singlespace}
  \inputminted[linenos]{haskell}{haskell-proto3-pentyl-skater.hs}
\end{singlespace}


\section{Syntax Modification}
\begin{singlespace}
  \inputminted[linenos]{haskell}{haskell-proto3-uplift-apart.hs}
\end{singlespace}

\section{Monadic Unification}
\begin{singlespace}
  \inputminted[linenos]{haskell}{haskell-proto3-bevy-icebox.hs}
\end{singlespace}


\section{Chapter Recapitulation}


\end{document}
