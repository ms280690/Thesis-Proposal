\documentclass[proposal.tex]{subfiles} 


\begin{document}

%-----------------------------------------------------------------------------
\section{Proposed Work}\label{sect:proposedWork}
%-----------------------------------------------------------------------------
\subsection{Current Work}
\paragraph{}
There have been several attempts at embedding \progLang{Prolog} into \progLang{Haskell} which are discussed below along with the 
shortcomings.

\begin{enumerate}
\item Very few embedded implementations exist which offer a perspective into the job at hand. One of the earliest implementations 
\cite{website:mini-prolog-hugs98} is for an older specification of \progLang{Haskell} called \progLang{Haskell 98} 
\texttt{\bfseries{hugs}}. It is more of a proof of concept providing a mechanism to include variable search strategies in order to produce 
a result. Another implementation \cite{website:takashi-workplace} based of it simplifies the notation to a list format. Nonetheless, both 
implementations lack simplicity and support for basic \progLang{Prolog} features such as \textit{cuts, fails, assert} among others.   

\begin{comment}
\item Only two embeddings exist, one of them is old and made for \texttt{\bfseries{hugs}} a functional programming system based on the 
\progLang{Haskell 98} specification. It is complex and also lacks a lot of \progLang{Prolog} like features including \textit{cuts, fails, 
assert} among others. The second one is based off the first one to make it simple but it loses the variable search strategy support which 
allows the programmer to choose the manner in which a solution is produced. 
\end{comment}

\item The papers that try to take the above further are also few in number and do not have any implementations with the proposed 
concepts. Moreover, none of them are complete and most lack many practical parts of \progLang{Prolog}.

\item Libraries, a few exist, most are old and are not currently maintained or updated. Many provide only a shell through which one has to do all the work, 
which is synonymous with the embeddings mentioned above. Some are far more feature rich than others that is with some practical \progLang{Prolog} 
concepts, but are not complete.

\item Moreover, none of the above have full list support that exist in \progLang{Prolog}.
\end{enumerate}

\par And as far as the idea of merging paradigms goes, it is not the main focus of this thesis and can be more of an "add-on". A handful of crossover hybrid 
languages based on \progLang{Haskell} exist, \progLang{Curry} \cite{website:curry} being the prominent one. Moving away from \progLang{Haskell} 
and exploring other languages from different paradigms, a respectable number of crossover implementations exist but again most of them have faded out.   

\par As discussed in the sections above, either an embedding or an integration approach is taken up for programming languages to work together. 
So, there is either a very shallow approach that does not utilize the constructs available in the host language and results in a mere translation of the 
characteristics, or the other is a fairly complex process which results in tackling the conflicting nature of different programming paradigms and 
languages, resulting in a toned-down  compromised language that takes advantages of neither paradigms. Mostly the trend is to build a library for 
extension to replicate the features as an add on.       

\subsection{Contributions}
\paragraph{}
Taking into consideration above, there is quite some room for improvement and additions. Moving onto what this thesis shall explore, first 
thing's first a complete, fully functional library which comes close to a \progLang{Prolog} like language and has practical abilities to 
carry out real-world tasks. They include predicates like \textit{cut, assert, fail, setOf, bagOf} among others. This would form the first 
stage of the implementation. Secondly, exploring aspects such as \textit{assert} and database capabilities. A third question to address is 
the accommodation of input and output, specifically dealing with the \textit{IO Monad} in \progLang{Haskell} with \progLang{Prolog} 
\textit{IO}. Moreover, \progLang{Prolog} is an untyped language which allows lists with elements of different types to be created. 
Something like this is not by default in \progLang{Haskell}. Hence syntactic support for the same is the next question to address. 
Furthermore, experimenting with how programs expressed with same declarative meaning differ operationally. Lastly, how would 
characteristics of hybrid languages fit into and play a role in an embedded setting.  

\subsection{Thesis Contributions}

\begin{enumerate}
\item Prototype 1 does 
flattening language
opening up the language (binding monad)
adding custom variables
monadic unification (stuff happens in a bubble)
rec type $\rightarrow$ non rec type $\rightarrow$ fix non rec type isomorphically == rec type




\item Prototype 2 does
extends current prolog-0.2.0.1
this is to show that we can plug out approach into existing implementation and things work

\item Prototype 3 does
variable search strategy
what ever method you do for searching at the point of unification you can do it with our approach

\item Prototype 4 does
how can io be squeezed into this model where whenever the resolver encounters an io operation it generates a thunk (sort of unsolved 
statement) which when executed would result in a side effect but till that point every thing is pure

\end{enumerate}




%\texttt{\bfseries{write a paragraph}}
\begin{comment}
\begin{enumerate}
\item Create a \textsc{Prolog} library which comes the closest to a complete \textsc{Prolog} like Language, not only the Declarative concepts but 
also a lot of Practical concepts like cuts, assert, fail, setOf, bagOf among others.

\item Do away with the shell approach. One must be able to write a program just the way one can write a \textsc{Haskell} program and the 
goal is passed as a parameter to the main function and the result is achieved by running it. 

\item Moreover, make an attempt to add some features which are related to Functional Logic Programming Languages, like narrowing 
and residuation among others.      

\item Furthermore, adding or adjusting \textsc{Prolog} IO into the \textsc{Haskell} IO Monad.

\item Translating a \textsc{Prolog} program into a \textsc{Haskell} program, for example two programs with same declarative meaning but different 
computational complexities in \textsc{Prolog} translated into \textsc{Haskell}. (Something similar is there in a paper called Higher Order 
Transformation of Logic Programs)

\item Tackling Translation of Tail Recursion in \textsc{Prolog} Programs. 

\item Able to recognize and work with all kinds of lists supported by \textsc{Prolog}.

\item Anything else ???????????????
\end{enumerate}
\end{comment}



\end{document}