\documentclass[thesis-solanki.tex]{subfiles}


\ifMain
\externaldocument{thesis-solanki}
\fi
\begin{document}

%-----------------------------------------------------------------------------
\chapter{Related Works}\label{chap:proposedWork}
%-----------------------------------------------------------------------------

\section{About this chapter}\label{sec:what-this-chapter:proposedWork}

This chapter discusses the current implementations and publications for embedding \progLang{Prolog} in languages
and specifically in \progLang{Haskell}. We take a look at their shortcomings and propose improvements.


\section{Existing work by others} \label{sec:existing-work-by-others}

There have been several attempts at embedding \progLang{Prolog} into \progLang{Haskell}, which are
discussed below, along with their shortcomings.

\begin{enumerate}
\item
  Very few embedded implementations exist which offer a starting point for the task of embedding.
  One of the earliest implementations \cite{website:mini-prolog-hugs98} is for an older specification of
  \progLang{Haskell} called \progLang{Haskell 98} \texttt{\bfseries{hugs}}.
  It is more of a proof of concept providing a mechanism to include variable search strategies in order to produce
  a result.
  Another implementation \cite{website:takashi-workplace} based of it simplifies the notation to a list format.
  Nonetheless, both implementations lack simplicity and support for basic \progLang{Prolog} features such as
  \prologConstruct{cuts},
  \prologConstruct{fails},
  \prologConstruct{assert}
  among others.

\begin{comment}
\item
  Only two embeddings exist, one of them is old and made for \texttt{\bfseries{hugs}} a functional programming
  system based on the \progLang{Haskell 98} specification.
  It is complex and also lacks a lot of \progLang{Prolog} like\eref{language-like} features including \textit{cuts, fails, assert}
  among others.
  The second one is based off the first one to make it simple but it loses the variable search strategy support
  which allows the programmer to choose the manner in which a solution is produced.
\end{comment}

\item
  The papers that try to take the above further are also few in number and do not have any implementations for the
  proposed concepts (see \cite{spivey1999embedding, seres1999algebra, seres2001algebra, spivey2000functional,
    seres2000optimisation}).
  Moreover, none of them are complete and most lack many practical parts of \progLang{Prolog}.

\item
  In the case of libraries, a few exist.
  Most are old and are not currently maintained or updated.
  Many provide only a shell through which one must do all the work, which is synonymous with
  the embeddings mentioned above.
  Some are more feature rich than others; that is, some have practical \progLang{Prolog} concepts,
  but are still not complete.

\end{enumerate}

As for the idea of merging paradigms goes, it is not the main focus of this thesis and can be more of an
``add-on''.
A handful of crossover hybrid languages based on \progLang{Haskell} exist,
\progLang{Curry} \cite{website:curry} being the prominent one.
Moving away from \progLang{Haskell} and exploring other languages from different paradigms, a respectable number of
crossover implementations exist but again most of them have faded out.


\section{Proposed improvements}\label{sec:things-fixed}

As discussed in Section~\ref{sec:existing-work-by-others},
either an embedding or an integration approach is taken up for programming languages to work together.
So, there is either a shallow approach that does not utilize the constructs available in the host language and
results in a translation of the characteristics, or the other is a fairly complex process which results in tackling
the conflicting nature of different programming paradigms and languages, resulting in a toned-down compromised
language that takes advantages of neither paradigms.
Mostly, the trend is towards the former.


Here is a discussion of problems mentioned above that this thesis tackles.
\begin{enumerate}
\item The \languageConstruct{eDSL} supports \prologConstruct{fails} and \prologConstruct{cuts}.

\item the implementation is more complete.

\item The implementation is not REPL based. You can write a program file and compile and run it like a normal \progLang{Haskell} file.
\end{enumerate}

\section{Chapter recapitulation}
This chapter reviewed current work and proposed improvements. The next chapter
provides the contributions of this thesis which are not necessarily built upon existing work.

\ifMain
\begin{scope}
  \nolinenumbers
  \enotesize
  \par
  \begin{singlespace}
  \setlength{\parskip}{12pt plus 2pt minus 1pt}
  \theendnotes
  \par
  \end{singlespace}
\end{scope}
\unbcbibliography{bibliography}
\fi

\end{document}
