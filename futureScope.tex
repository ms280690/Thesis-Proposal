\documentclass[thesis-solanki.tex]{subfiles}



\ifMain
\externaldocument{thesis-solanki}
\fi
\begin{document}

%----------------------------------------------------------------------------
\chapter{Future Scope}\label{chap:futureScope}


\section{About this chapter}

\noindent\rule{\textwidth}{0.5pt}
%-----------------------------------------------------------------------------


\begin{enumerate}

\item Quasi quoter to get something like,
\begin{minted}[linenos]{haskell}
[prolog|a(X) :- b(y)|]
\end{minted}
where \markWord{X} is a  \progLang{Prolog} variable and \markWord{y} is a \progLang{Haskell} variable injected into
the expression 


\item
  We already have variable search strategies, what if the query resolver could be instructed to use a particular
  search strategy to get the result.
\begin{minted}[linenos]{haskell}
queryResolver searcStrategy query knowledgeBase
\end{minted}


\item Add database operations

\item Multi type variable Language

\item Pure + IO Combined Language

\begin{minted}[linenos]{haskell}
data ResultWithIO typevariableforpureexpressions typevariableforioexpressions
	= PureConstructor_1 ....
	| PureConstructor_2 ....
	| IOContrcutor_1 .....
	| IOContructor_2 ...
	| ContructorWithBoth_1 .....
	| ContructorWithBoth_2 .....
	deriving(........)
\end{minted}

\item 

\end{enumerate}


\section{Chapter Recapitulation}

\ifMain
\begin{scope}
  \nolinenumbers
  \enotesize
  \par
  \begin{singlespace}
  \setlength{\parskip}{12pt plus 2pt minus 1pt}
  \theendnotes
  \par
  \end{singlespace}
\end{scope}
\unbcbibliography{bibliography}
\fi
\end{document}
