%
% proposal.tex
%
% Dissertation Proposal Template.
% School of Computing
% Clemson University
%
\documentclass[12pt]{ClemsonProposal}

% This is nice for source code listings
% \usepackage{listings}
\usepackage{comment}
\usepackage{subfiles}

\usepackage{color}
\usepackage[htt]{hyphenat}
\usepackage{hyperref}      
\usepackage{listing}
\usepackage{graphicx}
\usepackage{minted}

\usepackage{xr}
\externaldocument{background}
\externaldocument{conclusion}
\externaldocument{embedding}
\externaldocument{eplipl}
\externaldocument{introduction}
\externaldocument{miscrelcon}
\externaldocument{multiparadigm}
\externaldocument{prologin} 
\externaldocument{prologinhaskell}
\externaldocument{proposedwork}
\externaldocument{pwp}
\externaldocument{quasiquotation}
\externaldocument{relatedtermskeywords}
\externaldocument{relatedWork}
\externaldocument{ummcpp}
\externaldocument{workCompleted}

%\newcommand{\chapterReference}[1]{\textcolor{blue} {\nameref{#1} } }
\newcommand{\progLang}[1]{\textsc{#1}}
\newcommand{\chapterReference}[2]{\textcolor{blue}{\underline{\hyperref[#1]{#2}}}}

%\textcolor{blue}{\underline{\hyperref[sect:embedding]{Chapter 5}}}


% This is needed to include figures
% \usepackage{graphicx}

% Use any additional packages you might need


%
% Give values to the variables used in this document
%
\title{Embedding Programming Languages: \\* \textsc{Prolog} in \textsc{Haskell}}
\department{MCPS}
\documenttype{Master's Thesis}
\major{Computer Science}
\proposalday{\the \day}
\proposalmonth{\the \month}
\proposalyear{\the \year}
\author{Mehul Chandrakant Solanki \\230108015 \texttt{solanki@unbc.ca}}
\committeechair{Dr. David Casperson}
\committeememberone{Dr. Alex Aravind}
\committeemembertwo{Dr. Mark Shegelski}
%\committeememberthree{Member3 Name}

% Just in case you have more then 3 committee members
% \committeememberfour{Member4 Name}
% \committeememberfive{Member5 Name}
% \committeemembersix{Member6 Name}


%
% PDF Setup -- You should not need to change this
%
\hypersetup{
    colorlinks,
    linkcolor={black},
    citecolor={black},
    filecolor={black},
    urlcolor={black},
    pdftitle={\thetitle},
    pdfauthor={\theauthor},
    pdfsubject={\thedocumenttype},
    pdfkeywords={University of Northern British Columbia, \theauthor, \thedocumenttype},
    pdfstartpage={1}
}


%
% User-specified command definitions/redefinitions
%
%\newcommand{\cplusplus}{{\rm C\raise.5ex\hbox{\small ++}}}

%\lstloadlanguages{Haskell}

%\lstnewenvironment{code}
%    {\lstset{}%
%      {\csname lst@SetFirstLabel\endcsname}
 %   {\csname lst@SaveFirstLabel\endcsname}
 %   \lstset{
  %    basicstyle=\small\ttfamily,
   %   flexiblecolumns=false,
    %  basewidth={0.5em,0.45em},
     % literate={+}{{$+$}}1 {/}{{$/$}}1 {*}{{$*$}}1 {=}{{$=$}}1
      %         {>}{{$>$}}1 {<}{{$<$}}1 {\\}{{$\lambda$}}1
       %        {\\\\}{{\char`\\\char`\\}}1
        %       {->}{{$\rightarrow$}}2 {>=}{{$\geq$}}2 {<-}{{$\leftarrow$}}2
         %      {<=}{{$\leq$}}2 {=>}{{$\Rightarrow$}}2 
          %     {\ .}{{$\circ$}}2 {\ .\ }{{$\circ$}}2
           %    {>>}{{>>}}2 {>>=}{{>>=}}2
            %   {|}{{$\mid$}}1               
%    }}

\newcommand{\haskellscript}[2]{
\begin{itemize}
\item[]\lstinputlisting[caption=#2,label=#1]{#1.hs}
\end{itemize}
}

\setlength{\headheight}{35pt}
\begin{document}
%   ==========================================================================
%   Begin front matter (pages are numbered with roman numerals)
%   ==========================================================================
    \begin{frontmatter}
        \maketitle
		\tableofcontents
        \newpage
		\listoffigures
		\newpage
		\listoftables
		\newpage
        % Generate the abstract
        

\begin{unbcabstract}

This thesis focuses on combining the two most important and wide spread declarative programming paradigms, 
functional and logic programming. The proposed approach 
aims at adding logic programming features which are native to \progLang{Prolog} onto \progLang{Haskell}. We
develop extensions which replicate the target language by utilizing advanced features of the host language
for an efficient implementation.      

The thesis aims to provide insights into merging two declarative languages namely, \progLang{Haskell} and
\progLang{Prolog} by embedding the latter into the former and analyzing the results of doing so as the two languages have
conflicting characteristics.
The finished products will be something similar to a \textit{haskellised} \progLang{Prolog} which has logic programming-like capabilities.

\end{unbcabstract}


	\end{frontmatter}



%   ==========================================================================
%   Begin main matter (pages are numbered with arabic numerals)
%   ==========================================================================
    \doublespacing     % Text should be double spaced
    \pagestyle{fancy}  % Turn the nice header on for the rest of the document

    %
    % I use a file for every section.  Each of these corresponds to a file
    % with the specified name ending in '.tex' (e.g., introduction.tex).
    %
 
   \nocite{claessen2000typed,
   chin2003type,
   hinze1998prological,
   spivey2000functional,
   erwig2004escape,
   spivey1999embedding,
   seres1999algebra,
   seres2001algebra,
   seres2000optimisation,
   seres2001higher,
   spivey1999algebra,
   hoare1998unifying,
   gibbons2013unifying,
   friedman05reasoned,
   krishnamurthi2007programming,
   website:lambda-the-ultimate,
   website:lambda-the-ultimate-2,
   website:lambda-the-ultimate-3,
   website:takashi-workplace,
   website:mini-prolog-hugs98,
   website:logic-programming-haskell,
   website:stackoverflow,
   website:haskell-choice,
   website:prolog-steam,
   website:prolog-death,
   website:prolog-killer,
   somogyi1995logic,
   nanoprolog-lib,
   prolog-lib,
   cspm-To-Prolog-lib,
   prolog-graph-lib,
   prolog-graph-lib-lib,
   hswip-lib,
   logict-lib,
   logict2-lib,
   logic-classes-lib,proplogic-lib,
   cflp-lib,
   logic-grows-on-trees-lib,
   unification-fd-lib,
   cmu-lib,
   peg-lib,
   monadiccp-lib,
   monadiccp-gecode-lib,
   csp-lib,
   liquid-fix-point-lib,
   ramsey2003embedding,
   hudak1996building,
   benton2005embedded,
   Augustsson98cayenne--,
   barzilay2004foreign,
   pinto2003dot,
   reppy2006application,
   ait1999warren,
   castorc++,
   gnuprolog,
   jlog,
   jscriptlog,
   lang2001embedding,
   audklangembedd,
   swipembedd,
   quintusprolog,
   yieldprolog,
   komorowski1982qlog,
   robinson1982loglisp,
   robinson1980loglisp,
   ummlisp,
   racklog}
   
   
   
   \nocite{DBLP:conf/utp/2006,
   DBLP:conf/utp/2008,
   DBLP:conf/utp/2010,
   DBLP:conf/utp/2012,
   wikiquasi,
   haskellquasi,
   hughes1989functional,
   spivey1995introduction,
   Krishnamurthi:2008:TPL:1480828.1480846,
   wadler1992comprehending,
   website:logicprogexamplehaskell,
   wikiprolog,
   website:scala,
   website:virgil,
   website:closwiki,
   website:mercury,
   website:curry,
   lloyd1999programming:escher,
   website:alf,
   website:babel,
   moreno1992logic,
   moreno1988babel,
   website:ciao,
   website:life,
   bert1987lpg,
   website:nue-prolog,
   website:oz-mozart,
   website:oz-mozart,
   website:toy,
   website:lambda-prolog,
   website:visual-prolog,
   website:tablog,
   website:haskellwiki}
    
	\subfile{introduction}
	\clearpage	
	
	\subfile{background}
	\clearpage	
	
	\subfile{proposedWork}
	\clearpage
					
	\subfile{embedding}
	\clearpage
			
	\subfile{multiparadigm}
	\clearpage	
  	
  	\subfile{relatedWork}
	\clearpage
	
	\subfile{eplipl}	
	\clearpage
		
	\subfile{prologin}
	\clearpage
			
	\subfile{prologinhaskell}
	\clearpage
		
	\subfile{ummcpp}
	\clearpage
			
	\subfile{flpl}	
	\clearpage
		
	\subfile{quasiquotation}
	\clearpage
		
	\subfile{relatedtermskeywords}
	\clearpage
		
	\subfile{hwh}
	\clearpage
		
	\subfile{pwp}
	\clearpage
		
	\subfile{miscrelcon}
	\clearpage
  	
  	\subfile{proto1}
  	\clearpage
  	
  	\subfile{proto2.1}
  	\clearpage
  	
  	\subfile{proto2.2}
  	\clearpage
  	
  	\subfile{proto3}
  	\clearpage
  	
  	\subfile{proto4}
  	\clearpage
  	
  	\subfile{workCompleted}
	\clearpage	
	
	\subfile{results}
	\clearpage
			
  	\subfile{conclusion}
	\clearpage


%   ==========================================================================
%   Wrap up the document with the Bibliography (looks for the specified .bib)
%   ==========================================================================
   	

	\makebibliography{bibliography}
\end{document}
