\documentclass[proposal.tex]{subfiles} 


\begin{document}

\section{Prototype 2.1}{\label{proto2.1}}

\subsection{About this chapter}
This chapter attempts to infuse the generic methodology from \ref{proto1} in a current \progLang{Prolog} implementation \cite{prolog-lib}
and make the unification "monadic".

\subsection{How prolog-0.2.0.1 works}

The original syntax used by the library,

\begin{minted}[linenos]{haskell}
data Term = Struct Atom [Term]
          | Var VariableName
          | Wildcard -- Don't cares 
          | Cut Int
      deriving (Eq, Data, Typeable)

data Clause = Clause { lhs :: Term, rhs_ :: [Goal] }
            | ClauseFn { lhs :: Term, fn :: [Term] -> [Goal] }
      deriving (Data, Typeable)

rhs :: Clause -> [Term] -> [Goal]      
rhs (Clause   _ rhs) = const rhs
rhs (ClauseFn _ fn ) = fn

data VariableName = VariableName Int String
      deriving (Eq, Data, Typeable, Ord)

type Atom         = String
type Goal         = Term
type Program      = [Clause]
\end{minted} 

The above language suffers from most of the problems discussed in the previous chapter.

The above is used to construct \progLang{Prolog} "terms" which are of a "single type".   

A database is used to store the terms which can then be used to resolve a query.

An interpreter to solve a query and lastly the unifier,

\begin{minted}[linenos]{haskell}
unify, unify_with_occurs_check :: MonadPlus m => Term -> Term 
-> m Unifier

unify = fix unify'

unify_with_occurs_check =
   fix $ \self t1 t2 -> if (t1 `occursIn` t2 || t2 `occursIn` t1)
                           then fail "occurs check"
                           else unify' self t1 t2
 where
   occursIn t = everything (||) (mkQ False (==t))

unify' :: MonadPlus m => (Term -> Term -> m Unifier) -> Term -> 
Term -> m [(VariableName, Term)]

-- If either of the terms are don't cares then no unifiers exist
unify' _ Wildcard _ = return []
unify' _ _ Wildcard = return []

-- If one is a variable then equate the term to its value which 
-- forms the unifier
unify' _ (Var v) t  = return [(v,t)]
unify' _ t (Var v)  = return [(v,t)]

-- Match the names and the length of their parameter list and 
-- then match the elements of list one by one. 
unify' self (Struct a1 ts1) (Struct a2 ts2) 
            | a1 == a2 && same length ts1 ts2 = 
            unifyList self (zip ts1 ts2)

unify' _ _ _ = mzero

same :: Eq b => (a -> b) -> a -> a -> Bool
same f x y = f x == f y

-- Match the elements of each of the tuples in the list. 
unifyList :: Monad m => (Term -> Term -> m Unifier) -> 
[(Term, Term)] -> m Unifier
unifyList _ [] = return []
unifyList unify ((x,y):xys) = do
   u  <- unify x y
   u' <- unifyList unify (Prelude.map (both (apply u)) xys)
   return (u++u')
\end{minted}  


There are a few other components such as the REPL, Parser. 


\subsection{What we do in this prototype?}


\subsection{Current implementation (prolog-0.2.0.1)}


\subsection{Modifications}


\subsection{Results}



\end{document}