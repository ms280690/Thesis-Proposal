
\documentclass[thesis-solanki.tex]{subfiles}


\ifMain
\externaldocument{thesis-solanki}
\fi
\begin{document}
%----------------------------------------------------------------------------
\chapter{Introduction}\label{chap:introduction}
%----------------------------------------------------------------------------


\section{What is this chapter about}

This chapter introduces the scope of the thesis along with the preliminary arguments.


-----------------------------------------------------------------------------


\section{Beginnings}

Programming has become an integral part of working and interacting with computers and day by day more and more complex
problems are being tackled using the power of programming technologies.

A programming language must not only provide an easy to use environment but also adaptability towards the problem domain.

Over the last decade the declarative style of programming has gained popularity.
The methodologies that have stood out are the functional and logical approaches.\endnote{%
  \david{What style guide are you using to determine capitalization?}\newline
  \mehul{none at the moment but I will change certain capitals to small}
}
The former is based on functions and lambda calculus, while the latter is based on Horn clause logic.
Each of them has its own advantages and flaws.
How does one choose which approach to adopt?
Perhaps one does not need to choose!
This document looks at the attempts, improvements and future possibilities of uniting \progLang{Haskell}, a purely
functional programming language and \progLang{Prolog}, a logical programming language so that one is not forced to
choose. The task at hand involves replicating \progLang{Prolog} like
features in \progLang{Haskell} such as unification and a single typed system.
The thesis aims at leveraging the features of the host language in order to incorporate logic programming features
resulting in extending \progLang{Haskell} with capabilities like unification.
We achieve this by adopting various aspects from approaches related to merging paradigms and embedding techniques
for programming languages.
This results in a hybrid approach which provides a library taking advantage of the host language features.




\section{Thesis Statement}

The thesis aims to provide insights into merging two declarative languages namely, \progLang{Haskell} and
\progLang{Prolog} by embedding the latter into the former and analyzing the result of doing so as they have conflicting
characteristics.
The finished product will be something like a \textit{haskellised} \progLang{Prolog} which has logical programming like
capabilities.


%----------------------------------------------------------------------------
\section{Problem Statement}
%----------------------------------------------------------------------------

Over the years the development of programming languages has become more and more rapid.
Today the number of is in the thousands and counting \cite{website:timelineproglangwiki,
  website:historyofproglang}.
The successors attempt to introduce new concepts and features to simplify the process of coding a solution and
assist the programmer by lessening the burden of carrying out standard tasks and procedures.
A new one tries to capture the best of the old; learn from the mistakes, add new concepts and move on; which seems
to be good enough from an evolutionary perspective.
But all is not that straight forward when shifting from one language to another.
There are costs and incompatibilities to look at.
A language might be simple to use and provide better performance than its predecessor but not always be worth the
switch.
Another approach would be to replicate target features exhibited by a language in the present one to avoid the
hassle of jumping between the two.
Commonly this results in an embedded language or a foreign function interface.
A mixture of these ideologies results in a multi-paradigm / merged programming language.
We try to encapsulated both the approaches of embedding and merging to develop a hybrid approach.\endnote{%
  Removed a \P{} that followed, as the one that currently follows seems to say the same thing.
  That \P{} was
  \begin{quote}
    \slshape
    The adoption of \progLang{Prolog} has been difficult since procedural languages were making
    a mark when it was released. 
  \end{quote}
}

%An embedded language provides a lot of shortcuts

\begin{comment}
\progLang{Prolog} has a similar story. It was born in an era where procedural programming had made everyone notice their presence. Talking about
competition, it was against something radical; the \progLang{C} programming language. The languages \progLang{C} has influenced is off the chart and
so is the performance. It had paved the way for structured procedural programming and had given birth to the Unix operating system. Though the
original version of \progLang{Prolog} has given rise to a large number of different flavours but a few drawbacks remain through the bloodline and as a
result it did become the first choice. Some basic requirements such as modules are not provided by all compilers. To make it do real world stuff, a set of
practical features are pushed in now and then which results in the loss of the purely declarative charm. The problem is that \progLang{Prolog} is fading
away, \cite{website:prolog-steam,website:prolog-death,website:prolog-killer}, not many people use it and most of the times when it is used, the variant
is usually \textit{practical} \progLang{Prolog} and the area being academia. It is not used for building large programs \cite{wikiprolog,somogyi1995
logic,website:prolog1000db}. But there are a lot of good things about \progLang{Prolog} that should not die away. Moreover, \progLang{Prolog} is ideal
for search problems.
\end{comment}

\progLang{Prolog} is a language that has a hard time being adopted.
Born in an era where procedural languages were receiving a lot of attention, it suffered from competing against another
new kid on the block: \progLang{C}.

Some of the problems were due to its own limitations.
Basic features like modules were not provided by all compilers.
Practical features for real world problems were added in an ad hoc way resulting in the loss of its purely declarative
nature.
Some say that \progLang{Prolog} is fading away, \cite{website:prolog-steam,website:prolog-death,website:prolog-killer}.
It is apparently not used for building large programs \cite{wikiprolog,somogyi1995logic,website:prolog1000db}.
However there are a lot of good things about Prolog: it is ideal for search problems; it has a simple syntax, and a
strong underlying theory.
It is a language that should not die away.

So the question is how to have all the good qualities of \textsc{Prolog} without actually using \progLang{Prolog}?

One idea is to make \progLang{Prolog} an add-on to another language which is widely used and in demand.
Here the choice is \progLang{Haskell}; as both the languages are declarative they share a common background which can
help to blend the two.

Putting all of the above together, Domain Specific Languages\endnote{%
  \david{This footnote changed 2016-01-08.
    The point I want to make here is that you are talking about Domain Specific Languages as if you expect the
    reader to know what they are.  Should she?
    You have discussed general purpose languages somewhat, but you haven't said what you mean by ``Domain Specific
    Language'', nor have you given an example.
  }%
}
are pretty good in doing what they are designed to do, but nothing else, resulting in choosing a different language
every time.
On the other hand, a general purpose language can be used for solving a wide variety of problems but often the
programmer ends up writing some code dictated by the language rather than by the problem.

Generally speaking, programming languages with a wide scope over problem domains do not provide bespoke support for
accomplishing  mundane tasks.
Approaching towards the solution can be complicated and tiresome, but the programming language in question acts as the
master key.
A general purpose language, as the name suggests,
provides a general set of tools to cover many
problem domains.
The downside is that such environments\endnote{%
  Do you mean languages?  If not, what environments are you talking about?
  \mehul{the programming environment that the embedded language provides as a library}
}
lack the complete feature set which can make problem solving difficult.

Flipping the coin to the other side, we see, the more specific the language is to the problem domain, the easier it
is to solve the problem since the solution need not be moulded according to the capability  of the
language.
For example, a problem with a naturally recursive solution cannot take advantage of tail recursion in many
imperative languages.
Many domains require the system to be mutation free, but must deal with uncontrolled side-effects and so
on.\endnote{%
  \david{This example isn't as clear to me as the one in the last sentence.}
  \mehul{how do I say that mathematics can be translated to declarative easily in declarative paradigm}
}


The solution is to develop a programming language with a split personality, in our case, sometimes functional,
sometimes logical and sometimes both.
Depending upon the problem, the language shapes itself accordingly and exhibits the desired characteristics.
The ideal situation is a language with a rich feature set and the ability to mould itself according to the problem.
A language with the ability to take the appropriate skill set and provide them to the programmer will reduce
the hassle of jumping between languages or forcibly trying to solve a problem according within the limitations of a paradigm.

The subject in question here is \progLang{Haskell} and the split personality being \progLang{Prolog}. How far can \progLang{Haskell} be 
pushed to
don the avatar of \progLang{Prolog}?
This is the million dollar question.

A \progLang{Haskell} with \progLang{Prolog}-like\eref{language-like} features will result in a set of characteristics which are from
both the declarative paradigms.
%----------------------------------------------------------------------------------------------------------------------

%The issue being discussed here is that often when a problem is to solved using a given language, it has to be moulded %according to the capability the
%language can provide.
This can be achieved in two ways:
\begin{description}
\item [Embedding (\chapterReference{chap:embedding}{Chapter 4}):]

  This approach involves translating a complete language into the host language as an extension such as a
  library or module.
  The result is very shallow
  as all the positives as well as the negatives are brought into the host language.
  The negatives mentioned being, that languages from different paradigms usually have conflicting characteristics
  and result in inconsistent properties of the resulting embedding.
  Examples and further discussion on the same \yyy{is}{are}\endnote{%
      The word ``Examples'' is plural.
}
    provided in the chapters to come.

\item [Paradigm Integration (\chapterReference{chap:multiparadigm}{Chapter 5}):]

  This approach goes much deeper as it does not involve a direct translation.
  An attempt is made by taking a particular characteristic of a language and merging it with the characteristic of
  the host language in order to eliminate conflicts resulting in a multiparadigm language.
  It is more of weaving the two languages into one tight package with the best of both and maybe even the worst of
  both.
\end{description}


%----------------------------------------------------------------------------
%\section{Research Approach and Contributions}
%----------------------------------------------------------------------------
%



%\subsection{Contributions}
%----------------------------------------------------------------------------

%----------------------------------------------------------------------------


%----------------------------------------------------------------------------
\section{Thesis Organization}

The next chapter, \chapterReference{chap:proposedWork}{Chapter 2} provides details about the shortcomings of the
previous works and an approach for improvements.
\chapterReference{chap:background}{Chapter 3}, the background talks about the programming paradigms and languages
in general and the ones in question.
Then we look at the question from different angles namely, \chapterReference{chap:embedding}{Chapter 4},  Embedding
a Programming Language into another Programming Language and  \chapterReference{chap:embedding}{Chapter 5}, Multi
Paradigm Languages (Functional Logic Languages).
Some of the indirectly related content \chapterReference{chap:relatedWork}{Chapter 6} and finishing off with the
\chapterReference{chap:conclusion}{Chapter 7}, the expected outcomes.

%\section{The Plan}

\section{Chapter Recapitulation}

\ifMain
\begin{scope}
  \nolinenumbers
  \enotesize
  \par
  \begin{singlespace}
  \setlength{\parskip}{12pt plus 2pt minus 1pt}
  \theendnotes
  \par
  \end{singlespace}
\end{scope}
\unbcbibliography{bibliography}
\fi

\end{document}
